\documentclass{ctexbook}
\usepackage{imakeidx}
\makeindex[title=索引,intoc]
\usepackage[]{subcaption}
\usepackage[]{bicaption}
\usepackage{graphicx}
\usepackage[]{amsmath}
\usepackage[]{amssymb}
\usepackage[]{hyperref}
\hypersetup{colorlinks, linkcolor=blue}
\usepackage[]{bigstrut}
\usepackage[dvipsnames,svgnames]{xcolor}
\usepackage[]{multicol}
\usepackage{mhchem}
\usepackage[tone,extra,safe]{tipa}
\usepackage[normalem]{ulem}
% \usepackage[]{ninecolors}
% \NineColors{saturation=low}
\usepackage{tabularray}
\UseTblrLibrary{booktabs}
% \usepackage[]{paracol}
\usepackage[]{lipsum}
\usepackage[]{zhlipsum}
\usepackage[]{sidenotes}
\usepackage[]{subfiles}
\usepackage[]{siunitx}
\usepackage[]{metalogo}
\usepackage[]{tcolorbox}
\tcbuselibrary{skins}
\tcbuselibrary{breakable}
\tcbuselibrary{documentation}
\usepackage{geometry} 
\geometry{left=2.5cm,right=2.5cm,top=2.5cm,bottom=2.5cm}

\newtcolorbox[auto counter,number within=section]{sample}[2][]{%
arc is angular,breakable,parbox=false,before upper=\par,before lower=\par,drop shadow,skin=bicolor,colbacklower=white,colback=blue!8!white,colframe=blue!46!black, arc=1mm,fonttitle=\bfseries, title={\heiti Sample}~\thetcbcounter: #2,#1}

\newtcolorbox[auto counter,number within=section]{note}[2][]{%
breakable,parbox=false,before upper=\par,before lower=\par,drop shadow,skin={spartan,bicolor},colbacklower=white,colback=red!3!white,colframe=brown!46!black,fonttitle=\bfseries, title={\heiti Note}~\thetcbcounter: #2,#1}

\newtcolorbox[auto counter,number within=section]{task}[2][]{%
colback=Emerald!10,colframe=cyan!40!black,fonttitle=\bfseries,title={\heiti Task}~\thetcbcounter: #2,#1}

\newtcolorbox[auto counter,number within=subsection]{eg}[2][]{%
attach title to upper,coltitle=black,arc=0mm,top=0mm,boxrule=0pt,beforeafter skip balanced=0pt,colback=Thistle!80!red!20!white,colframe=white,fonttitle=\upshape\bfseries,fontupper=\itshape,title={\heiti E.g.}~\thetcbcounter:\ #2,#1}

  
\usepackage[]{adforn}
% \renewcommand{\chaptername}{第 \thechapter{} 章}
% \newcommand{\exercisename}{Exercice}
\newenvironment{problemset}[1][\chaptername~\;Exercice]{
  \begin{center}
    \phantomsection\addcontentsline{toc}{section}{\texorpdfstring{chapter\;Exercice}{Exercice}}
    % \markboth{#1}{\rightmark}
    \markright{#1}
    \textcolor{black}{\Large\bfseries\adftripleflourishleft~#1~\adftripleflourishright}
  \end{center}
  \begin{enumerate}}{
  \end{enumerate}}


\begin{document}


\begin{titlepage}
    \vspace*{\stretch{1}}
    \begin{center}
      {\huge\bfseries Abstract Writing for Medical \\\vspace{3pt} Research Papers}\\[3ex]
      {\huge\bfseries 医学论著英语摘要写作}\\[6.5ex]
      {\large\bfseries Qi Hui, Chen Feina, Guo Haiyan}           \\
      \vspace{4ex}
    %   Thesis  submitted to                    \\[5pt]
      \textit{Fujian Medical University}                \\[2cm]

    %   \includegraphics[width=0.4\textwidth]{fig/fjmulogo.jpg}

    %   in partial fulfilment for the award of the degree of \\[2cm]
    %   \textsc{\Large Doctor of Philosophy}    \\[2ex]
    %   \textsc{\large Mathematics}             \\[12ex]
      \vfill
      Department of Arts and Sciences\\
    %   Address                                 \\
      \vfill
      \date{}
    \end{center}
    \vspace{\stretch{2}}
\end{titlepage}

\frontmatter
\chapter*{译者序}

本书是由福建医科大学文理艺术学院的齐晖、陈菲娜、郭海燕老师为主编,陈晶为学术秘书,交由复旦大学出版社出版,专供福医大学生使用的英语摘要写作教科书。我本人也是三位老师的学生,纵然课堂生动有趣、干货满满,但苦于同校前辈制作的扫描件观感不佳,笔记整理不便,译者决心要进行文字重排处理。其中自觉原书排版不善之处,皆进行重新编排,以符合译者审美。

原书通本以英文编写,编者似乎意图借此提升我等英语阅读水平,奈何文本中穿插语言学专有名词,初学时疲于翻译、苦不堪言。此外,期末复习期间,全英文本并不利于提升复习效率,故译者对主要文本进行翻译,对照复习。本套重置本将基于该译本进行整理,包含三种排版样式——原文重排版、双语对照版、译文版。以上三个版本请学弟学妹们按需取用,20届学长祝各位期末考试顺利。

本书为个人翻译作品,若发现纰漏,请在GitHub上提交\verb|issue|。在此声明,此书仅供学习交流使用,请勿用于商业用途或其他领域。

\vfill

\begin{flushright}
    neurocane\\
    2023/7/7
\end{flushright}

\newpage

\section*{编译环境}

\begin{itemize}
    \item \textbf{操作系统}: Windows
    \item \textbf{语言}: \LaTeX
    \item \textbf{编译环境}: \XeLaTeX
    \item \textbf{TeX Live版本}: TeX Live 2022
\end{itemize}

\tableofcontents

\mainmatter

\newgeometry{includeall,left=2cm,right=1cm,top=2.5cm,bottom=2.5cm,textwidth=12cm,marginparwidth=5cm}

\chapter{Overview of Abstracts}\label{chapter1}
\section{Definition of an Abstract}

The American Psychological Association (APA) Style (2010) states that an abstract is a brief, comprehensive summary of the content of an article. According to the American National Standards Institute (1979), an abstract is an abbreviated accurate representation of the content of a document, preferably prepared by its author(s) for publication with it. In general, an abstract is a concise, accurate and comprehensive statement of the content of an article. It is original rather than excerpted.

\section{Importance and Functions of an Abstract}

An abstract is a distinct genre, and to some extent plays a pivotal role in academic readingand writing. In the era of information explosion, an enormous number of new publications are produced in the academic community each day. There is no practical way for every reader to get access to every new article, or to read every new publication even if it is accessible. Theabstracts published online, which are concise and comprehensive, can be obtained easily andquickly. Abstract reading, then, may be a useful starting point of any academic reading and writing. In this sense, an abstract is the most read part of an article.\sidenote{劳仑衣普桑,认至将指点效则机,最你更枝。想极整月正进好志次回总般,段然取向使张规军证回,世市总李率英茄持伴。}

An abstract has at least three functions (Huckin, 2001). First, it serves as a stand-alone mini text, giving readers a quick summary of a study's objectives, methodology, findings and conclusions, which are the major components of abstracts. Second, it serves as a screening device, and gives readers an adequate view on whether the full-length article is of great value to their needs and worth further reading. A good abstract, to some extent, increases the chance of being cited or referenced. Third, for those readers who do opt to read the article as a whole, the abetract serves as a preview, creating an interpretive frame that can guide reading.

\section{Types of Abstracts}

Generally, abstracts fall into two categories, indicative and informative, depending on the type of information they convey. A typical distinction between them is that the indicative abstract, viewed as the outline of the paper, is usually shorter and simpler, while the informative abstract, viewed as the summary of the paper is usually longer and more thoroagh. 

These two types of abstracts also differ in the components they contain. Indicative abstracts often include the purpose, scope, and methods of the report or study, but seldom include theresults or conclusions. Reading indicative abstracts could not substitute reading the paper, because not all the crucial components are covered. It is more widely used in social science papers. On the other hand, informative abstracts usually include all the crucial components of the study, such as the background, purpose, methods, results, and conclusions. It is the type of abstracts widely used in medical field. In this book, we focus on the writing of informative abstracts. The abstracts referred to in the following chapters are informative abstracts.

\section{Types of Informative Abstracts}

There are two types of informative abstracts, structured abstracts and unstructured abstracts.

Where a heading or label is used at the beginning of the text in each section, it is a structured abstract. Each section is usually written in a separate paragraph, but sometimes sections are written in a sole or continuous paragraph. Headings might be background, objectives, methods, results, conclusions, and so on. They vary according to the criteria set by different journals. Structured abstracts appear to be favored by medically-relevant publications.

Where no heading or label is used to indicate different parts of an abstract, it is an unstructured abstract. It is always a sole paragraph. The major difference between the two types of abstracts lies in whether there are headings or not. In an unstructured abstract the content and sequence of the items are written as it is in the structured one.

Journals mandate which style should be used, so check the author guidelines if you' re not sure. If it is not mentioned, keep an eye out for the type of abstracts preferable in the journals where you are willing to have your paper submitted and published. Write your abstracts in the style which dominates.

% \tcbset{colback=blue!8!white,colframe=blue!46!black,
% arc=1mm}
% \begin{tcolorbox}[arc is angular,skin=bicolor,colbacklower=white,title={\heiti 例题}]
% 这是一个 \textbf{tcolorbox}。
% \tcblower
% 这个是下部
% \end{tcolorbox}


\begin{sample}[label={myautocounter}]{\heiti}
  \textbf{BACKGROUND} 
  
  In patients with acute heart failure, early intervention with an intravenous vasodilator has been proposed as a therapeutic goal to reduce cardiac-wall stress and, potentially, myocardialinjury, thereby favorably affecting patients' long-term prognosis.

  \textbf{METHODS} 
  
  In this double-blind trial, we randomly assigned 2, 157 patients with acute heart failure to receive a continuous intravenous infusion of either ularitide at a dose of 15 ng per kilogram of body weight per minute or matching placebo for 48 hours, in addition to accepted therapy.
  Treatment was initiated a median of 6 hours after the initial clinical evaluation. The coprimaryd outcomes were death from cardiovascular causes during a median follow-up of 15 months and a hierarchical composite end point that evaluated the initial 48-hour clinical course.
  
  \textbf{RESULTS} 
  
  Death from cardiovascular causes occurred in 236 patients in the ularitide group and 225 patients in the placebo group (21.7\% vs. 21.0\%; hazard ratio, 1.03;96\% confidence interval, 0.85 to 1.25; P=0.75). In the intention-to-treat analysis, there was no significant between-group difference with respect to the hierarchical composite outcome. The ularitide group had greater reductions in systolic blood pressure and in levels of N-terminal pro-brain natriuretic peptide than the placebo group. However, changes in cardiac troponin T levels during the infusion did not differ between the two groups in the 55\% of patients with paired data.

  \textbf{CONCLUSIONS} 
  
  In patients with acute heart failure, ularitide exerted favorable physiological effects (without affecting cardiac troponin levels), but short-term treatment did not affect a clinical composite end point or reduce long-term cardiovascular mortality.

  \begin{flushright}
    ---Effect of Ularitide on Cardiovascular Mortality in Acute Heart Failure.

    \emph{New England Journal of Medicine (2017)}
  \end{flushright}

  
\end{sample}

\begin{sample}[label={myautocounter}]{\heiti}
  \textbf{OBJECTIVE} 
  
  To evaluate the association between the parameters of 24-hour multichannel intraluminal impedance (MII)-pH monitoring and the symptoms or quality of life (QoL) in laryngopharyngeal reflux (LPR) patients.

  \textbf{DESIGN} 
  
  Prospective cohort study without controls.

  \textbf{SETTING} 
  
  University teaching hospital.

  \textbf{METHODS} 
  
  Forty-five LPR patients were selected from subjects who underwent 24-hour MII-pH monitoring and were diagnosed with LPR from September 2014 to May 2015. Reflux Symptom Index (RSI), Health-related Quality of Life (HRQoL), Short Form 12 (SF-12) Survey questionnaires were surveyed. Spearman's correlation was used to analyse the association between the symptoms or QoL and 24-hour MII-pH monitoring.

  \textbf{RESULTS}

  Most parameters in 24-hour MII-pH monitoring showed weak or no correlation with RSI, HRQoL and SF-12. Only number of non-acid reflux events that reached the larynx and pharynx (LPR-non-acid) and number of total reflux events that reached the larynx and pharynx (LPR-total) parameters showed strong correlation with heartburn in RSI (R=0.520, $P<0.001$, R=0.478, $P=0.001$, respectively). Multiple regression analysis showed that there was only one significant regression coefficient between LPR-non-acid and voice/hoarseness portion of HRQoL (b=1.719, $P=0.022$).

  \textbf{CONCLUSION} 

  Most parameters of 24-hour MII-pH monitoring did not reflect subjective symptoms or QoL in patients with LPR.

  \begin{flushright}
    ---Association between 24-hour combined multichannel intraluminal impedance-pH monitoring and symptoms or quality of life in patients with laryngopharyngeal reflux. 
    
    \emph{Clinical Otolaryngology (2017)}
  \end{flushright}

  
\end{sample}

\begin{sample}[label={myautocounter}]{\heiti}
  Due to the high incidence of recurrent squamous cell carcinoma of the head and neck andd the toxicity profile of current salvage regimens, there is a need for tolerable and effective treatment options. We performed a retrospective matched case series to report our experience with recurrent high-risk patients who received capecitabine (CAP) therapy in the adjuvant setting after salvage therapy. The 5-year recurrence-free survival rates for the CAP and control cohorts were 54\% (95\% CI, 0.27\%--0.75\%) and 27\% (95\% CI, 0.09\%--0.50\%), respectively.Multivariable Cox modeling showed a significant improvement in recurrence-free survival in thed CAP cohort (hazard ratio, 0.19; 95\% CI, 0.04--0.92; $P=.0 392$). While this was a respective analysis that could not control for all variables, these exploratory findings offer insights that may inform a prospective study to determine CAP efficacy.
  
  \begin{flushright}
    ---Capecitabine after Surgical Salvage in Recurrent Squamous Cell Carcinoma of Head and Neck. 
    
    \emph{Otolaryngology---Head \& Neck Surgery (2017)}
  \end{flushright}


\end{sample}

\begin{note}[label={myautocounter}]{\heiti Corpus used for this book}
  The data used and analyzed in this book are from a custom-built corpus with 1.15 million tokens of medical research article (RA) abstracts. The discipline of medicine is divided into 18 sub-disciplines, and RA abstracts from 2 to 3 leading journals are randomly retrieved in each sub-discipline with relatively similar number of texts for each sub-discipline (\autoref{tab:Sub-disciplines and journals in each sub-discipline}). The journals selected are all with relatively high impact factors.

\end{note}

{\small
  \begin{longtblr}[
      caption = {Sub-disciplines and journals in each sub-discipline},
      label = {tab:Sub-disciplines and journals in each sub-discipline},
  ]{
      width = \textwidth,
      colspec = {X[1,l,h]  X[2,l,h]},
      rowhead = 1, rowfoot = 0, % 每个分页里表头表尾的数量
      % row{odd} = {blue8}, 
      row{even} = {LemonChiffon},
      column{2} = {font=\itshape}
  }
    
    \toprule
    Sub-discipline & Journal \\
    \midrule

    Anesthesiology & {British Journal of Anaesthesia \\ Anesthesiology \\ Anesthesia and Analgesia}\\
    Dermatology & {Journal of American Academy of Dermatology \\ Giornale Italiano di Dermatologia e Venereologia}\\
    Emergency Medicine & {Annals of Emergency Medicine \\ Internal and Emergency Medicine \\ Academic Emergency Medicine}\\
    Geriatrics & {Neurobiology of Aging \\ Aging Cell \\ Age and Ageing}\\
    Internal Medicine & {The New England Journal of Medicine \\ The Lancet \\ JAMA-Journal of the American Medical Association}\\
    Medical Imaging & {The Journal of Nuclear Medicine \\ Investigative Radiology \\ Radiology}\\
    Medical Laboratory & {Clinical Chemistry and Laboratory Medicine \\ Clinical Biochemistry}\\
    Neurology & {The Lancet Neurology \\ Annals of Neurology}\\
    Obstetrics and Gynecology & {Obstetrics \& Gynecology \\ American Journal of Obstetrics \& Gynecology \\ An International Journal of Obstetrics \& Gynecology}\\
    Oncology & {Journal of Clinical Oncology \\ The lancet Oncology}\\
    Ophthalmology & {Ophthalmology \\ American Journal of Ophthalmology \\ Archives of Ophthalmology}\\
    Otolaryngology (ENT) & {Head \& Neck \\ Clinical Otolaryngology \\ Otolaryngology---Head \& Neck Surgery}\\
    Pain Medicine & {The Clinical Journal of Pain \\ Pain Medicine \\ Regional Anesthesia and Pain Medicine}\\
    Pediatrics & {Journal of the American academy of child \& Adolescent psychiatry \\ Pediatrics \\ JAMA pediatrics}\\
    Physical medicine and rehabilitation & {Neurorehabilitation and neural repair \\ Journal of fluency disorders}\\
    Psychiatry & {Molecular psychiatry \\ The American journal of psychiatry \\ JAMA psychiatry}\\
    Sports medicine & {Medicine and Science in Sports and Exercise \\ Sports Medicine \\ The American Journal of Sports Medicine}\\
    Surgery & {Annals of Surgery \\ American Journal of Transplantation \\ Journal of Neurology, Neurosurgery \& Psychiatry}\\

    \bottomrule

  \end{longtblr}
  }

  \begin{task}[label={myautocounter}]{\heiti Corpus-based task}
    Can you build your own corpus with at least 100, 000 tokens?
  \end{task}

  \begin{note}[label={myautocounter}]{\heiti Corpus used for this book}
    In this book, "corpus-based tasks" are designed to enhance your ability to explore language realizations of medical RA abstracts with corpus approach. Most of these tasks might require the use of software such as AntConc, WordSmith, and so on.
  \end{note}

  \section{Glossary}

  {\small
  \begin{longtblr}[
      caption = {Glossary of Chapter 1},
      label = {tab:Glossary of Chapter 1},
  ]{
      width = \textwidth,
      colspec = {X[1,l,h]  X[1,l,h]  X[3,l,h]},
      rowhead = 1, rowfoot = 0, % 每个分页里表头表尾的数量
      % row{odd} = {blue8}, 
      row{even} = {azure9},
  }
      
  \toprule
  \textbf{WORDS} & \textbf{MEANING} & \textbf{MEANING OR EXAMPLE}\\
  \midrule

  \textbf{excerpt}/\textipa{"eks3:pt}/ & \emph{v.} 摘录;引用  & If a long piece of writing or music is excerpted, short pieces from it are printed or played on their own. \\
  \textbf{genre}/\textipa{"ZA:nr@}/ & \emph{n.} 体裁 & a particular type of art, writing, music etc, which has certain features that all examples of this type share.\\
  {\textbf{mandate} \\ /\textipa{"m\ae ndeIt}/} & \emph{v.} 授权;强制执行;委托办理 & to tell someone that they must do a particular thing. \\
  {\textbf{methodology} \\ /\textipa{""meT@"d6l@\textdyoghlig i}/} & \emph{n.} 方法学 & a set of methods and principles used to perform a particular activity. \\
  \textbf{opt}/\textipa{6pt}/ & \emph{v.} 选择;挑选 & to choose one thing or do one thing instead of another \\
  \textbf{pivotal}/\textipa{"pIv@tl}/ & \emph{adj.} 关键性的;核心的 & more important than anything else in a situation or system. \\
  
  \bottomrule

  \end{longtblr}
  }

  \begin{problemset}
    \item \textbf{Identify whether the following abstracts are structured or unstructured and tell the reasons.}
    
    \textbf{Abstract 1}

    \hspace*{2em}Objectives: The aim of this study was to analyze changes in health care utilization and cost among a sample of highly impaired children and adolescents who sought a 3-week intensive interdisciplinary pain treatment (IPT).

    \hspace*{2em}Materials and Methods: Claims data from 7 statutory health insurance companies were analyzed for 65 children and adolescents who sought IIPT at the German Paediatric Pain Centre. The annual health care utilization and cost were determined for the following 4 areas: outpatient care, inpatient care, medications, and remedies and aids. We analyzed the changes in resource utilization in the year before (pre\_1 y) IPT and in the subsequent year (post\_1 y).
    
    \hspace*{2em}Results: Within the first year after IPT, overall health care costs did not decrease significantly. However, the pattern of health care utilization changed. First, significantly more children and adolescents started outpatient psychotherapy ($P=0.001$). Second, the number of hospitalized children decreased significantly from 1-year pre to 1-year post ($P=0.001$). Accordingly, there were significantly fewer hospitalizations for primary chronic pain disorders at 1-year post ($P<0.001$). The prescription of nonopioids, co-analgesics and opioids was significantly reduced from 1-year pre to 1-year post (all $P<0. 013$).
    
    \hspace*{2em}Discussion: The present results indicate that the health care costs of children and adolescents with severe chronic pain disorders do not significantly decrease 1 year after IPT; however, the treatment becomes more goal-focused. Differential diagnosis measures and nonindicated therapeutic interventions decreased, and more indicated interventions, such as psychotherapy, were used. Future research is needed to investigate the economic long-term changes after IPT.
    
    \begin{flushright}
      ---Health Care Utilization and Cost in Children and Adolescents with Chronic Pain: Analysis of Health Care Claims Data 1 Year Before and After Intensive Interdisciplinary Pain Treatment. 
      
      \emph{The Clinical Journal of Pain (2017)}
    \end{flushright}

    \textbf{Abstract 2}

    \hspace*{2em}Previous studies of brain structure in Tourette syndrome (TS) have produced mixed results, and most had modest sample sizes. In the present multicenter study, we used structural magnetic resonance imaging (MRI) to compare 103 children and adolescents with TS to a well-matched group of 103 children without tics. We applied voxel-based morphometry methods to test gray matter (GM) and white matter (WM) volume differences between diagnostic groups, accounting for MRI scanner and sequence, age, sex and total GM$+$WM volume. The TS group demonstrated lower WM volume bilaterally in orbital and medial prefrontal cortex, and greater GM volume in posterior thalamus, hypothalamus and midbrain. These results demonstrate evidence for abnormal brain structure in children and youth with TS, consistent with and extending previous findings, and they point to new target regions and avenues of study in TS. For example, as orbital cortex is reciprocally connected with hypothalamus, structural abnormalities in these regions may relate to abnormal decision making, reinforcement learning or somatic processing in TS.

    \begin{flushright}
      ---Brain structure in pediatric Tourette syndrome. 

      \emph{Molecular Psychiatry (2017)}
    \end{flushright}

    \textbf{Abstract 3}

    \hspace*{2em}Objective: To assess the feasibility of detecting signature volatile organic compounds in the breath of patients with oral squamous cell carcinoma.
    
    \hspace*{2em}Study Design: Prospective cohort pilot study.
    
    \hspace*{2em}Setting: University hospital.
    
    \hspace*{2em}Subjects and Methods: Using gas chromatography and mass spectrometry, emitted volatile organic compounds in the breath of patients before and after curative surgery (n=10) were compared with those of healthy subjects (n=4). It was hypothesized that certain volatile organic compounds disappear after surgical therapy. A characteristic signature of these compounds for diseased patients was compiled and validated.
    
    \hspace*{2em}Results: Breath analyses revealed 125 volatile organic compounds in patients with oral cancer. A signature of 8 compounds that were characteristic for patients with oral cancer could be detected: 3 from this group presented were absent after surgery.
    
    \hspace*{2em}Conclusion: The presented results confirmed the hypothesis of an absence of cancer-associated volatile organic compounds in the breath after therapy. In this pilot study, we proved the feasibility of this test approach. Further studies should be initiated to establish protocols for usage in a clinical setting.
    
    \begin{flushright}
      ---Volatile Organic Compounds in the Breath of Oral Squamous Cell Carcinoma Patients: A Pilot Study. 

      \emph{Otolaryngology Head and Neck Surgery (2017)}
    \end{flushright}
   
  \end{problemset}


\chapter{Move and Step Identification}\label{chapter2}

\section{Move Identification}

A medical RA abstract consists of moves which work together to achieve its communicative purposes. A move in this sense is ``a section of a text that performs a specific communicative function'' (Kanoksilapatham, 2007, p.23).

A four move scheme is used in most medical RA abstracts, with move 1 (M1) creating a research space move 2 (M2) describing research process, move 3 (M3) summarizing principal results and move 4 (M4) drawing conclusions. All of them are conventional moves in medical RA abstracts.

In a structured abstract, M2, M3 and M4 could be easily recognized via the headings of methods, results and conclusions. Although there is no specific heading of ``methods'' in some abstracts, M2 is subdivided into several steps which could be clearly recognized. They could be labelled ``design''; ``setting'', ``participants'', `` interventions'', `` main outcome measures", and so on.

When it comes to M1, sections labelled with either ``objectives" or ``background'' or both are included. In some structured abstracts, both ``background'' and ``"objectives"'' are labelled, in some only ``objectives", and in others only ``background". Those sections with the label of either ``objectives'' or ``background'' are usually comprised of both of them in terms of actual contents. Moreover, in practice, objectives and background are closely related and usually viewed as a whole to provide a specific communicative purpose of creating research space.

For a structured abstract, four moves could be recognized by headings, one of the lexical signals. For unstructured abstracts in which there are no headings or labels, manual recognition of four moves is needed. Other lexical signals could be helpful in recognizing the moves, which are illustrated in \hyperref[chapter3]{Chapter 3} to \hyperref[chapter6]{Chapter 6}.

\begin{sample}[label={myautocounter}]{\heiti}

\textbf{(M1)}

\textbf{BACKGROUND }

it is still equivocal whether there is a potential role of late-life physical activity in ameliorating the challenges of increasing healthcare expenditure due to the consequence of global population ageing.

\textbf{OBJECTIVE} 

this study aimed to examine the prospective association between physical activity and subsequent hospital care utilisation in older adults and to explore the optimal dose of physical activity required to reduce hospital care utilisation.

\textbf{(M2)}

\textbf{DESIGN} 

this was a prospective cohort study based on the data from the Taiwan 2005 National Health Interview Survey, which were linked to the 2005-12 claims data from the National Health Insurance system.

\textbf{PARTICIPANTS}

1,760 older adults aged 65 or more.

\textbf{METHODS} 

the frequency, duration and intensity for physical activity were assessed, and total physical activity energy expenditure was estimated. The average annualised hospital care utilisation for the period 2006 through 2012, including number of hospitalisations, number of days in hospital and the costs of hospitalisation, were calculated.

\textbf{(M3)}

\textbf{RESULTS }

older adults engaging in at least moderate volume of physical activity ($\geqslant 1,000$ kcal/week) experienced fewer subsequent hospital admissions and fewer days in hospital than did sedentary individuals, after adjusting for covariates. Trends for reduced hospitalisation costs were also found. These associations persisted in sensitivity analyses, including tests of reverse causation.

\textbf{(M4)}

\textbf{CONCLUSION}

this study has provided evidence that older adults who are at least moderately active may minimise utilisation of hospital care services. The findings highlight the importance of maintaining a physically active lifestyle in later life.

\begin{flushright}
  ---Prospective association between late-life physical activity and hospital care utilisation: a 7-year nationwide follow-up study. \emph{Age and Aging (2017)}
\end{flushright}

\end{sample}

\begin{sample}[label={myautocounter}]{\heiti}

\textbf{(M1)}

\textbf{PURPOSE }

Stuttering can trigger anxiety and other psychological and emotional reactions, and limit participation in society. It is possible that psychological counseling could enhance stuttering treatment outcomes; however, little is known about how clients view such counseling. The purpose of this study was to gain an understanding of clients' experiences with, and perceptions of, a psychological counseling service that was offered as an optional adjunct to speech therapy for stuttering.

\textbf{(M2)}

\textbf{METHOD }

Nine individuals who stutter (13--38 years old) participated in semi-structured interviews. Six participants had taken part in psychological counseling; three participants did not do so. Interview data were analyzed using grounded theory as a guiding framework.

\textbf{(M3)}

\textbf{RESULTS }

Four thematic clusters emerged from participants' accounts: insights into personal decision-making, why others may not participate in counseling, psychological counseling as a worthwhile part of therapy, and counseling as a necessary component in a stuttering treatment program.

\textbf{(M4)}

\textbf{CONCLUSION}

In addition to experiencing barriers and facilitators to help-seeking that are reported in related fields, participants accounts also revealed novel facilitators (i.e., a `why not' mentality and the importance of having a pre-existing relationship with the clinician who offered the service) and barriers (i. e., viewing the service as a `limited resource,' and, the overwhelming nature of intensive stuttering treatment programs). Findings suggest that clients value the option to access psychological counseling with trained mental health professionals to support the stuttering treatment provided by speech-language pathologists. Participants made recommendations for the integration of psychological counseling into stuttering treatment programs.

\begin{flushright}
  ---Psychological counseling as an adjunct to stuttering treatment: Clients' experiences and perceptions. 
  
  \emph{Journal of Fluency Disorder (2017)}
\end{flushright}


\end{sample}


\begin{sample}[label={myautocounter}]{\heiti}
\textbf{(M1)}

Genetic and neuroimaging research has identified neurobiological correlates of obesity. However, evidence for an integrated model of genetic risk and brain structural alterations in the pathophysiology of obesity is still absent.

\textbf{(M2)}

Here we investigated the relationship between polygenic risk for obesity, gray matter structure and body mass index (BMI) by the use of univariate and multivariate analyses in two large, independent cohorts ($n=330$ and $n=347$).

\textbf{(M3)}

Higher BMI and higher polygenic risk for obesity were significantly associated with medial prefrontal gray matter decrease, and prefrontal gray matter was further shown to significantly mediate the effect of polygenic risk for obesity on BMI in both samples.

\textbf{(M4)}

Building on this, the successful individualized prediction of BMI by means of multivariate pattern classification algorithms trained on whole-brain imaging data and external validations in the second cohort points to potential clinical applications of this imaging trait marker.

\begin{flushright}
  ---Prefrontal gray matter volume mediates genetic risks for obesity. \emph{Molecular Psychiatry (2017)}
\end{flushright}

\end{sample}

\section{Step Identification}

In the genre with an obvious hierarchical structure, moves are usually composed of the steps or sub-moves, which are the subordinate units. In medical RA abstracts, some steps are conventional steps and others are optional steps. The type and frequency of the steps also show the rhetorical purpose of the author. Lexical signals could be helpful in recognizing the steps, which are illustrated in \hyperref[chapter3]{Chapter 3} to \hyperref[chapter6]{Chapter 6}.

The abstracts of different journals in the corpus are randomly extracted, with 6 articles in each sub-discipline, totaling 108 articles. Through manual recognition, the steps and communication functions that constitute each step are established (Table 1). If the step is used in more than 80\% of articles, it is considered conventional, otherwise optional.

{\small
\begin{longtblr}[
    caption = {Move/Step Scheme of Medical RA Abstracts},
    label = {tab:Move/Step Scheme of Medical RA Abstracts},
]{
    width = \textwidth,
    colspec = {X[1,l,h]  X[1,c,h]  X[3,l,h] X[1,c,h]},
    rowhead = 1, rowfoot = 0, % 每个分页里表头表尾的数量
    % row{odd} = {blue8}, 
    % row{even} = {azure9},
    row{2-6} = {azure9},
    row{7-14} = {white},
    row{15-17} = {azure9},
    row{18-22} = {white},
}
    
\toprule
\textbf{Move/Step} & \textbf{Move/Step Abbr.} & \textbf{Communicative functions} & \textbf{Percentage}\\
\midrule
Move1 & M1 & Creating a research territory/space &  \\
\hspace*{1ex}Move1Step1 & 1S1 & Presenting current knowledge or relevant information established by previous studies & 56.5\% \\
\hspace*{1ex}Move1Step2 & 1S2 & Establishing a niche/problem\sidenotemark[1] & 41.7\% \\
\hspace*{1ex}Move1Step3a & 1S3a & Indicating main purposes & 83.3\% \\
\hspace*{1ex}Move1Step3b & 1S3b & Raising hypotheses & 2.8\% \\
Move2 & M2 & Describing research process &  \\
\hspace*{1ex}Move2Step1 & 2S1 & Reporting on medical ethics review & 2.8\% \\
\hspace*{1ex}Move2Step2 & 2S2 & Explaining briefly research design & 54.6\% \\
\hspace*{1ex}Move2Step3 & 2S3 & Describing subjects or data and their selection criteria & 92.6\% \\
\hspace*{1ex}Move2Step4 & 2S4 & Describing experimemal procedure, such as interventions, examinations, etc. & 60.2\% \\
\hspace*{1ex}Move2Step5 & 2S5 & Describing main outcomes and their measures & 93.5\% \\
\hspace*{1ex}Move2Step6 & 2S6 & Describing data analysis methods & 10.2\% \\
\hspace*{1ex}Move2Step7 & 2S7 & Reporting on registration information & 9.3\% \\
Move3 & M3 & Summarizing results &  \\
\hspace*{1ex}Move3Step1 & 3S1 & Providing information on valid samples & 43.5\% \\
\hspace*{1ex}Move3Step2 & 3S2 & Illustrating overall observation or main results & 100\% \\
Move4 & M4 & Drawing conclusions &  \\
\hspace*{1ex}Move4Step1 & 4S1 & Reiterating pivotal results & 16.7\% \\
\hspace*{1ex}Move4Step2 & 4S2 & Indicating limitations & 3.7\% \\
\hspace*{1ex}Move4Step3 & 4S3 & Stating the significance of the results & 99.1\% \\
\hspace*{1ex}Move4Step4 & 4S4 & Predicting future studies & 19.4\% \\

\bottomrule

\end{longtblr}
}

\sidenotetext[1]{{\small niche 原本指法国天主教徒房屋墙壁上预留用于放置圣母玛利亚的神龛。20世纪80年代,该词以“利基市场 (Niche Market) ”被美国商学家引入市场营销领域,意指那些被市场中的统治者/有绝对优势的企业忽略的某些细分市场。英文论文写作中,niche引申为被主流研究忽视的问题,即现有研究的缺陷、不足,指向该论文意图解决的问题。}}

\begin{sample}[label={myautocounter}]{\heiti}
  \textbf{BACKGROUND}

  \textbf{(1S1)} Graded exercise therapy is an effective and safe treatment for chronic fatigue syndrome, \textbf{(1S2)} but it is therapist intensive and availability is limited. \textbf{(1S3a)} We aimed to test the efficacy and safety of graded exercise delivered as guided self-help.
  
  \textbf{METHODS}

  \textbf{(2S2)} In this pragmatic randomised controlled trial, \textbf{(2S3)} we recruited adult patients (18 years and older) who met the UK National Institute for Health and Care Excellence criteria for chronic fatigue syndrome from two secondary-care clinics in the UK. \textbf{(2S4)} Patients were randomly assigned to receive specialist medical care (SMC) alone (control group) or SMC with additional guided graded exercise self-help (GES). Block randomisation (randomly varying block sizes) was done at the level of the individual with a computer-generated sequence and was stratified by centre, depression score, and severity of physical disability. Patients and physiotherapists were necessarily unmasked from intervention assignment; the statistician was masked from intervention assignment. SMC was delivered by specialist doctors but was not standardised; GES consisted of a self-help booklet describing a six-step graded exercise programme that would take roughly 12 weeks to complete, and up to four guidance sessions with a physiotherapist over 8 weeks (maximum 90 min in total). \textbf{(2S5)} Primary outcomes were fatigue (measured by the Chalder Fatigue Questionnaire) and physical function (assessed by the Short Form-36 physical function subscale); both were self-rated by patients at 12 weeks after randomisation and analysed in all randomised patients with outcome data at follow-up (ie, by modified intention to treat). We recorded adverse events, including serious adverse reactions to trial interventions.\textbf{ (2S6) }We used multiple linear regression analysis to compare SMC with GES, adjusting for baseline and stratification factors.\textbf{ (2S7) }This trial is registered at ISRCTN, number ISRCTN22975026.
  
  \textbf{FINDINGS}

  \textbf{(3S1)} Between May 15,2012, and Dec 24,2014, we recruited 211 eligible patients, of whom 107 were assigned to the GES group and 104 to the control group. \textbf{(3S2) }At 12 weeks, compared with the control group, mean fatigue score was 19.1 (SD-7.6) in the GES group and
  22.9(6.9) in the control group (adjusted difference-4.2 points, 95\% CI -6.1 to -2.3, $P<0.0 001$; effect size 0.53) and mean physical function score was 55.7(23.3) in the GES groupand 50. 8(25.3) in the control group (adjusted difference 6.3 points, 1.8 to 10.8, P=0.006;
  0.20). No serious adverse reactions were recorded and other safety measures did not differ between the groups, after allowing for missing data.

  \textbf{INTERPRETATION}

  \textbf{(4S2)} GES is a safe intervention that might reduce fatigue and, to a lesser extent, physical disability for patients with chronic fatigue syndrome. \textbf{(4S4)} These findings need confirmation and extension to other health-care settings.
  
  \begin{flushright}
    ---Guided graded exercise self-help plus specialist medical care versus specialist medical care alone for chronic fatigue syndrome (GETSET): a pragmatic randomised controlled trial. 
    
    \emph{The Lancet (2017)}
  \end{flushright}
  
\end{sample}

\section{Glossary}

{\small
\begin{longtblr}[
    caption = {Glossary of Chapter 2},
    label = {tab:Glossary of Chapter 2},
    note{a} = {英文论文中指代当前文献中的差距、问题或缺陷。即现有研究尚未解决的部分。},
]{
    width = \textwidth,
    colspec = {X[1,l,h]  X[1,l,h]  X[3,l,h]},
    rowhead = 1, rowfoot = 0, % 每个分页里表头表尾的数量
    % row{odd} = {blue8}, 
    row{even} = {azure9},
}
    
\toprule
\textbf{WORDS} & \textbf{MEANING} & \textbf{MEANING OR EXAMPLE}\\
\midrule

\textbf{conventional}/\textipa{k@n"venS@nl}/ & \emph{adj.} 传统的;常规的 & a conventional method, product, practice, etc. has been used for a long time and is considered the usual type \\
\textbf{ethics}/\textipa{"eTiks}/ & \emph{n.} 伦理标准 & [plural] moral rules or principles of behaviour for deciding what is right and wrong \\
\textbf{intervention}/\textipa{""Int@"venSn}/ & \emph{n.} 干预;介入 & an action or ministration that produces an effect or is intended to alter the course of a pathologic process \\
\textbf{niche}/\textipa{ni:S}/ & \emph{n.} 利基\TblrNote{a};生态位;微环境  & a gap in the previous research \\
\textbf{reiterate}/\textipa{ri:"It@reIt}/ & \emph{v.} 重申  & to say something again, usually in order to emphasize it \\

\bottomrule

\end{longtblr}
}

\begin{problemset}
  \item \textbf{Identify the four moves in the abstracts.}
  
  \textbf{Abstract 1}

  \hspace*{2em}Objective: To evaluate the association between the parameters of 24-hour multichannel intraluminal impedance (MII)-pH monitoring and the symptoms or quality of life (QoL) in laryngopharyngeal reflux (LPR) patients.
  
  \hspace*{2em}Design: Prospective cohort study without controls.
  
  \hspace*{2em}Setting: University teaching hospital.
  
  \hspace*{2em}Methods: Forty-five LPR patients were selected from subjects who underwent 24-hour MII-
  pH monitoring and were diagnosed with LPR from September 2014 to May 2015. Reflux Symptom Index (RSI), Health-related Quality of Life (HRQoL), Short Form 12 (SF-12) Survey questionnaires were surveyed. Spearman's correlation was used to analyse the association between the symptoms or QoL and 24-hour MII-pH monitoring.
  
  \hspace*{2em}Results: Most parameters in 24-hour MII-pH monitoring showed weak or no correlation with RSI, HRQoL and SF-12. Only number of non-acid reflux events that reached the larynxand pharynx (LPR-non-acid) and number of total reflux events that reached the larynx and pharynx (LPR-total) parameters showed strong correlation with heartburn in RSI (R=0.520, $P<0$.001, R=0.478, P=0.001, respectively). Multiple regression analysis showed that there was only one significant regression coefficient between LPR-non-acid and voice/hoarseness portion of HRQoL (b=1.719, P=0.022).
  
  \hspace*{2em}Conclusion: Most parameters of 24-hour MII-pH monitoring did not reflect subjective symptoms or QoL in patients with LPR.
  
  \begin{flushright}
    ---Association between 24-hour combined multichannel intraluminal impedance-pH monitoring and symptoms or quality of life in patients with laryngopharyngeal reflux. \emph{Clinical Otolaryngology (2017)}
  \end{flushright}
  
  \textbf{Abstract 2}
  
  \hspace*{2em}Skeletal muscle mitochondrial oxidative capacity declines with age and negatively affects walking performance, but the mechanism for this association is not fully clear. We tested the hypothesis that impaired oxidative capacity affects muscle performance and, through this mechanism, has a negative effect on walking speed. Muscle mitochondrial oxidative capacity was measured by in vivo phosphorus magnetic resonance spectroscopy as the postexercise phosphocreatine resynthesis rate, Kpc, in 326 participants (154 men), aged 24-97 years (mean 71), in the Baltimore Longitudinal Study of Aging. Muscle strength and quality were determined by knee extension isokinetic strength, and the ratio of knee extension strength to thigh muscle cross-sectional area derived from computed topography, respectively. In multivariate linear regression analyses, kpc, was associated with muscle strength ($\beta=0. 140, P=0.007$) and muscle quality ($\beta=0.127, P=0.022$), independent of age, sex, height, and weight; muscle strength was also a significant independent correlate of walking speed ($P<0. 02$ for all tasks) and in a formal mediation analysis significantly attenuated the association between kpc and three of four walking tasks (18\%-29\% reduction in $\beta$ for kpcr). This is the first demonstration in human adults that mitochondrial function affects muscle strength and that inefficiency in muscle bioenergetics partially accounts for differences in mobility through this mechanism.

  \begin{flushright}
    ---Muscle strength mediates the relationship between mitochondrial energetics and walking performance. \emph{Aging Cell (2017)}
  \end{flushright}

  \textbf{Abstract 3}

  \hspace*{2em}Propacetamol, a water-soluble prodrug form of paracetamol, is hydrolyzed by esterase to generate paracetamol in the blood. Each gram of propacetamol is equal to 0.5 g of paracetamol. It has been reported to cause hypotension in critically ill patients with a fever. We aimed to investigate the hemodynamic effects of propacetamol for the control of fever in patients with diverse severities of illness who were managed in the emergency department (ED). We also aimed to identify clinical factors related to significant hemodynamic alterations in ED patients. This was a retrospective study of 1507 ED patients who received propacetamol. Significant hemodynamic alterations were defined as systolic blood pressure (SBP) <90 mmHg or diastolic blood pressure (DBP) <60 mmHg, or a drop in SBP >30 mmHg, which required treatments with a bolus of fluid or vasopressor administration. Postinfusion SBP and DBP were significantly lower than the preinfusion SBP and DBP. A clinically significant drop in BP occurred in 162 (10.7\%) patients, and interventions were necessary. Among the predictors assessed, congestive heart failure (OR 6.21,95\% CI 2.67--14.45) and chills (OR 3.10,95\% CI 2.04--4.70) were independent factors for a significant hemodynamic change. Administrationof propacetamol can provoke a reduction in BP in ED patients. This reduction was clinically significant for 10\% of infusions. Clinicians should be aware of this potential deleterious effect, especially in patients with congestive heart failure or who experience chills prior to the administration of propacetamol.

  \begin{flushright}
    ---Clinically significant hemodynamic alterations after propacetamol injection in the emergency department: prevalence and risk factors. \emph{Internal and Emergency Medicine (2016)}
  \end{flushright}

  \item \textbf{Reorder the four moves in the abstracts.}
  
  \textbf{Abstract 1}

  \hspace*{2em}Methods: A dual-virus tracing strategy combining retroviral birthdating with rabies virus-
  mediated putative retrograde trans-synaptic tracing was used to identify and compare presynaptic inputs onto adult-born and early-born DGCs in the rat pilocarpine model of mTLE.

  \hspace*{2em}---Move \uline{\hspace*{3em}}
  
  \hspace*{2em}Objective: To understand how monosynaptic inputs onto adult-born dentate granule cells (DGCs) are altered in experimental mesial temporal lobe epilepsy (mTLE) and whether their integration differs from early-born DGCs that are mature at the time of epileptogenesis.

  \hspace*{2em}---Move \uline{\hspace*{3em}}
  
  \hspace*{2em}Interpretation: These data support the presence of substantial hippocampal circuit remodeling after an epileptogenic insult that generates prominent excitatory monosynaptic inputs, both local recurrent and widespread feedback loops, onto DGCs. Both adult-born and early-bom DGCs are targets of new inputs from other DGCs as well as from CA3 and CA1 pyramidal cells after pilocarpine treatment, changes that likely contribute to epileptogenesis in experimental mTLE.
  
  \hspace*{2em}---Move \uline{\hspace*{3em}}

  \hspace*{2em}Results: Our results demonstrate that hilar ectopic DGCs preferentially synapse onto adult-born DGCs after pilocarpine-induced status epilepticus (SE), whereas normotopic DGCs synapse onto both adult-born and early-born DGCs. We also find that parvalbumin- and somatostatin-interneuron inputs are greatly diminished onto early-born DGCs after SE. However, somatostatin-interneuron inputs onto adult-born DGCs are maintained, likely due topreferential sprouting. Intriguingly, CA3 pyramidal cell backprojections that specifically target adult-born DGCs arise in the epileptic brain, whereas axons of interneurons and pyramidal cells in CA1 appear to sprout across the hippocampal fssure to preferentilly synapse onto early-bomDGCs.

  \hspace*{2em}---Move \uline{\hspace*{3em}}

  \begin{flushright}
    ---Rabiestracing of birthdated dentae granule cells in rat temporal lobe epilepsy. \emph{Annals of Neurology(2017)}
  \end{flushright}

  \textbf{Abstract 2}
  
  \hspace*{2em}We identified a mssense Asn396Ser mutation (rs77960347) in the endothelial lipase (LIG) gene, occurring with an allele frequency of 1\% in the general population, which was significantly associated with depressive symptoms (P-value $= 5.2 \times 10--08, \beta=7.2$). Replication in three independent data sets ($N= 3612$) confirmed the association of Asn396Ser (P-value = $7.1 \times 10-03$, $\beta = 2.55$) with depressive symptoms.

  \hspace*{2em}---Move \uline{\hspace*{3em}}

  \hspace*{2em}Despite a substantial genetic component, efforts to identify common genetic variation underlying depression have largely been unsuccessful. In the current study we aimed to identify rare genetic variants that might have large effects on depression in the general population.

  \hspace*{2em}---Move \uline{\hspace*{3em}}

  \hspace*{2em}Using high-coverage exome-sequencing, we studied the exonic variants in 1 265 individuals from the Rotterdam study (RS), who were assessed for depressive symptoms.

  \hspace*{2em}---Move \uline{\hspace*{3em}}

  \hspace*{2em}LIPG is predicted to have enzymatic function in steroid biosynthesis, cholesterol biosynthesis and thyroid hormone metabolic processes. The Asn396Ser variant is predicted tod have a damaging effect on the function of LIPG. Within the discovery population, carriers also showed an increased burden of white matter lesions (P-value=$3.3 \times 10-02$) and a higher risk of Alzheimer's disease (odds ratio=2.01; P-value = 2.8x 10-02) compared with the noncarriers. Together, these findings implicate the Asn396Ser variant of LIPG in the pathogenesis of depressive symptoms in the general population.

  \hspace*{2em}---Move \uline{\hspace*{3em}}

  \begin{flushright}
    ---Exome-sequencing in a large population-based study reveals a rare Asn396Ser variant in the LIPG gene associated with depressive symptoms. \emph{Molecular Psychiatry (2017)}
  \end{flushright}

  \textbf{Abstract 3}

  \hspace*{2em}Conclusion: This study suggests that hearing problems in later life could increase the risk of having difficulties performing IADLs, which include more complex everyday tasks such as shopping and light housework. However, further studies are needed to determine the associations observed including the underlying pathways.
  
  \hspace*{2em}---Move \uline{\hspace*{3em}} 
  
  \hspace*{2em}Methods: Data were collected on self-reported hearing impairment including hearing aid use, and disability assessed as mobility limitations (problems walking/taking stairs), difficulties with activities of daily living (ADL) and instrumental ADL (IADL). Mortality data were obtained from the National Health Service register.
  
  \hspace*{2em}---Move \uline{\hspace*{3em}}
  
  \hspace*{2em}Background and objective: Hearing impairment is common in older adults and has been implicated in the risk of disability and mortality. We examined the association between hearing impairment and risk of incident disability and all-cause mortality.

  \hspace*{2em}---Move \uline{\hspace*{3em}}
  
  \hspace*{2em}Results: Among 3,981 men, 1,074(27\%) reported hearing impairment. Compared with men with no hearing impairment, men who could hear and used a hearing aid, and men who could not hear despite a hearing aid had increased risks of IADL difficulties (age-adjusted OR 1.86, 95\% CI 1.29--2.70; OR 2.74, 95\% CI 1.53--4.93, respectively). The associations remained after further adjustment for covariates including social class, lifestyle factors, comorbidities and social engagement. Associations of hearing impairment with incident mobility limitations, incident ADL difficulties and all-cause mortality were attenuated on adjustment for covariates.
  
  \hspace*{2em}---Move \uline{\hspace*{3em}}
  
  \begin{flushright}
    ---Hearing impairment and incident disability and all-cause mortality in older British community-dwelling men. \emph{Age and Ageing (2016)}
  \end{flushright}
  
\end{problemset}


\chapter{Move One}\label{chapter3}

Move One (M1) is the first part of an abstract. It creates a research territory or space for the research

\section{Steps in M1}

M1 usually involves three steps, each with a clear communicative purpose.

\begin{itemize}
  \item 1S1 Presenting current knowledge or relevant information established by previous studies

  \item 1S2 Establishing a niche/problem

  \item 1S3a Indicating main purposes

  \item 1S3b Raising hypotheses
\end{itemize}

Usually in the last step either the objectives or the hypotheses are presented while only a few abstracts cover both.

\begin{sample}[label={myautocounter}]{\heiti}
  Senescent cells are present in premalignant lesions and sites of tissue damage and accumulate in tissues with age. In vivo identification, quantification and characterization of senescent cells are challenging tasks that limit our understanding of the role of senescent cells in diseases and aging. Here, we present a new way to precisely quantify and identify senescent cells in tissues on a single-cell basis.

  \begin{flushright}
    ---Quantitative identification of senescent cells in aging and disease. \emph{Aging Cell (2017)}
  \end{flushright}

  \tcblower

  \noindent \textbf{STEP IDENTIFICATION}

  % \begin{tblr}[]{width=0.9\textwidth,colspec={Q[c] Q[l]}}
  %   \toprule
  %   Step & Sample \\ 
  %   \midrule
  %   1S1 & Senescent cells are present in premalignant lesions and sites of tissue damage and accumulate in tissues with age.  \\
  %   1S2 & In vivo identification, quantification and characterization of senescent cells are challenging tasks that limit our understanding of the role of senescent cells in diseases and aging. \\
  %   1S3a & Here, we present a new way to precisely quantify and identify senescent cells in tissues on a single-cell basis. \\
  %   \bottomrule
  % \end{tblr}

  {\small
  \begin{longtblr}[
      caption = {Common Prefixes},
      label = {tab:Common_Prefixes},
  ]{
      width = \textwidth,
      colspec = {X[1,c,h]  X[5,l,h]},
      rowhead = 1, rowfoot = 0, % 每个分页里表头表尾的数量
      % row{odd} = {blue8}, 
      row{even} = {azure9},
  }
      
    \toprule
    \textbf{Step} & \textbf{Sample} \\ 
    \midrule
    1S1 & Senescent cells are present in premalignant lesions and sites of tissue damage and accumulate in tissues with age.  \\
    1S2 & In vivo identification, quantification and characterization of senescent cells are challenging tasks that limit our understanding of the role of senescent cells in diseases and aging. \\
    1S3a & Here, we present a new way to precisely quantify and identify senescent cells in tissues on a single-cell basis. \\
    \bottomrule

  \end{longtblr}
  }

  \noindent \textbf{ANALYSIS} 
  
  This is a typical example of M1 involving three steps. The first sentence informs the readers of the background information related to the research. The second sentence, where the word ``no'' can be seen as a signal for 1S2, identifies the problem-the lack of a standard. The last sentence, with the subject ``aim'' and the to-infinitive, shows the research objective.
\end{sample}



\begin{sample}[label={myautocounter}]{\heiti}

  \textbf{OBJECTIVE} 
  
  Seizures are more frequent in patients with Alzheimer's disease (AD) and can hasten cognitive decline. However, the incidence of subclinical epileptiform activity in AD and its consequences are unknown. Motivated by results from animal studies, we hypothesized higherthan expected rates of subclinical epileptiform activity in AD with deleterious effects on cognition.

  
  \begin{flushright}
    ---Incidence and impact of subclinical epileptiform activity in Alzheimer's disease. \emph{Annals of Neurology (2016)}
  \end{flushright}

  \tcblower

  \noindent \textbf{STEP IDENTIFICATION}

  {\small
  \begin{longtblr}[
      caption = {Common Prefixes},
      label = {tab:Common_Prefixes},
  ]{
      width = \textwidth,
      colspec = {X[1,c,h]  X[5,l,h]},
      rowhead = 1, rowfoot = 0, % 每个分页里表头表尾的数量
      % row{odd} = {blue8}, 
      row{even} = {azure9},
  }
      
    \toprule
    \textbf{Step} & \textbf{Sample} \\ 
    \midrule
    
    1S1 & Seizures are more frequent in patients with Alzheimer's disease (AD) and can hasten cognitive decline. \\
    1S2 & However, the incidence of subclinical epileptiform activity in AD and its consequences are unknown. \\
    1S3b & Motivated by results from animal studies, we hypothesized higherthan expected rates of subclinical epileptiform activity in AD with deleterious effects on cognition. \\

    \bottomrule

  \end{longtblr}
  }
  
\end{sample}

1S3a is a comparatively conventional step in this move, while others are optional. These steps follow the orders listed above to fulfill the authors' communicative purposes, which are to clarify what is already known, what remains unknown and what is to be known.

\begin{sample}[label={myautocounter}]{\heiti}

  \textbf{BACKGROUND} 
  
  CT-P6 is a proposed biosimilar to reference trastuzumab. In this study, we aimed to establish equivalence of CT-P6 to reference trastuzumab in neoadjuvant treatment of HER2-positive early-stage breast cancer.

  
  \begin{flushright}
    ---CT-P6 compared with reference trastuzumab for HER2-positive breast cancer: a randomised, double-blind, active-controlled, phase 3 equivalence trial. \emph{Lancet Oncology (2017)}
  \end{flushright}

  \tcblower

  \noindent \textbf{STEP IDENTIFICATION}

  {\small
  \begin{longtblr}[
      caption = {Common Prefixes},
      label = {tab:Common_Prefixes},
  ]{
      width = \textwidth,
      colspec = {X[1,c,h]  X[5,l,h]},
      rowhead = 1, rowfoot = 0, % 每个分页里表头表尾的数量
      % row{odd} = {blue8}, 
      row{even} = {azure9},
  }
      
    \toprule
    \textbf{Step} & \textbf{Sample} \\ 
    \midrule
    
     1S1 & \textbf{BACKGROUND}: CT-P6 is a proposed biosimilar to reference trastuzumab. \\
     1S3a & In this study, we aimed to establish equivalence of CT-P6 to reference trastuzumab in neoadjuvant treatment of HER2-positive early-stage breast cancer. \\

    \bottomrule

  \end{longtblr}
  }

\end{sample}

\begin{sample}[label={myautocounter}]{\heiti}

  At a population level, dietary consumption of fish rich in docosahexaenoic acid (DHA) is associated with prevention of cognitive decline but this association is not clear in carriers of the apolipoprotein E epsilon 4 allele (E4). Plasma and liver DHA concentrations show significant alterations in EA carriers, in part corrected by DHA supplementation. However, whether DHA sufficiency in E4 carriers has consequences on cognition is unknown.

  
  \begin{flushright}
    ---Docosahexaenoic acid prevents cognitive deficits in human apolipoprotein E epsilon 4-targeted replacement mice. \emph{Neurobiology of Aging (2017)}
  \end{flushright}

  \tcblower

  \noindent \textbf{STEP IDENTIFICATION}

  {\small
  \begin{longtblr}[
      caption = {Common Prefixes},
      label = {tab:Common_Prefixes},
  ]{
      width = \textwidth,
      colspec = {X[1,c,h]  X[5,l,h]},
      rowhead = 1, rowfoot = 0, % 每个分页里表头表尾的数量
      % row{odd} = {blue8}, 
      row{even} = {azure9},
  }
      
    \toprule
    \textbf{Step} & \textbf{Sample} \\ 
    \midrule
    
     1S1 & At a population level, dietary consumption of fish rich in docosahexaenoic acid (DHA) is associated with prevention of cognitive decline but this association is not clear in carriers of the apolipoprotein E epsilon 4 allele (E4). Plasma and liver DHA concentrations show significant alterations in EA carriers, in part corrected by DHA supplementation. \\
     1S2 & However, whether DHA sufficiency in E4 carriers has consequences on cognition is unknown. \\

    \bottomrule

  \end{longtblr}
  }

\end{sample}

\begin{sample}[label={myautocounter}]{\heiti}

  The aim of this cohort study is to compare the symptom burden of patients who have an planned admission. unplanned admission to an acute palliative care unit (APCU) with patients who have a regular planned admission.

  
  \begin{flushright}
    ---Characteristics of patients with an unplanned admission to an acute palliative care unit. \emph{Internal and Emergency Medicine (2017)}
  \end{flushright}

  \tcblower

  \noindent \textbf{STEP IDENTIFICATION}

  {\small
  \begin{longtblr}[
      caption = {Common Prefixes},
      label = {tab:Common_Prefixes},
  ]{
      width = \textwidth,
      colspec = {X[1,c,h]  X[5,l,h]},
      rowhead = 1, rowfoot = 0, % 每个分页里表头表尾的数量
      % row{odd} = {blue8}, 
      row{even} = {azure9},
  }
      
    \toprule
    \textbf{Step} & \textbf{Sample} \\ 
    \midrule
    
     1S3a & The aim of this cohort study is to compare the symptom burden of patients who have an planned admission. unplanned admission to an acute palliative care unit (APCU) with patients who have a regular planned admission. \\

    \bottomrule

  \end{longtblr}
  }

\end{sample}

It needs to be mentioned, however, that these steps can also be presented with different orders in a small number of cases.

\begin{sample}[label={myautocounter}]{\heiti}
  
  \textbf{OBJECTIVE}

  To compare the rate of positive resection margins between radioactive seed localization (RSL) and wire-guided localization (WGL) after breast conserving surgery (BCS).
 
  \textbf{BACKGROUND} 
  
  WGL is the current standard for localization of nonpalpable breast lesions in BCS, but there are several difficulties related to the method.

  
  \begin{flushright}
    ---Radioactive Seed Localization or Wire-guided Localization of Nonpalpable Invasive and InSitu Breast Cancer: A Randomized, Multicenter, Open-label Trial. \emph{Annals of Surgery (2017)}
  \end{flushright}

  \tcblower

  \noindent \textbf{STEP IDENTIFICATION}

  {\small
  \begin{longtblr}[
      caption = {Common Prefixes},
      label = {tab:Common_Prefixes},
  ]{
      width = \textwidth,
      colspec = {X[1,c,h]  X[5,l,h]},
      rowhead = 1, rowfoot = 0, % 每个分页里表头表尾的数量
      % row{odd} = {blue8}, 
      row{even} = {azure9},
  }
      
    \toprule
    \textbf{Step} & \textbf{Sample} \\ 
    \midrule
    
     1S3a & \textbf{OBJECTIVE}: To compare the rate of positive resection margins between radioactive seed localization (RSL) and wire-guided localization (WGL) after breast conserving surgery (BCS). \\
     1S1 & \textbf{BACKGROUND}: WGL is the current standard for localization of nonpalpable breast lesions in BCS, \\
     1S2 & but there are several difficulties related to the method. \\

    \bottomrule

  \end{longtblr}
  }  

\end{sample}

\begin{sample}[label={myautocounter}]{\heiti}

  \textbf{OBJECTIVE} 
  
  The aim of this study was to investigate the efficacy of intraperitoneal local anesthetic (IPLA) on pain after acute laparoscopic appendectomy in children.

  \textbf{SUMMARY OF BACKGROUND} 
  
  IPLA reduces pain in adult elective surgery. It has not been well studied in acute peritoneal inflammatory conditions. We hypothesized that IPLA would improve recovery in pediatric acute laparoscopic appendectomy.

  
  \begin{flushright}
    ---Intraperitoneal Local Anesthetic for Laparoscopic Appendectomy in Children: A Randomized Controlled Trial. \emph{Annals of Surgery (2017)}
  \end{flushright}

  \tcblower

  \noindent \textbf{STEP IDENTIFICATION}

  {\small
  \begin{longtblr}[
      caption = {Common Prefixes},
      label = {tab:Common_Prefixes},
  ]{
      width = \textwidth,
      colspec = {X[1,c,h]  X[5,l,h]},
      rowhead = 1, rowfoot = 0, % 每个分页里表头表尾的数量
      % row{odd} = {blue8}, 
      row{even} = {azure9},
  }
      
    \toprule
    \textbf{Step} & \textbf{Sample} \\ 
    \midrule
    
     1S3a & \textbf{OBJECTIVE}: The aim of this study was to investigate the efficacy of intraperitoneal local anesthetic (IPLA) on pain after acute laparoscopic appendectomy in children. \\
     1S1 & \textbf{SUMMARY OF BACKGROUND}: IPLA reduces pain in adult elective surgery.  \\
     1S2 & It has not been well studied in acute peritoneal inflammatory conditions. \\
     1S3b & We hypothesized that IPLA would improve recovery in pediatric acute laparoscopic appendectomy. \\

    \bottomrule

  \end{longtblr}
  }

\end{sample}

\section{Language Features in Each Step}

  \subsection{Step 1(1S1) Presenting current knowledge or relevant information established by previous studies}

    \subsubsection{Step Analysis}

    With this step authors introduce naturally the research having been conducted. In this step, current knowledge or relevant information established by previous studies is presented or explained, which might include the pertinent mechanism or definition, the possibility of a phenomenon, the significance of a certain study, etc.
    
    There are two ways that this step could be put forward. Authors could present the knowledge or information in the same scope (E.g.~\ref{eg:1s1-1}). Also, the authors could introduce the conclusions of relevant previous studies (E.g.~\ref{eg:1s1-2}). In the latter one authors always employ words or phrases, such as “preclinical studies", to indicate the source of the information (E.g.~\ref{eg:1s1-2})
    
    \begin{eg}[label={eg:1s1-1}]{}
      Topical immunomodulators(Tl)-including corticosteroids, calcineurin inhibitors, and vitamin D analogues-are commonly prescribed in multiple specialties,$\dots$
    \end{eg}

    \begin{eg}[label={eg:1s1-2}]{}
      \uline{Preclinical studies} have found radiotherapy enhances antitumour immune responses.
    \end{eg}

    \subsubsection{Language Realizations}

    Simple present tense is often used in this step to describe the current understanding of the topic.

    \begin{eg}[label={myautocounter}]{}
      In chronic hemodialysis, physical functioning (PF) \uline{is known} to be poor.
    \end{eg}

    While ``previous studies'' or other phrases with similar meaning are used as the subject of the sentence, present perfect tense is preferred. However, simple present tense is still used to refer to the current knowledge or relevant information established by previous studies.

    \begin{eg}[label={myautocounter}]{}
      Previous studies have shown that more active older adults have better cognition and brain health based on a variety of structural neuroimaging measures.
    \end{eg}

    \begin{task}[label={myautocounter}]{\heiti Corpus-based task}
      What kind of verb is often used in present perfect tense in M1?
    \end{task}

    Both active voice and passive voice could be used at this step.

    \begin{eg}[label={myautocounter}]{}
      Physicians \uline{are} often \uline{asked} to prognosticate soon after a patient presents with stroke.
    \end{eg}

    \begin{eg}[label={myautocounter}]{}
      Youth baseball frequently \uline{results} in repetitive strain injuries.
    \end{eg}

    Modal auxiliary verbs, such as ``may'', ``can'', as well as other words indicating likelihood, such as ``possible'', are commonly used in this step.

    \begin{eg}[label={myautocounter}]{}
      Urban design \uline{may} affect children's habitual physical activity by influencing active commuting and neighborhood play.
    \end{eg}

    \begin{eg}[label={myautocounter}]{}
      Hirsutism in females \uline{can} be a source of considerable psychological distress and a threat to female identity.
    \end{eg}

    \begin{eg}[label={myautocounter}]{}
      It is \uline{possible} that psychological counseling could enhance stuttering treatment outcomes.
    \end{eg}

    \subsubsection{Lexical Chunks}

    \begin{enumerate}
      \item has been associated with
      \begin{eg}[label={myautocounter}]{}
        Ethnicity \uline{has been associated with} clinical and experimental pain responses.
      \end{eg}

      \item has been shown to
      \begin{eg}[label={myautocounter}]{}
        Estrogen administration following menopause \uline{has been shown to} support hippocampally mediated cognitive processes.
      \end{eg}
    \end{enumerate}
  
  \subsection{Step 2(1S2) Establishing a niche/problem}

    \subsubsection{Step Analysis}

    In this step, negative evaluations are often given to the information provided at Step 1, including the limitations or the defects of the previous studies, which leaves a gap to be filled, a problem to be addressed or an idea to be tested. In other words, this step implies the value and significance of the authors' research.

    \begin{eg}{}
      However, to date, underlying neuronal mechanisms of these WM load-dependent activation changes in aging remain \uline{poorly understood}.
    \end{eg}

    \subsubsection{Language Realizations}

    Simple present tense is usually seen in this step. But when the word ``study'' serves as the subject of an active sentence, or in a passive sentence as the unexpressed agent, present perfect tense is more often employed.

    \begin{eg}{}
      Despite the availability of objective tests, gastroesophageal reflux disease (GERD) diagnosis and management in infants \uline{remains} controversial and highly variable.
    \end{eg}

    \begin{eg}{}
      However, no study \uline{has} directly \uline{compared} these outcomes between sports.
    \end{eg}

    Words such as ``however'' and ``but'' could be used to lead the readers from Step 1 to Step2, placing an emphasis on the problem or gap to be pointed out.

    \begin{eg}{}
      Delirium is associated with adverse postoperative outcomes, \uline{but} controversy exists regarding whether delirium is an independent predictor of mortality.
    \end{eg}

    \begin{eg}{}
      Excellent outcomes have been reported for anterior cruciate ligament (ACL) reconstruction (ACLR) in professional athletes in a number of different sports. \uline{However}, no study has directly compared these outcomes between sports.
    \end{eg}

    Besides the two words mentioned above, other words indicating a contrast or a negative meaning, such as ``not'' or ``unknown'', can also be seen as a signal for 1S2.

    \begin{eg}{}
      Validated models to predict risk for complications are \uline{not} available, and the effect of treatment on risk is \uline{unknown}.
    \end{eg}

    \subsubsection{Lexical Chunks}

    \begin{enumerate}
      \item little is known about the
      \begin{eg}{}
        However, \uline{little is known about the} effect of pharmacological PHD inhibition on tumor expansion, and on liver regeneration after surgical resection.
      \end{eg}
    \end{enumerate}

\section{Sample Reading}

Three samples of M1 are presented here, followed by a detailed analysis on the language features of the steps included.

\begin{sample}[label={myautocounter}]{\heiti}
  
  \textbf{BACKGROUND} 
  
  Head impacts and resulting head accelerations cause concussive injuries. There is no standard for reporting head impact data in sports to enable comparison between studies.

  \textbf{OBJECTIVE }
  
  The aim was to outline methods for reporting head impact acceleration data in sport and the effect of the acceleration thresholds on the number of impacts reported.


  \begin{flushright}
    ---The Influence of Head Impact Threshold for Reporting Data in Contact and Collision Sports: Systematic Review and Original Data Analysis. \emph{Sports Medicine (2015)}
  \end{flushright}

  \tcblower

  \noindent \textbf{STEP IDENTIFICATION}

  {\small
  \begin{longtblr}[
      caption = {Common Prefixes},
      label = {tab:Common_Prefixes},
  ]{
      width = \textwidth,
      colspec = {X[1,c,h]  X[5,l,h]},
      rowhead = 1, rowfoot = 0, % 每个分页里表头表尾的数量
      % row{odd} = {blue8}, 
      row{even} = {azure9},
  }
      
    \toprule
    \textbf{Step} & \textbf{Sample} \\ 
    \midrule
    
     1S1 & \textbf{BACKGROUND}: Head impacts and resulting head accelerations cause concussive injuries. \\
     1S2 & There is no standard for reporting head impact data in sports to enable comparison between studies. \\
     1S3a & \textbf{OBJECTIVE }: The aim was to outline methods for reporting head impact acceleration data in sport and the effect of the acceleration thresholds on the number of impacts reported. \\

    \bottomrule

  \end{longtblr}
  }

  \noindent \textbf{ANALYSIS}

  This is a typical example of M1 involving three steps. The first sentence informs the readers of the background information related to the research. The second sentence, where the word ``no'' can be seen as a signal for 1S2, identifies the problem---the lack of a standard. The last sentence, with the subject ``aim'' and the to-infinitive, shows the research objective.

\end{sample}

\begin{sample}[label={myautocounter}]{\heiti}
  \textbf{BACKGROUND} 
  
  Ergometrine is a uterotonic agent that is recommended in the prevention and management of postpartum hemorrhage. Despite its long-standing use, the mechanism by which it acts in humans has never been elucidated fully. The objective of this study was to investigate the role of adrenoreceptors in ergometrine's mechanism of action in human myometrium. The study examined the hypothesis that a-adrenoreceptor antagonism would result in the reversal of the uterotonic effects of ergometrine.

  \begin{flushright}
    ---A Role for Adrenergic Receptors in the Uterotonic Effects of Ergometrine in Isolated Human Term Nonlaboring Myometrium. \emph{Anesthesia and-Analgesia (2017)}
  \end{flushright}

  \tcblower

  \noindent \textbf{STEP IDENTIFICATION}

  {\small
  \begin{longtblr}[
      caption = {Common Prefixes},
      label = {tab:Common_Prefixes},
  ]{
      width = \textwidth,
      colspec = {X[1,c,h]  X[5,l,h]},
      rowhead = 1, rowfoot = 0, % 每个分页里表头表尾的数量
      % row{odd} = {blue8}, 
      row{even} = {azure9},
  }
      
    \toprule
    \textbf{Step} & \textbf{Sample} \\ 
    \midrule
    
    1S1  & \textbf{BACKGROUND}: Ergometrine is a uterotonic agent that is recommended in the prevention and management of postpartum hemorrhage.  \\
    1S2  & Despite its long-standing use, the mechanism by which it acts in humans has never been elucidated fully. \\
    1S3a  & The objective of this study was to investigate the role of adrenoreceptors in ergometrine's mechanism of action in human myometrium. \\
    1S3b  & The study examined the hypothesis that a-adrenoreceptor antagonism would result in the reversal of the uterotonic effects of ergometrine. \\

    \bottomrule

  \end{longtblr}
  }

  \noindent \textbf{ANALYSIS} 

  This paragraph involves four sentences. The first one provides the readers with the necessary knowledge of Ergometrine, a drug used to promote the contractions of the muscle ofthe womb (uterus). The second sentence, with the word "never"serving as the signal for 1S2, points out the gap that needs to be filled -the mechanism has not "been elucidated fully". Here, "elucidate"means to make clear. Present perfect tense is used, because the omitted agent in this passive sentence is "the previous study". The third sentence introduces the objective and the last one raises a hypothesis. 
  
  What needs to be noticed is that 1S3a and 1S3b are not very often seen with each other. This sample is one of the few exceptions.

\end{sample}

\begin{sample}[label={myautocounter}]{\heiti}
  
  \textbf{OBJECTIVE} 
  
  To investigate the prognostic significance of p16 in patients with hypopharyngeal squamous cell carcinoma (HPSCC) and to evaluate the relationship between p16 and human papillomavirus (HPV). Unlike in oropharyngeal SCC (OPSCC), the prognostic significance of p16 in HPSCC and its association with HPV is unclear.


  \begin{flushright}
    ---p16 not a prognostic marker for hypopharyngeal squamous cell carcinoma.
    
    \emph{Archives of Otorhinolaryngology-Head \& Neck Surgery (2012)}
  \end{flushright}

  \tcblower

  \noindent \textbf{STEP IDENTIFICATION}

  {\small
  \begin{longtblr}[
      caption = {Common Prefixes},
      label = {tab:Common_Prefixes},
  ]{
      width = \textwidth,
      colspec = {X[1,c,h]  X[5,l,h]},
      rowhead = 1, rowfoot = 0, % 每个分页里表头表尾的数量
      % row{odd} = {blue8}, 
      row{even} = {azure9},
  }
      
    \toprule
    \textbf{Step} & \textbf{Sample} \\ 
    \midrule
    
     1S3a & \textbf{OBJECTIVE}: To investigate the prognostic significance of p16 in patients with hypopharyngeal squamous cell carcinoma (HPSCC) and to evaluate the relationship between p16 and human papillomavirus (HPV). \\
     1S2 & Unlike in oropharyngeal SCC (OPSCC), the prognostic significance of p16 in HPSCC and its association with HPV is unclear. \\

    \bottomrule

  \end{longtblr}
  }

  \noindent \textbf{ANALYSIS} 
  
  This sample presents the steps with orders different from Sample 1 and 2. It starts with to-infinitives showing the objectives, and proceeds with the research gap, "unclear"being a signal.

\end{sample}


\chapter{}\label{chapter4}
\chapter{}\label{chapter5}
\chapter{}\label{chapter6}

\restoregeometry

\backmatter



	
\end{document}