\documentclass{ctexbook}
\usepackage{imakeidx}
\makeindex[title=索引,intoc]
\usepackage[]{subcaption}
\usepackage[]{bicaption}
\usepackage{graphicx}
\usepackage[]{amsmath}
\usepackage[]{hyperref}
\usepackage[]{bigstrut}
\usepackage[]{xcolor}
\usepackage[]{multicol}
% \usepackage{chemfig}
\usepackage{mhchem}
\usepackage[tone,extra,safe]{tipa}
\usepackage[normalem]{ulem}
\usepackage{tabularray}
\UseTblrLibrary{booktabs}
% \usepackage[]{paracol}
\usepackage[]{lipsum}
\usepackage[]{zhlipsum}
\usepackage[]{sidenotes}
\usepackage[]{subfiles}
\usepackage[]{siunitx}
\usepackage[]{metalogo}
\usepackage{geometry} 
\geometry{left=2.5cm,right=2.5cm,top=2.5cm,bottom=2.5cm}




\begin{document}


\begin{titlepage}
    \vspace*{\stretch{1}}
    \begin{center}
      {\huge\bfseries Abstract Writing for Medical \\\vspace{3pt} Research Papers}\\[3ex]
      {\huge\bfseries 医学论著英语摘要写作}\\[6.5ex]
      {\large\bfseries Qi Hui, Chen Feina, Guo Haiyan}           \\
      \vspace{4ex}
    %   Thesis  submitted to                    \\[5pt]
      \textit{Fujian Medical University}                \\[2cm]

    %   \includegraphics[width=0.4\textwidth]{fig/fjmulogo.jpg}

    %   in partial fulfilment for the award of the degree of \\[2cm]
    %   \textsc{\Large Doctor of Philosophy}    \\[2ex]
    %   \textsc{\large Mathematics}             \\[12ex]
      \vfill
      Department of Arts and Sciences\\
    %   Address                                 \\
      \vfill
      \date{}
    \end{center}
    \vspace{\stretch{2}}
\end{titlepage}

\frontmatter
\chapter*{译者序}

本书是由福建医科大学文理艺术学院的齐晖、陈菲娜、郭海燕老师为主编,陈晶为学术秘书,交由复旦大学出版社出版,专供福医大学生使用的英语摘要写作教科书。我本人也是三位老师的学生,纵然课堂生动有趣、干货满满,但苦于同校前辈制作的扫描件观感不佳,笔记整理不便,译者决心要进行文字重排处理。其中自觉原书排版不善之处,皆进行重新编排,以符合译者审美。

原书通本以英文编写,编者似乎意图借此提升我等英语阅读水平,奈何文本中穿插语言学专有名词,初学时疲于翻译、苦不堪言。此外,期末复习期间,全英文本并不利于提升复习效率,故译者对主要文本进行翻译,对照复习。本套重置本将基于该译本进行整理,包含三种排版样式——原文重排版、双语对照版、译文版。以上三个版本请学弟学妹们按需取用,20届学长祝各位期末考试顺利。

本书为个人翻译作品,若发现纰漏,请在GitHub上提交\verb|issue|。在此声明,此书仅供学习交流使用,请勿用于商业用途或其他领域。

\vfill

\begin{flushright}
    neurocane\\
    2023/7/7
\end{flushright}

\newpage

\section*{编译环境}

\begin{itemize}
    \item \textbf{操作系统}: Windows
    \item \textbf{语言}: \LaTeX
    \item \textbf{编译环境}: \XeLaTeX
    \item \textbf{TeX Live版本}: TeX Live 2022
\end{itemize}

\tableofcontents

\mainmatter



\backmatter



	
\end{document}