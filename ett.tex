\documentclass{ctexbook}
\usepackage{imakeidx}
\makeindex[title=索引,intoc]
\usepackage[]{subcaption}
\usepackage[]{bicaption}
\usepackage{graphicx}
\usepackage[]{amsmath}
\usepackage[]{hyperref}
\hypersetup{colorlinks, linkcolor=blue}
\usepackage[]{bigstrut}
\usepackage[]{xcolor}
\usepackage[]{multicol}
% \usepackage{chemfig}
\usepackage{mhchem}
\usepackage[tone,extra,safe]{tipa}
\usepackage[normalem]{ulem}
\usepackage{tabularray}
\UseTblrLibrary{booktabs}
% \usepackage[]{paracol}
\usepackage[]{lipsum}
\usepackage[]{zhlipsum}
\usepackage[]{sidenotes}
\usepackage[]{subfiles}
\usepackage[]{siunitx}
\usepackage[]{metalogo}
\usepackage[]{tcolorbox}
\tcbuselibrary{skins}
\usepackage{geometry} 
\geometry{left=2.5cm,right=2.5cm,top=2.5cm,bottom=2.5cm}

\newtcolorbox[auto counter,number within=section]{sample}[2][]{%
arc is angular,skin=bicolor,colbacklower=white,colback=blue!8!white,colframe=blue!46!black, arc=1mm,fonttitle=\bfseries, title={\heiti Sample}~\thetcbcounter: #2,#1}


\begin{document}


\begin{titlepage}
    \vspace*{\stretch{1}}
    \begin{center}
      {\huge\bfseries Abstract Writing for Medical \\\vspace{3pt} Research Papers}\\[3ex]
      {\huge\bfseries 医学论著英语摘要写作}\\[6.5ex]
      {\large\bfseries Qi Hui, Chen Feina, Guo Haiyan}           \\
      \vspace{4ex}
    %   Thesis  submitted to                    \\[5pt]
      \textit{Fujian Medical University}                \\[2cm]

    %   \includegraphics[width=0.4\textwidth]{fig/fjmulogo.jpg}

    %   in partial fulfilment for the award of the degree of \\[2cm]
    %   \textsc{\Large Doctor of Philosophy}    \\[2ex]
    %   \textsc{\large Mathematics}             \\[12ex]
      \vfill
      Department of Arts and Sciences\\
    %   Address                                 \\
      \vfill
      \date{}
    \end{center}
    \vspace{\stretch{2}}
\end{titlepage}

\frontmatter
\chapter*{译者序}

本书是由福建医科大学文理艺术学院的齐晖、陈菲娜、郭海燕老师为主编,陈晶为学术秘书,交由复旦大学出版社出版,专供福医大学生使用的英语摘要写作教科书。我本人也是三位老师的学生,纵然课堂生动有趣、干货满满,但苦于同校前辈制作的扫描件观感不佳,笔记整理不便,译者决心要进行文字重排处理。其中自觉原书排版不善之处,皆进行重新编排,以符合译者审美。

原书通本以英文编写,编者似乎意图借此提升我等英语阅读水平,奈何文本中穿插语言学专有名词,初学时疲于翻译、苦不堪言。此外,期末复习期间,全英文本并不利于提升复习效率,故译者对主要文本进行翻译,对照复习。本套重置本将基于该译本进行整理,包含三种排版样式——原文重排版、双语对照版、译文版。以上三个版本请学弟学妹们按需取用,20届学长祝各位期末考试顺利。

本书为个人翻译作品,若发现纰漏,请在GitHub上提交\verb|issue|。在此声明,此书仅供学习交流使用,请勿用于商业用途或其他领域。

\vfill

\begin{flushright}
    neurocane\\
    2023/7/7
\end{flushright}

\newpage

\section*{编译环境}

\begin{itemize}
    \item \textbf{操作系统}: Windows
    \item \textbf{语言}: \LaTeX
    \item \textbf{编译环境}: \XeLaTeX
    \item \textbf{TeX Live版本}: TeX Live 2022
\end{itemize}

\tableofcontents

\mainmatter

\newgeometry{includeall,left=2cm,right=1cm,top=2.5cm,bottom=2.5cm,textwidth=12cm,marginparwidth=5cm}

\chapter{Overview of Abstracts}
\section{Definition of an Abstract}

The American Psychological Association (APA) Style (2010) states that an abstract is a brief, comprehensive summary of the content of an article. According to the American National Standards Institute (1979), an abstract is an abbreviated accurate representation of the content of a document, preferably prepared by its author(s) for publication with it. In general, an abstract is a concise, accurate and comprehensive statement of the content of an article. It is original rather than excerpted.

\section{Importance and Functions of an Abstract}

An abstract is a distinct genre, and to some extent plays a pivotal role in academic readingand writing. In the era of information explosion, an enormous number of new publications are produced in the academic community each day. There is no practical way for every reader to get access to every new article, or to read every new publication even if it is accessible. Theabstracts published online, which are concise and comprehensive, can be obtained easily andquickly. Abstract reading, then, may be a useful starting point of any academic reading and writing. In this sense, an abstract is the most read part of an article.\sidenote{劳仑衣普桑,认至将指点效则机,最你更枝。想极整月正进好志次回总般,段然取向使张规军证回,世市总李率英茄持伴。}

An abstract has at least three functions (Huckin, 2001). First, it serves as a stand-alone mini text, giving readers a quick summary of a study's objectives, methodology, findings and conclusions, which are the major components of abstracts. Second, it serves as a screening device, and gives readers an adequate view on whether the full-length article is of great value to their needs and worth further reading. A good abstract, to some extent, increases the chance of being cited or referenced. Third, for those readers who do opt to read the article as a whole, the abetract serves as a preview, creating an interpretive frame that can guide reading.

\section{Types of Abstracts}

Generally, abstracts fall into two categories, indicative and informative, depending on the type of information they convey. A typical distinction between them is that the indicative abstract, viewed as the outline of the paper, is usually shorter and simpler, while the informative abstract, viewed as the summary of the paper is usually longer and more thoroagh. 

These two types of abstracts also differ in the components they contain. Indicative abstracts often include the purpose, scope, and methods of the report or study, but seldom include theresults or conclusions. Reading indicative abstracts could not substitute reading the paper, because not all the crucial components are covered. It is more widely used in social science papers. On the other hand, informative abstracts usually include all the crucial components of the study, such as the background, purpose, methods, results, and conclusions. It is the type of abstracts widely used in medical field. In this book, we focus on the writing of informative abstracts. The abstracts referred to in the following chapters are informative abstracts.

\section{Types of Informative Abstracts}

There are two types of informative abstracts, structured abstracts and unstructured abstracts.

Where a heading or label is used at the beginning of the text in each section, it is a structured abstract. Each section is usually written in a separate paragraph, but sometimes sections are written in a sole or continuous paragraph. Headings might be background, objectives, methods, results, conclusions, and so on. They vary according to the criteria set by different journals. Structured abstracts appear to be favored by medically-relevant publications.

Where no heading or label is used to indicate different parts of an abstract, it is an unstructured abstract. It is always a sole paragraph. The major difference between the two types of abstracts lies in whether there are headings or not. In an unstructured abstract the content and sequence of the items are written as it is in the structured one.

Journals mandate which style should be used, so check the author guidelines if you' re not sure. If it is not mentioned, keep an eye out for the type of abstracts preferable in the journals where you are willing to have your paper submitted and published. Write your abstracts in the style which dominates.

% \tcbset{colback=blue!8!white,colframe=blue!46!black,
% arc=1mm}
% \begin{tcolorbox}[arc is angular,skin=bicolor,colbacklower=white,title={\heiti 例题}]
% 这是一个 \textbf{tcolorbox}。
% \tcblower
% 这个是下部
% \end{tcolorbox}


\begin{sample}[label={myautocounter}]{\heiti}
  \textbf{BACKGROUND} 
  
  In patients with acute heart failure, early intervention with an intravenous vasodilator has been proposed as a therapeutic goal to reduce cardiac-wall stress and, potentially, myocardialinjury, thereby favorably affecting patients' long-term prognosis.

  \textbf{METHODS} 
  
  In this double-blind trial, we randomly assigned 2, 157 patients with acute heart failure to receive a continuous intravenous infusion of either ularitide at a dose of 15 ng per kilogram of body weight per minute or matching placebo for 48 hours, in addition to accepted therapy.
  Treatment was initiated a median of 6 hours after the initial clinical evaluation. The coprimaryd outcomes were death from cardiovascular causes during a median follow-up of 15 months and a hierarchical composite end point that evaluated the initial 48-hour clinical course.
  
  \textbf{RESULTS} 
  
  Death from cardiovascular causes occurred in 236 patients in the ularitide group and 225 patients in the placebo group (21.7\% vs. 21.0\%; hazard ratio, 1.03;96\% confidence interval, 0.85 to 1.25; P=0.75). In the intention-to-treat analysis, there was no significant between-group difference with respect to the hierarchical composite outcome. The ularitide group had greater reductions in systolic blood pressure and in levels of N-terminal pro-brain natriuretic peptide than the placebo group. However, changes in cardiac troponin T levels during the infusion did not differ between the two groups in the 55\% of patients with paired data.

  \textbf{CONCLUSIONS} 
  
  In patients with acute heart failure, ularitide exerted favorable physiological effects (without affecting cardiac troponin levels), but short-term treatment did not affect a clinical composite end point or reduce long-term cardiovascular mortality.

  \begin{flushright}
    ---Effect of Ularitide on Cardiovascular Mortality in Acute Heart Failure.

    \emph{New England Journal of Medicine (2017)}
  \end{flushright}

  
\end{sample}

\chapter{}
\chapter{}
\chapter{}
\chapter{}
\chapter{}

\restoregeometry

\backmatter



	
\end{document}