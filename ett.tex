\documentclass[a4paper]{ctexbook}
\usepackage{imakeidx}
\makeindex[title=索引,intoc]
\usepackage{tocbibind}
\usepackage[]{subcaption}
\usepackage[]{bicaption}
\usepackage{graphicx}
\usepackage[]{amsmath}
\usepackage[]{amssymb}
\usepackage[]{bigstrut}
\usepackage[dvipsnames,svgnames]{xcolor}
\usepackage[]{multicol}
\usepackage{mhchem}
\usepackage[tone,extra,safe]{tipa}
\usepackage[normalem]{ulem}
% \usepackage[]{ninecolors}
% \NineColors{saturation=low}
% \usepackage[figuresright]{rotating}
\usepackage[]{lscape}
\usepackage{tabularray}
\UseTblrLibrary{booktabs}
% \usepackage[]{paracol}
\usepackage[]{lipsum}
\usepackage[]{zhlipsum}
\usepackage[]{sidenotes}
\usepackage[]{subfiles}
\usepackage[]{siunitx}
\usepackage[]{metalogo}
\usepackage[]{tcolorbox}
\tcbuselibrary{skins}
\tcbuselibrary{breakable}
\tcbuselibrary{documentation}
\usepackage{geometry} 
\geometry{left=2.5cm,right=2.5cm,top=2.5cm,bottom=2.5cm}

\newtcolorbox[auto counter,number within=section]{sample}[2][]{%
arc is angular,breakable,parbox=false,before upper=\par,before lower=\par,drop shadow,skin=bicolor,colbacklower=white,colback=blue!8!white,colframe=blue!46!black, arc=1mm,fonttitle=\bfseries, title={\heiti Sample}~\thetcbcounter: #2,#1}

\newtcolorbox[auto counter,number within=section]{note}[2][]{%
breakable,parbox=false,before upper=\par,before lower=\par,drop shadow,skin={spartan,bicolor},colbacklower=white,colback=red!3!white,colframe=brown!46!black,fonttitle=\bfseries, title={\heiti Note}~\thetcbcounter: #2,#1}

\newtcolorbox[auto counter,number within=section]{task}[2][]{%
colback=Emerald!10,colframe=cyan!40!black,fonttitle=\bfseries,title={\heiti Task}~\thetcbcounter: #2,#1}

\newtcolorbox[auto counter,number within=subsection]{eg}[2][]{%
attach title to upper,coltitle=black,arc=0mm,top=0mm,boxrule=0pt,beforeafter skip balanced=0pt,colback=Thistle!80!red!20!white,colframe=white,fonttitle=\upshape\bfseries,fontupper=\itshape,title={\heiti E.g.}~\thetcbcounter:\ #2,#1}

  
\usepackage[]{adforn}
% \renewcommand{\chaptername}{第 \thechapter{} 章}
% \newcommand{\exercisename}{Exercice}
\newenvironment{problemset}[1][\chaptername~\;Exercice]{
  \begin{center}
    \phantomsection\addcontentsline{toc}{section}{\texorpdfstring{Chapter~\;Exercice}{Exercice}}
    % \markboth{#1}{\rightmark}
    \markright{#1}
    \textcolor{black}{\Large\bfseries\adftripleflourishleft~#1~\adftripleflourishright}
  \end{center}
  \begin{enumerate}}{
  \end{enumerate}}


\usepackage[]{hyperref}
\hypersetup{colorlinks, linkcolor=blue}

\begin{document}


\begin{titlepage}
    \vspace*{\stretch{1}}
    \begin{center}
      {\huge\bfseries Abstract Writing for Medical \\\vspace{3pt} Research Papers}\\[3ex]
      {\huge\bfseries 医学论著英语摘要写作}\\[6.5ex]
      {\large\bfseries Qi Hui, Chen Feina, Guo Haiyan}           \\
      \vspace{4ex}
    %   Thesis  submitted to                    \\[5pt]
      \textit{Fujian Medical University}                \\[2cm]

    %   \includegraphics[width=0.4\textwidth]{fig/fjmulogo.jpg}

    %   in partial fulfilment for the award of the degree of \\[2cm]
    %   \textsc{\Large Doctor of Philosophy}    \\[2ex]
    %   \textsc{\large Mathematics}             \\[12ex]
      \vfill
      Department of Arts and Sciences\\
    %   Address                                 \\
      \vfill
      \date{}
    \end{center}
    \vspace{\stretch{2}}
\end{titlepage}

\frontmatter
\chapter*{前言}\addcontentsline{toc}{chapter}{前言}

近年来,针对医学人才培养和大学外语教育教学,国家层面提出新的指导思想。2012年教育部和原卫生部颁布《关于实施卓越医生教育培养计划的意见》,2018年教育部、国家卫生健康委员会、国家中医药管理局发布《关于加强医教协同实施卓越医生教育培养计划2.0的意见》,指出卓越医师培养在语言层面的要求为“拓展医学生的国际视野,为培养高层次、国际化的医学拔尖创新人才奠定基础”。2017年教育部高等学校大学英语教学指导委员会发布的《大学英语教学指南》提出“学生可以通过学习与专业或未来工作有关的学术英语或职业英语,获得在学术或职业领域进行交流的相关能力”的要求。

根据国家政策的新要求,我们将医学论著英语摘要写作纳入本科生医学英语教学,但是教学过程中发现目前市面上几乎没有合适的基于语料库的医学英语论文摘要写作教材。为解决此问题,我们组建编写团队,依托省级教改项目,进行基于语料库的教材编写。

本教材属于专门用途英语课程教材,可供医学英语摘要写作教学使用,是依托医学论著英语摘要语料库编写的教材。

编写团队成员首先基于语料库语言学理论,建设 115万个词的医学论著英语摘要语料库,为学生医学英语学习提供海量真实的语言数据。为了确保语料的代表性,根据教育部 《学位授予和人才培养学科目录》医学专业一级学科“临床医学”下设的18个二级学科,收集了2016-2018年来自各二级学科影响因子位列前10名的 $2 \sim 3$ 本国际期刊的论著摘要,每个二级学科的摘要篇数基本相同。基于自建语料库,运用体裁分析理论,从宏观层面分析医学论著摘要的篇章结构,并采用定性与定量分析相结合的方法,从微观层面归纳每个语步和语阶的典型语言实现方式。

教材分为六章。第一章作为导入章节,介绍医学论著摘要的定义、重要性及分类;第二章讲述医学论著摘要的语步和语阶的划分;第三至六章分别具体讲授医学论著英语摘要四个语步中每个语阶的语言实现方式。依据在语料库中的出现频率,对每个语步中的常用词、词块等予以排序。教材中所有案例及练习均为该库中的真实语料。

本教材突出体现以下特点: 

1. 教学资源真实,具有科学性 

我们将基于语料库的定量方法与定性观察话语及语境特征的方法相结合。基于海量真实的语料进行研究,强调了经验的重要性,降低了编写者直觉对数据选择的影响,使结果更具说服力。

2. 内容编排合理,具有逻辑性 

教材内容各部分结构平衡协调、浑然一体,能帮助学生了解并掌握医学论著英语摘要的词汇、句法和篇章层面的基本架构。第一章介绍医学论著摘要写作的基本概念;第二章明确摘要中的语步及语阶,四个语步分别为:语步1 (M1) 确定研究领域,语步2 (M2) 描述研究过程,语步3 (M3) 总结研究结果,语步4 (M4) 讨论研究结果;第四章至第六章分别从语用功能、语言实现及案例分析等方面,对每个语步和语阶进行详细讲解。课后练习形式多样,能帮助学生巩固所学知识。

\chapter*{译者序}\addcontentsline{toc}{chapter}{译者序}


本书是由福建医科大学文理艺术学院的齐晖、陈菲娜、郭海燕老师为主编,陈晶为学术秘书,交由复旦大学出版社出版,专供福医大学生使用的英语摘要写作教科书。我本人也是三位老师的学生,纵然课堂生动有趣、干货满满,但苦于同校前辈制作的扫描件观感不佳,笔记整理不便,译者决心要进行文字重排处理。其中自觉原书排版不善之处,皆进行重新编排,以符合译者审美。

原书通本以英文编写,编者似乎意图借此提升我等英语阅读水平,奈何文本中穿插语言学专有名词,初学时疲于翻译、苦不堪言。此外,期末复习期间,全英文本并不利于提升复习效率,故译者对主要文本进行翻译,对照复习。本套重置本将基于该译本进行整理,包含三种排版样式——原文重排版、双语对照版、译文版。

原文重排版和译文版保留了较大的边注区域,便于读者阅读时进行笔记标注,同时译者在此区域添加了部分专有名词的解释。同时,根据排版需要设计了多种类型的盒子对文本区块进行包裹,提升视觉效果的同时,排版的连续性改善了部分区域注意力被引导分散的问题。

本书初稿为个人翻译作品,在此感谢后期参与校对工作的朋友们。若发现纰漏,请在本项目 \href{https://github.com/Neurocane/Abstract-Writing-for-Medical-Research-Papers}{GitHub} 上提交\verb|issue|~\sout{提交了也不一定马上改,咕咕咕}。在此声明,此套图书仅供学习交流使用,请勿用于商业用途或其他领域。本项目工作人员均为爱发电,若发现其他平台账号引流或要求付费购买,都是要被钉在耻辱柱上的。原书内容的著作权归原书作者所有,请多支持正版原著。编辑此书的模板,您可以基于 \href{https://www.latex-project.org/lppl/lppl-1-3c/}{\LaTeX 项目公共许可证 v1.3c} 进行修改和使用。

以上三个版本请学弟学妹们按需取用,20届的老前辈们祝各位期末考试顺利。

\vfill

\begin{flushright}
    \emph{Neurocane}\\
    2023/7/7
\end{flushright}

\newpage

\section*{编译环境}

\begin{itemize}
    \item \textbf{操作系统}: Windows
    \item \textbf{语言}: \LaTeX
    \item \textbf{编译环境}: \XeLaTeX
    \item \textbf{TeX Live版本}: TeX Live 2022
\end{itemize}

\section*{命令环境注释}

\paragraph*{e.g.环境} 标签命名规则:eg:\verb|<语步/语阶缩写>|-\verb|本章节e.g.环境序号|,如:\verb|label={eg:1s1-1}|

\tableofcontents

\mainmatter

\newgeometry{includeall,left=2cm,right=1cm,top=2.5cm,bottom=2.5cm,textwidth=12cm,marginparwidth=5cm}

\chapter{Overview of Abstracts}\label{chapter1}
\section{Definition of an Abstract}

The American Psychological Association (APA) Style (2010) states that an abstract is a brief, comprehensive summary of the content of an article. According to the American National Standards Institute (1979), an abstract is an abbreviated accurate representation of the content of a document, preferably prepared by its author(s) for publication with it. In general, an abstract is a concise, accurate and comprehensive statement of the content of an article. It is original rather than excerpted.

\section{Importance and Functions of an Abstract}

An abstract is a distinct genre, and to some extent plays a pivotal role in academic readingand writing. In the era of information explosion, an enormous number of new publications are produced in the academic community each day. There is no practical way for every reader to get access to every new article, or to read every new publication even if it is accessible. Theabstracts published online, which are concise and comprehensive, can be obtained easily andquickly. Abstract reading, then, may be a useful starting point of any academic reading and writing. In this sense, an abstract is the most read part of an article.

An abstract has at least three functions (Huckin, 2001). First, it serves as a stand-alone mini text, giving readers a quick summary of a study's objectives, methodology, findings and conclusions, which are the major components of abstracts. Second, it serves as a screening device, and gives readers an adequate view on whether the full-length article is of great value to their needs and worth further reading. A good abstract, to some extent, increases the chance of being cited or referenced. Third, for those readers who do opt to read the article as a whole, the abetract serves as a preview, creating an interpretive frame that can guide reading.

\section{Types of Abstracts}

Generally, abstracts fall into two categories, indicative and informative, depending on the type of information they convey. A typical distinction between them is that the indicative abstract, viewed as the outline of the paper, is usually shorter and simpler, while the informative abstract, viewed as the summary of the paper is usually longer and more thoroagh. 

These two types of abstracts also differ in the components they contain. Indicative abstracts often include the purpose, scope, and methods of the report or study, but seldom include theresults or conclusions. Reading indicative abstracts could not substitute reading the paper, because not all the crucial components are covered. It is more widely used in social science papers. On the other hand, informative abstracts usually include all the crucial components of the study, such as the background, purpose, methods, results, and conclusions. It is the type of abstracts widely used in medical field. In this book, we focus on the writing of informative abstracts. The abstracts referred to in the following chapters are informative abstracts.

\section{Types of Informative Abstracts}

There are two types of informative abstracts, structured abstracts and unstructured abstracts.

Where a heading or label is used at the beginning of the text in each section, it is a structured abstract. Each section is usually written in a separate paragraph, but sometimes sections are written in a sole or continuous paragraph. Headings might be background, objectives, methods, results, conclusions, and so on. They vary according to the criteria set by different journals. Structured abstracts appear to be favored by medically-relevant publications.

Where no heading or label is used to indicate different parts of an abstract, it is an unstructured abstract. It is always a sole paragraph. The major difference between the two types of abstracts lies in whether there are headings or not. In an unstructured abstract the content and sequence of the items are written as it is in the structured one.

Journals mandate which style should be used, so check the author guidelines if you' re not sure. If it is not mentioned, keep an eye out for the type of abstracts preferable in the journals where you are willing to have your paper submitted and published. Write your abstracts in the style which dominates.

% \tcbset{colback=blue!8!white,colframe=blue!46!black,
% arc=1mm}
% \begin{tcolorbox}[arc is angular,skin=bicolor,colbacklower=white,title={\heiti 例题}]
% 这是一个 \textbf{tcolorbox}。
% \tcblower
% 这个是下部
% \end{tcolorbox}


\begin{sample}[label={myautocounter}]{\heiti}
  \textbf{BACKGROUND} 
  
  In patients with acute heart failure, early intervention with an intravenous vasodilator has been proposed as a therapeutic goal to reduce cardiac-wall stress and, potentially, myocardialinjury, thereby favorably affecting patients' long-term prognosis.

  \textbf{METHODS} 
  
  In this double-blind trial, we randomly assigned 2, 157 patients with acute heart failure to receive a continuous intravenous infusion of either ularitide at a dose of 15 ng per kilogram of body weight per minute or matching placebo for 48 hours, in addition to accepted therapy.
  Treatment was initiated a median of 6 hours after the initial clinical evaluation. The coprimaryd outcomes were death from cardiovascular causes during a median follow-up of 15 months and a hierarchical composite end point that evaluated the initial 48-hour clinical course.
  
  \textbf{RESULTS} 
  
  Death from cardiovascular causes occurred in 236 patients in the ularitide group and 225 patients in the placebo group (21.7\% vs. 21.0\%; hazard ratio, 1.03;96\% confidence interval, 0.85 to 1.25; P=0.75). In the intention-to-treat analysis, there was no significant between-group difference with respect to the hierarchical composite outcome. The ularitide group had greater reductions in systolic blood pressure and in levels of N-terminal pro-brain natriuretic peptide than the placebo group. However, changes in cardiac troponin T levels during the infusion did not differ between the two groups in the 55\% of patients with paired data.

  \textbf{CONCLUSIONS} 
  
  In patients with acute heart failure, ularitide exerted favorable physiological effects (without affecting cardiac troponin levels), but short-term treatment did not affect a clinical composite end point or reduce long-term cardiovascular mortality.

  \begin{flushright}
    ---Effect of Ularitide on Cardiovascular Mortality in Acute Heart Failure.

    \emph{New England Journal of Medicine (2017)}
  \end{flushright}

  
\end{sample}

\begin{sample}[label={myautocounter}]{\heiti}
  \textbf{OBJECTIVE} 
  
  To evaluate the association between the parameters of 24-hour multichannel intraluminal impedance (MII)-pH monitoring and the symptoms or quality of life (QoL) in laryngopharyngeal reflux (LPR) patients.

  \textbf{DESIGN} 
  
  Prospective cohort study without controls.

  \textbf{SETTING} 
  
  University teaching hospital.

  \textbf{METHODS} 
  
  Forty-five LPR patients were selected from subjects who underwent 24-hour MII-pH monitoring and were diagnosed with LPR from September 2014 to May 2015. Reflux Symptom Index (RSI), Health-related Quality of Life (HRQoL), Short Form 12 (SF-12) Survey questionnaires were surveyed. Spearman's correlation was used to analyse the association between the symptoms or QoL and 24-hour MII-pH monitoring.

  \textbf{RESULTS}

  Most parameters in 24-hour MII-pH monitoring showed weak or no correlation with RSI, HRQoL and SF-12. Only number of non-acid reflux events that reached the larynx and pharynx (LPR-non-acid) and number of total reflux events that reached the larynx and pharynx (LPR-total) parameters showed strong correlation with heartburn in RSI (R=0.520, $P<0.001$, R=0.478, $P=0.001$, respectively). Multiple regression analysis showed that there was only one significant regression coefficient between LPR-non-acid and voice/hoarseness portion of HRQoL (b=1.719, $P=0.022$).

  \textbf{CONCLUSION} 

  Most parameters of 24-hour MII-pH monitoring did not reflect subjective symptoms or QoL in patients with LPR.

  \begin{flushright}
    ---Association between 24-hour combined multichannel intraluminal impedance-pH monitoring and symptoms or quality of life in patients with laryngopharyngeal reflux. 
    
    \emph{Clinical Otolaryngology (2017)}
  \end{flushright}

  
\end{sample}

\begin{sample}[label={myautocounter}]{\heiti}
  Due to the high incidence of recurrent squamous cell carcinoma of the head and neck andd the toxicity profile of current salvage regimens, there is a need for tolerable and effective treatment options. We performed a retrospective matched case series to report our experience with recurrent high-risk patients who received capecitabine (CAP) therapy in the adjuvant setting after salvage therapy. The 5-year recurrence-free survival rates for the CAP and control cohorts were 54\% (95\% CI, 0.27\%--0.75\%) and 27\% (95\% CI, 0.09\%--0.50\%), respectively.Multivariable Cox modeling showed a significant improvement in recurrence-free survival in thed CAP cohort (hazard ratio, 0.19; 95\% CI, 0.04--0.92; $P=.0 392$). While this was a respective analysis that could not control for all variables, these exploratory findings offer insights that may inform a prospective study to determine CAP efficacy.
  
  \begin{flushright}
    ---Capecitabine after Surgical Salvage in Recurrent Squamous Cell Carcinoma of Head and Neck. 
    
    \emph{Otolaryngology---Head \& Neck Surgery (2017)}
  \end{flushright}


\end{sample}

\begin{note}[label={myautocounter}]{\heiti Corpus used for this book}
  The data used and analyzed in this book are from a custom-built corpus with 1.15 million tokens of medical research article (RA) abstracts. The discipline of medicine is divided into 18 sub-disciplines, and RA abstracts from 2 to 3 leading journals are randomly retrieved in each sub-discipline with relatively similar number of texts for each sub-discipline (\autoref{tab:Sub-disciplines and journals in each sub-discipline}). The journals selected are all with relatively high impact factors.

\end{note}

{\small
  \begin{longtblr}[
      caption = {Sub-disciplines and journals in each sub-discipline},
      label = {tab:Sub-disciplines and journals in each sub-discipline},
  ]{
      width = \textwidth,
      colspec = {X[1,l,h]  X[2,l,h]},
      rowhead = 1, rowfoot = 0, % 每个分页里表头表尾的数量
      % row{odd} = {blue8}, 
      row{even} = {LemonChiffon},
      column{2} = {font=\itshape}
  }
    
    \toprule
    Sub-discipline & Journal \\
    \midrule

    Anesthesiology & {British Journal of Anaesthesia \\ Anesthesiology \\ Anesthesia and Analgesia}\\
    Dermatology & {Journal of American Academy of Dermatology \\ Giornale Italiano di Dermatologia e Venereologia}\\
    Emergency Medicine & {Annals of Emergency Medicine \\ Internal and Emergency Medicine \\ Academic Emergency Medicine}\\
    Geriatrics & {Neurobiology of Aging \\ Aging Cell \\ Age and Ageing}\\
    Internal Medicine & {The New England Journal of Medicine \\ The Lancet \\ JAMA-Journal of the American Medical Association}\\
    Medical Imaging & {The Journal of Nuclear Medicine \\ Investigative Radiology \\ Radiology}\\
    Medical Laboratory & {Clinical Chemistry and Laboratory Medicine \\ Clinical Biochemistry}\\
    Neurology & {The Lancet Neurology \\ Annals of Neurology}\\
    Obstetrics and Gynecology & {Obstetrics \& Gynecology \\ American Journal of Obstetrics \& Gynecology \\ An International Journal of Obstetrics \& Gynecology}\\
    Oncology & {Journal of Clinical Oncology \\ The lancet Oncology}\\
    Ophthalmology & {Ophthalmology \\ American Journal of Ophthalmology \\ Archives of Ophthalmology}\\
    Otolaryngology (ENT) & {Head \& Neck \\ Clinical Otolaryngology \\ Otolaryngology---Head \& Neck Surgery}\\
    Pain Medicine & {The Clinical Journal of Pain \\ Pain Medicine \\ Regional Anesthesia and Pain Medicine}\\
    Pediatrics & {Journal of the American academy of child \& Adolescent psychiatry \\ Pediatrics \\ JAMA pediatrics}\\
    Physical medicine and rehabilitation & {Neurorehabilitation and neural repair \\ Journal of fluency disorders}\\
    Psychiatry & {Molecular psychiatry \\ The American journal of psychiatry \\ JAMA psychiatry}\\
    Sports medicine & {Medicine and Science in Sports and Exercise \\ Sports Medicine \\ The American Journal of Sports Medicine}\\
    Surgery & {Annals of Surgery \\ American Journal of Transplantation \\ Journal of Neurology, Neurosurgery \& Psychiatry}\\

    \bottomrule

  \end{longtblr}
  }

  \begin{task}[label={myautocounter}]{\heiti Corpus-based task}
    Can you build your own corpus with at least 100, 000 tokens?
  \end{task}

  \begin{note}[label={myautocounter}]{\heiti Corpus used for this book}
    In this book, "corpus-based tasks" are designed to enhance your ability to explore language realizations of medical RA abstracts with corpus approach. Most of these tasks might require the use of software such as AntConc, WordSmith, and so on.
  \end{note}

  \section{Glossary}

  {\small
  \begin{longtblr}[
      caption = {Glossary of Chapter 1},
      label = {tab:Glossary of Chapter 1},
  ]{
      width = \textwidth,
      colspec = {X[1,l,h]  X[1,l,h]  X[3,l,h]},
      rowhead = 1, rowfoot = 0, % 每个分页里表头表尾的数量
      % row{odd} = {blue8}, 
      row{even} = {azure9},
  }
      
  \toprule
  \textbf{WORDS} & \textbf{MEANING} & \textbf{MEANING or EXAMPLE}\\
  \midrule

  \textbf{excerpt}/\textipa{"eks3:pt}/ & \emph{v.} 摘录;引用  & If a long piece of writing or music is excerpted, short pieces from it are printed or played on their own. \\
  \textbf{genre}/\textipa{"ZA:nr@}/ & \emph{n.} 体裁 & a particular type of art, writing, music etc, which has certain features that all examples of this type share.\\
  {\textbf{mandate} \\ /\textipa{"m\ae ndeIt}/} & \emph{v.} 授权;强制执行;委托办理 & to tell someone that they must do a particular thing. \\
  {\textbf{methodology} \\ /\textipa{""meT@"d6l@\textdyoghlig i}/} & \emph{n.} 方法学 & a set of methods and principles used to perform a particular activity. \\
  \textbf{opt}/\textipa{6pt}/ & \emph{v.} 选择;挑选 & to choose one thing or do one thing instead of another \\
  \textbf{pivotal}/\textipa{"pIv@tl}/ & \emph{adj.} 关键性的;核心的 & more important than anything else in a situation or system. \\
  
  \bottomrule

  \end{longtblr}
  }

  \begin{problemset}
    \item \textbf{Identify whether the following abstracts are structured or unstructured and tell the reasons.}
    
    \textbf{Abstract 1}

    \hspace*{2em}Objectives: The aim of this study was to analyze changes in health care utilization and cost among a sample of highly impaired children and adolescents who sought a 3-week intensive interdisciplinary pain treatment (IPT).

    \hspace*{2em}Materials and Methods: Claims data from 7 statutory health insurance companies were analyzed for 65 children and adolescents who sought IIPT at the German Paediatric Pain Centre. The annual health care utilization and cost were determined for the following 4 areas: outpatient care, inpatient care, medications, and remedies and aids. We analyzed the changes in resource utilization in the year before (pre\_1 y) IPT and in the subsequent year (post\_1 y).
    
    \hspace*{2em}Results: Within the first year after IPT, overall health care costs did not decrease significantly. However, the pattern of health care utilization changed. First, significantly more children and adolescents started outpatient psychotherapy ($P=0.001$). Second, the number of hospitalized children decreased significantly from 1-year pre to 1-year post ($P=0.001$). Accordingly, there were significantly fewer hospitalizations for primary chronic pain disorders at 1-year post ($P<0.001$). The prescription of nonopioids, co-analgesics and opioids was significantly reduced from 1-year pre to 1-year post (all $P<0. 013$).
    
    \hspace*{2em}Discussion: The present results indicate that the health care costs of children and adolescents with severe chronic pain disorders do not significantly decrease 1 year after IPT; however, the treatment becomes more goal-focused. Differential diagnosis measures and nonindicated therapeutic interventions decreased, and more indicated interventions, such as psychotherapy, were used. Future research is needed to investigate the economic long-term changes after IPT.
    
    \begin{flushright}
      ---Health Care Utilization and Cost in Children and Adolescents with Chronic Pain: Analysis of Health Care Claims Data 1 Year Before and After Intensive Interdisciplinary Pain Treatment. 
      
      \emph{The Clinical Journal of Pain (2017)}
    \end{flushright}

    \textbf{Abstract 2}

    \hspace*{2em}Previous studies of brain structure in Tourette syndrome (TS) have produced mixed results, and most had modest sample sizes. In the present multicenter study, we used structural magnetic resonance imaging (MRI) to compare 103 children and adolescents with TS to a well-matched group of 103 children without tics. We applied voxel-based morphometry methods to test gray matter (GM) and white matter (WM) volume differences between diagnostic groups, accounting for MRI scanner and sequence, age, sex and total GM$+$WM volume. The TS group demonstrated lower WM volume bilaterally in orbital and medial prefrontal cortex, and greater GM volume in posterior thalamus, hypothalamus and midbrain. These results demonstrate evidence for abnormal brain structure in children and youth with TS, consistent with and extending previous findings, and they point to new target regions and avenues of study in TS. For example, as orbital cortex is reciprocally connected with hypothalamus, structural abnormalities in these regions may relate to abnormal decision making, reinforcement learning or somatic processing in TS.

    \begin{flushright}
      ---Brain structure in pediatric Tourette syndrome. 

      \emph{Molecular Psychiatry (2017)}
    \end{flushright}

    \textbf{Abstract 3}

    \hspace*{2em}Objective: To assess the feasibility of detecting signature volatile organic compounds in the breath of patients with oral squamous cell carcinoma.
    
    \hspace*{2em}Study Design: Prospective cohort pilot study.
    
    \hspace*{2em}Setting: University hospital.
    
    \hspace*{2em}Subjects and Methods: Using gas chromatography and mass spectrometry, emitted volatile organic compounds in the breath of patients before and after curative surgery (n=10) were compared with those of healthy subjects (n=4). It was hypothesized that certain volatile organic compounds disappear after surgical therapy. A characteristic signature of these compounds for diseased patients was compiled and validated.
    
    \hspace*{2em}Results: Breath analyses revealed 125 volatile organic compounds in patients with oral cancer. A signature of 8 compounds that were characteristic for patients with oral cancer could be detected: 3 from this group presented were absent after surgery.
    
    \hspace*{2em}Conclusion: The presented results confirmed the hypothesis of an absence of cancer-associated volatile organic compounds in the breath after therapy. In this pilot study, we proved the feasibility of this test approach. Further studies should be initiated to establish protocols for usage in a clinical setting.
    
    \begin{flushright}
      ---Volatile Organic Compounds in the Breath of Oral Squamous Cell Carcinoma Patients: A Pilot Study. 

      \emph{Otolaryngology Head and Neck Surgery (2017)}
    \end{flushright}
   
  \end{problemset}


\chapter{Move and Step Identification}\label{chapter2}

\section{Move Identification}

A medical RA abstract consists of moves which work together to achieve its communicative purposes. A move in this sense is ``a section of a text that performs a specific communicative function'' (Kanoksilapatham, 2007, p.23).

A four move scheme is used in most medical RA abstracts, with move 1 (M1) creating a research space move 2 (M2) describing research process, move 3 (M3) summarizing principal results and move 4 (M4) drawing conclusions. All of them are conventional moves in medical RA abstracts.

In a structured abstract, M2, M3 and M4 could be easily recognized via the headings of methods, results and conclusions. Although there is no specific heading of ``methods'' in some abstracts, M2 is subdivided into several steps which could be clearly recognized. They could be labelled ``design''; ``setting'', ``participants'', `` interventions'', `` main outcome measures", and so on.

When it comes to M1, sections labelled with either ``objectives" or ``background'' or both are included. In some structured abstracts, both ``background'' and ``"objectives"'' are labelled, in some only ``objectives", and in others only ``background". Those sections with the label of either ``objectives'' or ``background'' are usually comprised of both of them in terms of actual contents. Moreover, in practice, objectives and background are closely related and usually viewed as a whole to provide a specific communicative purpose of creating research space.

For a structured abstract, four moves could be recognized by headings, one of the lexical signals. For unstructured abstracts in which there are no headings or labels, manual recognition of four moves is needed. Other lexical signals could be helpful in recognizing the moves, which are illustrated in \hyperref[chapter3]{Chapter 3} to \hyperref[chapter6]{Chapter 6}.

\begin{sample}[label={myautocounter}]{\heiti}

\textbf{(M1)}

\textbf{BACKGROUND }

it is still equivocal whether there is a potential role of late-life physical activity in ameliorating the challenges of increasing healthcare expenditure due to the consequence of global population ageing.

\textbf{OBJECTIVE} 

this study aimed to examine the prospective association between physical activity and subsequent hospital care utilisation in older adults and to explore the optimal dose of physical activity required to reduce hospital care utilisation.

\textbf{(M2)}

\textbf{DESIGN} 

this was a prospective cohort study based on the data from the Taiwan 2005 National Health Interview Survey, which were linked to the 2005-12 claims data from the National Health Insurance system.

\textbf{PARTICIPANTS}

1,760 older adults aged 65 or more.

\textbf{METHODS} 

the frequency, duration and intensity for physical activity were assessed, and total physical activity energy expenditure was estimated. The average annualised hospital care utilisation for the period 2006 through 2012, including number of hospitalisations, number of days in hospital and the costs of hospitalisation, were calculated.

\textbf{(M3)}

\textbf{RESULTS }

older adults engaging in at least moderate volume of physical activity ($\geqslant 1,000$ kcal/week) experienced fewer subsequent hospital admissions and fewer days in hospital than did sedentary individuals, after adjusting for covariates. Trends for reduced hospitalisation costs were also found. These associations persisted in sensitivity analyses, including tests of reverse causation.

\textbf{(M4)}

\textbf{CONCLUSION}

this study has provided evidence that older adults who are at least moderately active may minimise utilisation of hospital care services. The findings highlight the importance of maintaining a physically active lifestyle in later life.

\begin{flushright}
  ---Prospective association between late-life physical activity and hospital care utilisation: a 7-year nationwide follow-up study. \emph{Age and Aging (2017)}
\end{flushright}

\end{sample}

\begin{sample}[label={myautocounter}]{\heiti}

\textbf{(M1)}

\textbf{PURPOSE }

Stuttering can trigger anxiety and other psychological and emotional reactions, and limit participation in society. It is possible that psychological counseling could enhance stuttering treatment outcomes; however, little is known about how clients view such counseling. The purpose of this study was to gain an understanding of clients' experiences with, and perceptions of, a psychological counseling service that was offered as an optional adjunct to speech therapy for stuttering.

\textbf{(M2)}

\textbf{METHOD }

Nine individuals who stutter (13--38 years old) participated in semi-structured interviews. Six participants had taken part in psychological counseling; three participants did not do so. Interview data were analyzed using grounded theory as a guiding framework.

\textbf{(M3)}

\textbf{RESULTS }

Four thematic clusters emerged from participants' accounts: insights into personal decision-making, why others may not participate in counseling, psychological counseling as a worthwhile part of therapy, and counseling as a necessary component in a stuttering treatment program.

\textbf{(M4)}

\textbf{CONCLUSION}

In addition to experiencing barriers and facilitators to help-seeking that are reported in related fields, participants accounts also revealed novel facilitators (i.e., a `why not' mentality and the importance of having a pre-existing relationship with the clinician who offered the service) and barriers (i. e., viewing the service as a `limited resource,' and, the overwhelming nature of intensive stuttering treatment programs). Findings suggest that clients value the option to access psychological counseling with trained mental health professionals to support the stuttering treatment provided by speech-language pathologists. Participants made recommendations for the integration of psychological counseling into stuttering treatment programs.

\begin{flushright}
  ---Psychological counseling as an adjunct to stuttering treatment: Clients' experiences and perceptions. 
  
  \emph{Journal of Fluency Disorder (2017)}
\end{flushright}


\end{sample}


\begin{sample}[label={myautocounter}]{\heiti}
\textbf{(M1)}

Genetic and neuroimaging research has identified neurobiological correlates of obesity. However, evidence for an integrated model of genetic risk and brain structural alterations in the pathophysiology of obesity is still absent.

\textbf{(M2)}

Here we investigated the relationship between polygenic risk for obesity, gray matter structure and body mass index (BMI) by the use of univariate and multivariate analyses in two large, independent cohorts ($n=330$ and $n=347$).

\textbf{(M3)}

Higher BMI and higher polygenic risk for obesity were significantly associated with medial prefrontal gray matter decrease, and prefrontal gray matter was further shown to significantly mediate the effect of polygenic risk for obesity on BMI in both samples.

\textbf{(M4)}

Building on this, the successful individualized prediction of BMI by means of multivariate pattern classification algorithms trained on whole-brain imaging data and external validations in the second cohort points to potential clinical applications of this imaging trait marker.

\begin{flushright}
  ---Prefrontal gray matter volume mediates genetic risks for obesity. \emph{Molecular Psychiatry (2017)}
\end{flushright}

\end{sample}

\section{Step Identification}

In the genre with an obvious hierarchical structure, moves are usually composed of the steps or sub-moves, which are the subordinate units. In medical RA abstracts, some steps are conventional steps and others are optional steps. The type and frequency of the steps also show the rhetorical purpose of the author. Lexical signals could be helpful in recognizing the steps, which are illustrated in \hyperref[chapter3]{Chapter 3} to \hyperref[chapter6]{Chapter 6}.

The abstracts of different journals in the corpus are randomly extracted, with 6 articles in each sub-discipline, totaling 108 articles. Through manual recognition, the steps and communication functions that constitute each step are established (Table 1). If the step is used in more than 80\% of articles, it is considered conventional, otherwise optional.

{\small
\begin{longtblr}[
    caption = {Move/Step Scheme of Medical RA Abstracts},
    label = {tab:Move/Step Scheme of Medical RA Abstracts},
]{
    width = \textwidth,
    colspec = {X[1,l,h]  X[1,c,h]  X[3,l,h] X[1,c,h]},
    rowhead = 1, rowfoot = 0, % 每个分页里表头表尾的数量
    % row{odd} = {blue8}, 
    % row{even} = {azure9},
    row{2-6} = {azure9},
    row{7-14} = {white},
    row{15-17} = {azure9},
    row{18-22} = {white},
}
    
\toprule
\textbf{Move/Step} & \textbf{Move/Step Abbr.} & \textbf{Communicative functions} & \textbf{Percentage}\\
\midrule
Move1 & M1 & Creating a research territory/space &  \\
\hspace*{1ex}Move1Step1 & 1S1 & Presenting current knowledge or relevant information established by previous studies & 56.5\% \\
\hspace*{1ex}Move1Step2 & 1S2 & Establishing a niche/problem\sidenotemark[1] & 41.7\% \\
\hspace*{1ex}Move1Step3a & 1S3a & Indicating main purposes & 83.3\% \\
\hspace*{1ex}Move1Step3b & 1S3b & Raising hypotheses & 2.8\% \\
Move2 & M2 & Describing research process &  \\
\hspace*{1ex}Move2Step1 & 2S1 & Reporting on medical ethics review & 2.8\% \\
\hspace*{1ex}Move2Step2 & 2S2 & Explaining briefly research design & 54.6\% \\
\hspace*{1ex}Move2Step3 & 2S3 & Describing subjects or data and their selection criteria & 92.6\% \\
\hspace*{1ex}Move2Step4 & 2S4 & Describing experimemal procedure, such as interventions, examinations, etc. & 60.2\% \\
\hspace*{1ex}Move2Step5 & 2S5 & Describing main outcomes and their measures & 93.5\% \\
\hspace*{1ex}Move2Step6 & 2S6 & Describing data analysis methods & 10.2\% \\
\hspace*{1ex}Move2Step7 & 2S7 & Reporting on registration information & 9.3\% \\
Move3 & M3 & Summarizing results &  \\
\hspace*{1ex}Move3Step1 & 3S1 & Providing information on valid samples & 43.5\% \\
\hspace*{1ex}Move3Step2 & 3S2 & Illustrating overall observation or main results & 100\% \\
Move4 & M4 & Drawing conclusions &  \\
\hspace*{1ex}Move4Step1 & 4S1 & Reiterating pivotal results & 16.7\% \\
\hspace*{1ex}Move4Step2 & 4S2 & Indicating limitations & 3.7\% \\
\hspace*{1ex}Move4Step3 & 4S3 & Stating the significance of the results & 99.1\% \\
\hspace*{1ex}Move4Step4 & 4S4 & Predicting future studies & 19.4\% \\

\bottomrule

\end{longtblr}
}

\sidenotetext[1]{{\small niche 原本指法国天主教徒房屋墙壁上预留用于放置圣母玛利亚的神龛。20世纪80年代,该词以“利基市场 (Niche Market) ”被美国商学家引入市场营销领域,意指那些被市场中的统治者/有绝对优势的企业忽略的某些细分市场。英文论文写作中,niche引申为被主流研究忽视的问题,即现有研究的缺陷、不足,指向该论文意图解决的问题。}}

\begin{sample}[label={myautocounter}]{\heiti}
  \textbf{BACKGROUND}

  \textbf{(1S1)} Graded exercise therapy is an effective and safe treatment for chronic fatigue syndrome, \textbf{(1S2)} but it is therapist intensive and availability is limited. \textbf{(1S3a)} We aimed to test the efficacy and safety of graded exercise delivered as guided self-help.
  
  \textbf{METHODS}

  \textbf{(2S2)} In this pragmatic randomised controlled trial, \textbf{(2S3)} we recruited adult patients (18 years and older) who met the UK National Institute for Health and Care Excellence criteria for chronic fatigue syndrome from two secondary-care clinics in the UK. \textbf{(2S4)} Patients were randomly assigned to receive specialist medical care (SMC) alone (control group) or SMC with additional guided graded exercise self-help (GES). Block randomisation (randomly varying block sizes) was done at the level of the individual with a computer-generated sequence and was stratified by centre, depression score, and severity of physical disability. Patients and physiotherapists were necessarily unmasked from intervention assignment; the statistician was masked from intervention assignment. SMC was delivered by specialist doctors but was not standardised; GES consisted of a self-help booklet describing a six-step graded exercise programme that would take roughly 12 weeks to complete, and up to four guidance sessions with a physiotherapist over 8 weeks (maximum 90 min in total). \textbf{(2S5)} Primary outcomes were fatigue (measured by the Chalder Fatigue Questionnaire) and physical function (assessed by the Short Form-36 physical function subscale); both were self-rated by patients at 12 weeks after randomisation and analysed in all randomised patients with outcome data at follow-up (ie, by modified intention to treat). We recorded adverse events, including serious adverse reactions to trial interventions.\textbf{ (2S6) }We used multiple linear regression analysis to compare SMC with GES, adjusting for baseline and stratification factors.\textbf{ (2S7) }This trial is registered at ISRCTN, number ISRCTN22975026.
  
  \textbf{FINDINGS}

  \textbf{(3S1)} Between May 15,2012, and Dec 24,2014, we recruited 211 eligible patients, of whom 107 were assigned to the GES group and 104 to the control group. \textbf{(3S2) }At 12 weeks, compared with the control group, mean fatigue score was 19.1 (SD-7.6) in the GES group and
  22.9(6.9) in the control group (adjusted difference-4.2 points, 95\% CI -6.1 to -2.3, $P<0.0 001$; effect size 0.53) and mean physical function score was 55.7(23.3) in the GES groupand 50. 8(25.3) in the control group (adjusted difference 6.3 points, 1.8 to 10.8, P=0.006;
  0.20). No serious adverse reactions were recorded and other safety measures did not differ between the groups, after allowing for missing data.

  \textbf{INTERPRETATION}

  \textbf{(4S2)} GES is a safe intervention that might reduce fatigue and, to a lesser extent, physical disability for patients with chronic fatigue syndrome. \textbf{(4S4)} These findings need confirmation and extension to other health-care settings.
  
  \begin{flushright}
    ---Guided graded exercise self-help plus specialist medical care versus specialist medical care alone for chronic fatigue syndrome (GETSET): a pragmatic randomised controlled trial. 
    
    \emph{The Lancet (2017)}
  \end{flushright}
  
\end{sample}

\section{Glossary}

{\small
\begin{longtblr}[
    caption = {Glossary of Chapter 2},
    label = {tab:Glossary of Chapter 2},
    note{a} = {英文论文中指代当前文献中的差距、问题或缺陷。即现有研究尚未解决的部分。},
]{
    width = \textwidth,
    colspec = {X[1,l,h]  X[1,l,h]  X[3,l,h]},
    rowhead = 1, rowfoot = 0, % 每个分页里表头表尾的数量
    % row{odd} = {blue8}, 
    row{even} = {azure9},
}
    
\toprule
\textbf{WORDS} & \textbf{MEANING} & \textbf{MEANING or EXAMPLE}\\
\midrule

\textbf{conventional}/\textipa{k@n"venS@nl}/ & \emph{adj.} 传统的;常规的 & a conventional method, product, practice, etc. has been used for a long time and is considered the usual type \\
\textbf{ethics}/\textipa{"eTiks}/ & \emph{n.} 伦理标准 & [plural] moral rules or principles of behaviour for deciding what is right and wrong \\
\textbf{intervention}/\textipa{""Int@"venSn}/ & \emph{n.} 干预;介入 & an action or ministration that produces an effect or is intended to alter the course of a pathologic process \\
\textbf{niche}/\textipa{ni:S}/ & \emph{n.} 利基\TblrNote{a};生态位;微环境  & a gap in the previous research \\
\textbf{reiterate}/\textipa{ri:"It@reIt}/ & \emph{v.} 重申  & to say something again, usually in order to emphasize it \\

\bottomrule

\end{longtblr}
}

\begin{problemset}
  \item \textbf{Identify the four moves in the abstracts.}
  
  \textbf{Abstract 1}

  \hspace*{2em}Objective: To evaluate the association between the parameters of 24-hour multichannel intraluminal impedance (MII)-pH monitoring and the symptoms or quality of life (QoL) in laryngopharyngeal reflux (LPR) patients.
  
  \hspace*{2em}Design: Prospective cohort study without controls.
  
  \hspace*{2em}Setting: University teaching hospital.
  
  \hspace*{2em}Methods: Forty-five LPR patients were selected from subjects who underwent 24-hour MII-
  pH monitoring and were diagnosed with LPR from September 2014 to May 2015. Reflux Symptom Index (RSI), Health-related Quality of Life (HRQoL), Short Form 12 (SF-12) Survey questionnaires were surveyed. Spearman's correlation was used to analyse the association between the symptoms or QoL and 24-hour MII-pH monitoring.
  
  \hspace*{2em}Results: Most parameters in 24-hour MII-pH monitoring showed weak or no correlation with RSI, HRQoL and SF-12. Only number of non-acid reflux events that reached the larynxand pharynx (LPR-non-acid) and number of total reflux events that reached the larynx and pharynx (LPR-total) parameters showed strong correlation with heartburn in RSI (R=0.520, $P<0$.001, R=0.478, P=0.001, respectively). Multiple regression analysis showed that there was only one significant regression coefficient between LPR-non-acid and voice/hoarseness portion of HRQoL (b=1.719, P=0.022).
  
  \hspace*{2em}Conclusion: Most parameters of 24-hour MII-pH monitoring did not reflect subjective symptoms or QoL in patients with LPR.
  
  \begin{flushright}
    ---Association between 24-hour combined multichannel intraluminal impedance-pH monitoring and symptoms or quality of life in patients with laryngopharyngeal reflux. \emph{Clinical Otolaryngology (2017)}
  \end{flushright}
  
  \textbf{Abstract 2}
  
  \hspace*{2em}Skeletal muscle mitochondrial oxidative capacity declines with age and negatively affects walking performance, but the mechanism for this association is not fully clear. We tested the hypothesis that impaired oxidative capacity affects muscle performance and, through this mechanism, has a negative effect on walking speed. Muscle mitochondrial oxidative capacity was measured by in vivo phosphorus magnetic resonance spectroscopy as the postexercise phosphocreatine resynthesis rate, Kpc, in 326 participants (154 men), aged 24-97 years (mean 71), in the Baltimore Longitudinal Study of Aging. Muscle strength and quality were determined by knee extension isokinetic strength, and the ratio of knee extension strength to thigh muscle cross-sectional area derived from computed topography, respectively. In multivariate linear regression analyses, kpc, was associated with muscle strength ($\beta=0. 140, P=0.007$) and muscle quality ($\beta=0.127, P=0.022$), independent of age, sex, height, and weight; muscle strength was also a significant independent correlate of walking speed ($P<0. 02$ for all tasks) and in a formal mediation analysis significantly attenuated the association between kpc and three of four walking tasks (18\%-29\% reduction in $\beta$ for kpcr). This is the first demonstration in human adults that mitochondrial function affects muscle strength and that inefficiency in muscle bioenergetics partially accounts for differences in mobility through this mechanism.

  \begin{flushright}
    ---Muscle strength mediates the relationship between mitochondrial energetics and walking performance. \emph{Aging Cell (2017)}
  \end{flushright}

  \textbf{Abstract 3}

  \hspace*{2em}Propacetamol, a water-soluble prodrug form of paracetamol, is hydrolyzed by esterase to generate paracetamol in the blood. Each gram of propacetamol is equal to 0.5 g of paracetamol. It has been reported to cause hypotension in critically ill patients with a fever. We aimed to investigate the hemodynamic effects of propacetamol for the control of fever in patients with diverse severities of illness who were managed in the emergency department (ED). We also aimed to identify clinical factors related to significant hemodynamic alterations in ED patients. This was a retrospective study of 1507 ED patients who received propacetamol. Significant hemodynamic alterations were defined as systolic blood pressure (SBP) <90 mmHg or diastolic blood pressure (DBP) <60 mmHg, or a drop in SBP >30 mmHg, which required treatments with a bolus of fluid or vasopressor administration. Postinfusion SBP and DBP were significantly lower than the preinfusion SBP and DBP. A clinically significant drop in BP occurred in 162 (10.7\%) patients, and interventions were necessary. Among the predictors assessed, congestive heart failure (OR 6.21,95\% CI 2.67--14.45) and chills (OR 3.10,95\% CI 2.04--4.70) were independent factors for a significant hemodynamic change. Administrationof propacetamol can provoke a reduction in BP in ED patients. This reduction was clinically significant for 10\% of infusions. Clinicians should be aware of this potential deleterious effect, especially in patients with congestive heart failure or who experience chills prior to the administration of propacetamol.

  \begin{flushright}
    ---Clinically significant hemodynamic alterations after propacetamol injection in the emergency department: prevalence and risk factors. \emph{Internal and Emergency Medicine (2016)}
  \end{flushright}

  \item \textbf{Reorder the four moves in the abstracts.}
  
  \textbf{Abstract 1}

  \hspace*{2em}Methods: A dual-virus tracing strategy combining retroviral birthdating with rabies virus-
  mediated putative retrograde trans-synaptic tracing was used to identify and compare presynaptic inputs onto adult-born and early-born DGCs in the rat pilocarpine model of mTLE.

  \hspace*{2em}---Move \uline{\hspace*{3em}}
  
  \hspace*{2em}Objective: To understand how monosynaptic inputs onto adult-born dentate granule cells (DGCs) are altered in experimental mesial temporal lobe epilepsy (mTLE) and whether their integration differs from early-born DGCs that are mature at the time of epileptogenesis.

  \hspace*{2em}---Move \uline{\hspace*{3em}}
  
  \hspace*{2em}Interpretation: These data support the presence of substantial hippocampal circuit remodeling after an epileptogenic insult that generates prominent excitatory monosynaptic inputs, both local recurrent and widespread feedback loops, onto DGCs. Both adult-born and early-bom DGCs are targets of new inputs from other DGCs as well as from CA3 and CA1 pyramidal cells after pilocarpine treatment, changes that likely contribute to epileptogenesis in experimental mTLE.
  
  \hspace*{2em}---Move \uline{\hspace*{3em}}

  \hspace*{2em}Results: Our results demonstrate that hilar ectopic DGCs preferentially synapse onto adult-born DGCs after pilocarpine-induced status epilepticus (SE), whereas normotopic DGCs synapse onto both adult-born and early-born DGCs. We also find that parvalbumin- and somatostatin-interneuron inputs are greatly diminished onto early-born DGCs after SE. However, somatostatin-interneuron inputs onto adult-born DGCs are maintained, likely due topreferential sprouting. Intriguingly, CA3 pyramidal cell backprojections that specifically target adult-born DGCs arise in the epileptic brain, whereas axons of interneurons and pyramidal cells in CA1 appear to sprout across the hippocampal fssure to preferentilly synapse onto early-bomDGCs.

  \hspace*{2em}---Move \uline{\hspace*{3em}}

  \begin{flushright}
    ---Rabiestracing of birthdated dentae granule cells in rat temporal lobe epilepsy. \emph{Annals of Neurology(2017)}
  \end{flushright}

  \textbf{Abstract 2}
  
  \hspace*{2em}We identified a mssense Asn396Ser mutation (rs77960347) in the endothelial lipase (LIG) gene, occurring with an allele frequency of 1\% in the general population, which was significantly associated with depressive symptoms (P-value $= 5.2 \times 10--08, \beta=7.2$). Replication in three independent data sets ($N= 3612$) confirmed the association of Asn396Ser (P-value = $7.1 \times 10-03$, $\beta = 2.55$) with depressive symptoms.

  \hspace*{2em}---Move \uline{\hspace*{3em}}

  \hspace*{2em}Despite a substantial genetic component, efforts to identify common genetic variation underlying depression have largely been unsuccessful. In the current study we aimed to identify rare genetic variants that might have large effects on depression in the general population.

  \hspace*{2em}---Move \uline{\hspace*{3em}}

  \hspace*{2em}Using high-coverage exome-sequencing, we studied the exonic variants in 1 265 individuals from the Rotterdam study (RS), who were assessed for depressive symptoms.

  \hspace*{2em}---Move \uline{\hspace*{3em}}

  \hspace*{2em}LIPG is predicted to have enzymatic function in steroid biosynthesis, cholesterol biosynthesis and thyroid hormone metabolic processes. The Asn396Ser variant is predicted tod have a damaging effect on the function of LIPG. Within the discovery population, carriers also showed an increased burden of white matter lesions (P-value=$3.3 \times 10-02$) and a higher risk of Alzheimer's disease (odds ratio=2.01; P-value = 2.8x 10-02) compared with the noncarriers. Together, these findings implicate the Asn396Ser variant of LIPG in the pathogenesis of depressive symptoms in the general population.

  \hspace*{2em}---Move \uline{\hspace*{3em}}

  \begin{flushright}
    ---Exome-sequencing in a large population-based study reveals a rare Asn396Ser variant in the LIPG gene associated with depressive symptoms. \emph{Molecular Psychiatry (2017)}
  \end{flushright}

  \textbf{Abstract 3}

  \hspace*{2em}Conclusion: This study suggests that hearing problems in later life could increase the risk of having difficulties performing IADLs, which include more complex everyday tasks such as shopping and light housework. However, further studies are needed to determine the associations observed including the underlying pathways.
  
  \hspace*{2em}---Move \uline{\hspace*{3em}} 
  
  \hspace*{2em}Methods: Data were collected on self-reported hearing impairment including hearing aid use, and disability assessed as mobility limitations (problems walking/taking stairs), difficulties with activities of daily living (ADL) and instrumental ADL (IADL). Mortality data were obtained from the National Health Service register.
  
  \hspace*{2em}---Move \uline{\hspace*{3em}}
  
  \hspace*{2em}Background and objective: Hearing impairment is common in older adults and has been implicated in the risk of disability and mortality. We examined the association between hearing impairment and risk of incident disability and all-cause mortality.

  \hspace*{2em}---Move \uline{\hspace*{3em}}
  
  \hspace*{2em}Results: Among 3,981 men, 1,074(27\%) reported hearing impairment. Compared with men with no hearing impairment, men who could hear and used a hearing aid, and men who could not hear despite a hearing aid had increased risks of IADL difficulties (age-adjusted OR 1.86, 95\% CI 1.29--2.70; OR 2.74, 95\% CI 1.53--4.93, respectively). The associations remained after further adjustment for covariates including social class, lifestyle factors, comorbidities and social engagement. Associations of hearing impairment with incident mobility limitations, incident ADL difficulties and all-cause mortality were attenuated on adjustment for covariates.
  
  \hspace*{2em}---Move \uline{\hspace*{3em}}
  
  \begin{flushright}
    ---Hearing impairment and incident disability and all-cause mortality in older British community-dwelling men. \emph{Age and Ageing (2016)}
  \end{flushright}
  
\end{problemset}


\chapter{Move One}\label{chapter3}

Move One (M1) is the first part of an abstract. It creates a research territory or space for the research

\section{Steps in M1}

M1 usually involves three steps, each with a clear communicative purpose.

1S1 Presenting current knowledge or relevant information established by previous studies

1S2 Establishing a niche/problem

1S3a Indicating main purposes

1S3b Raising hypotheses

Usually in the last step either the objectives or the hypotheses are presented while only a few abstracts cover both.

\begin{sample}[label={myautocounter}]{\heiti}
  Senescent cells are present in premalignant lesions and sites of tissue damage and accumulate in tissues with age. In vivo identification, quantification and characterization of senescent cells are challenging tasks that limit our understanding of the role of senescent cells in diseases and aging. Here, we present a new way to precisely quantify and identify senescent cells in tissues on a single-cell basis.

  \begin{flushright}
    ---Quantitative identification of senescent cells in aging and disease. \emph{Aging Cell (2017)}
  \end{flushright}

  \tcblower

  \noindent \textbf{STEP IDENTIFICATION}

  \vspace*{10pt}
  {\small\noindent
  \begin{tblr}{colspec={X[1,c]X[5,l]},cell{even}{1,2} = {azure9}}
    \toprule
    \textbf{Step} & \textbf{Sample} \\ 
    \midrule
    
    1S1 & Senescent cells are present in premalignant lesions and sites of tissue damage and accumulate in tissues with age.  \\
    1S2 & In vivo identification, quantification and characterization of senescent cells are challenging tasks that limit our understanding of the role of senescent cells in diseases and aging. \\
    1S3a & Here, we present a new way to precisely quantify and identify senescent cells in tissues on a single-cell basis. \\
      
    \bottomrule
  \end{tblr}
  }

  \noindent \textbf{ANALYSIS} 
  
  This is a typical example of M1 involving three steps. The first sentence informs the readers of the background information related to the research. The second sentence, where the word ``no'' can be seen as a signal for 1S2, identifies the problem-the lack of a standard. The last sentence, with the subject ``aim'' and the to-infinitive, shows the research objective.
\end{sample}



\begin{sample}[label={myautocounter}]{\heiti}

  \textbf{OBJECTIVE} 
  
  Seizures are more frequent in patients with Alzheimer's disease (AD) and can hasten cognitive decline. However, the incidence of subclinical epileptiform activity in AD and its consequences are unknown. Motivated by results from animal studies, we hypothesized higherthan expected rates of subclinical epileptiform activity in AD with deleterious effects on cognition.

  
  \begin{flushright}
    ---Incidence and impact of subclinical epileptiform activity in Alzheimer's disease. \emph{Annals of Neurology (2016)}
  \end{flushright}

  \tcblower

  \noindent \textbf{STEP IDENTIFICATION}

  \vspace*{10pt}
  {\small\noindent
  \begin{tblr}{colspec={X[1,c]X[5,l]},cell{even}{1,2} = {azure9}}
    \toprule
    \textbf{Step} & \textbf{Sample} \\ 
    \midrule
    
    1S1 & Seizures are more frequent in patients with Alzheimer's disease (AD) and can hasten cognitive decline. \\
    1S2 & However, the incidence of subclinical epileptiform activity in AD and its consequences are unknown. \\
    1S3b & Motivated by results from animal studies, we hypothesized higherthan expected rates of subclinical epileptiform activity in AD with deleterious effects on cognition. \\
      
    \bottomrule
  \end{tblr}
  }
  
\end{sample}

1S3a is a comparatively conventional step in this move, while others are optional. These steps follow the orders listed above to fulfill the authors' communicative purposes, which are to clarify what is already known, what remains unknown and what is to be known.

\begin{sample}[label={myautocounter}]{\heiti}

  \textbf{BACKGROUND} 
  
  CT-P6 is a proposed biosimilar to reference trastuzumab. In this study, we aimed to establish equivalence of CT-P6 to reference trastuzumab in neoadjuvant treatment of HER2-positive early-stage breast cancer.

  
  \begin{flushright}
    ---CT-P6 compared with reference trastuzumab for HER2-positive breast cancer: a randomised, double-blind, active-controlled, phase 3 equivalence trial. \emph{Lancet Oncology (2017)}
  \end{flushright}

  \tcblower

  \noindent \textbf{STEP IDENTIFICATION}

  \vspace*{10pt}
  {\small\noindent
  \begin{tblr}{colspec={X[1,c]X[5,l]},cell{even}{1,2} = {azure9}}
    \toprule
    \textbf{Step} & \textbf{Sample} \\ 
    \midrule
    
    1S1 & BACKGROUND: CT-P6 is a proposed biosimilar to reference trastuzumab. \\
    1S3a & In this study, we aimed to establish equivalence of CT-P6 to reference trastuzumab in neoadjuvant treatment of HER2-positive early-stage breast cancer. \\
      
    \bottomrule
  \end{tblr}
  }

\end{sample}

\begin{sample}[label={myautocounter}]{\heiti}

  At a population level, dietary consumption of fish rich in docosahexaenoic acid (DHA) is associated with prevention of cognitive decline but this association is not clear in carriers of the apolipoprotein E epsilon 4 allele (E4). Plasma and liver DHA concentrations show significant alterations in EA carriers, in part corrected by DHA supplementation. However, whether DHA sufficiency in E4 carriers has consequences on cognition is unknown.

  
  \begin{flushright}
    ---Docosahexaenoic acid prevents cognitive deficits in human apolipoprotein E epsilon 4-targeted replacement mice. \emph{Neurobiology of Aging (2017)}
  \end{flushright}

  \tcblower

  \noindent \textbf{STEP IDENTIFICATION}

  \vspace*{10pt}
  {\small\noindent
  \begin{tblr}{colspec={X[1,c]X[5,l]},cell{even}{1,2} = {azure9}}
    \toprule
    \textbf{Step} & \textbf{Sample} \\ 
    \midrule
    
    1S1 & At a population level, dietary consumption of fish rich in docosahexaenoic acid (DHA) is associated with prevention of cognitive decline but this association is not clear in carriers of the apolipoprotein E epsilon 4 allele (E4). Plasma and liver DHA concentrations show significant alterations in EA carriers, in part corrected by DHA supplementation. \\
    1S2 & However, whether DHA sufficiency in E4 carriers has consequences on cognition is unknown. \\
      
    \bottomrule
  \end{tblr}
  }

\end{sample}

\begin{sample}[label={myautocounter}]{\heiti}

  The aim of this cohort study is to compare the symptom burden of patients who have an planned admission. unplanned admission to an acute palliative care unit (APCU) with patients who have a regular planned admission.

  
  \begin{flushright}
    ---Characteristics of patients with an unplanned admission to an acute palliative care unit. \emph{Internal and Emergency Medicine (2017)}
  \end{flushright}

  \tcblower

  \noindent \textbf{STEP IDENTIFICATION}

  \vspace*{10pt}
  {\small\noindent
  \begin{tblr}{colspec={X[1,c]X[5,l]},cell{even}{1,2} = {azure9}}
    \toprule
    \textbf{Step} & \textbf{Sample} \\ 
    \midrule
    
    1S3a & The aim of this cohort study is to compare the symptom burden of patients who have an planned admission. unplanned admission to an acute palliative care unit (APCU) with patients who have a regular planned admission. \\
      
    \bottomrule
  \end{tblr}
  }

\end{sample}

It needs to be mentioned, however, that these steps can also be presented with different orders in a small number of cases.

\begin{sample}[label={myautocounter}]{\heiti}
  
  \textbf{OBJECTIVE}

  To compare the rate of positive resection margins between radioactive seed localization (RSL) and wire-guided localization (WGL) after breast conserving surgery (BCS).
 
  \textbf{BACKGROUND} 
  
  WGL is the current standard for localization of nonpalpable breast lesions in BCS, but there are several difficulties related to the method.

  
  \begin{flushright}
    ---Radioactive Seed Localization or Wire-guided Localization of Nonpalpable Invasive and InSitu Breast Cancer: A Randomized, Multicenter, Open-label Trial. \emph{Annals of Surgery (2017)}
  \end{flushright}

  \tcblower

  \noindent \textbf{STEP IDENTIFICATION}

  \vspace*{10pt}
  {\small\noindent
  \begin{tblr}{colspec={X[1,c]X[5,l]},cell{even}{1,2} = {azure9}}
    \toprule
    \textbf{Step} & \textbf{Sample} \\ 
    \midrule
    
    1S3a & OBJECTIVE: To compare the rate of positive resection margins between radioactive seed localization (RSL) and wire-guided localization (WGL) after breast conserving surgery (BCS). \\
    1S1 & BACKGROUND: WGL is the current standard for localization of nonpalpable breast lesions in BCS, \\
    1S2 & but there are several difficulties related to the method. \\
      
    \bottomrule
  \end{tblr}
  }

\end{sample}

\begin{sample}[label={myautocounter}]{\heiti}

  \textbf{OBJECTIVE} 
  
  The aim of this study was to investigate the efficacy of intraperitoneal local anesthetic (IPLA) on pain after acute laparoscopic appendectomy in children.

  \textbf{SUMMARY OF BACKGROUND} 
  
  IPLA reduces pain in adult elective surgery. It has not been well studied in acute peritoneal inflammatory conditions. We hypothesized that IPLA would improve recovery in pediatric acute laparoscopic appendectomy.

  
  \begin{flushright}
    ---Intraperitoneal Local Anesthetic for Laparoscopic Appendectomy in Children: A Randomized Controlled Trial. \emph{Annals of Surgery (2017)}
  \end{flushright}

  \tcblower

  \noindent \textbf{STEP IDENTIFICATION}

  \vspace*{10pt}
  {\small\noindent
  \begin{tblr}{colspec={X[1,c]X[5,l]},cell{even}{1,2} = {azure9}}
    \toprule
    \textbf{Step} & \textbf{Sample} \\ 
    \midrule
    
    1S3a & \textbf{OBJECTIVE}: The aim of this study was to investigate the efficacy of intraperitoneal local anesthetic (IPLA) on pain after acute laparoscopic appendectomy in children. \\
    1S1 & \textbf{SUMMARY OF BACKGROUND}: IPLA reduces pain in adult elective surgery.  \\
    1S2 & It has not been well studied in acute peritoneal inflammatory conditions. \\
    1S3b & We hypothesized that IPLA would improve recovery in pediatric acute laparoscopic appendectomy. \\
      
    \bottomrule
  \end{tblr}
  }

\end{sample}

\section{Language Features in Each Step}

  \subsection{Step 1(1S1) Presenting current knowledge or relevant information established by previous studies}

    \subsubsection{Step Analysis}

    With this step authors introduce naturally the research having been conducted. In this step, current knowledge or relevant information established by previous studies is presented or explained, which might include the pertinent mechanism or definition, the possibility of a phenomenon, the significance of a certain study, etc.
    
    There are two ways that this step could be put forward. Authors could present the knowledge or information in the same scope (E.g.~\ref{eg:1s1-1}). Also, the authors could introduce the conclusions of relevant previous studies (E.g.~\ref{eg:1s1-2}). In the latter one authors always employ words or phrases, such as “preclinical studies", to indicate the source of the information (E.g.~\ref{eg:1s1-2})
    
    \begin{eg}[label={eg:1s1-1}]{}
      Topical immunomodulators(Tl)-including corticosteroids, calcineurin inhibitors, and vitamin D analogues-are commonly prescribed in multiple specialties,$\dots$
    \end{eg}

    \begin{eg}[label={eg:1s1-2}]{}
      \uline{Preclinical studies} have found radiotherapy enhances antitumour immune responses.
    \end{eg}

    \subsubsection{Language Realizations}

    Simple present tense is often used in this step to describe the current understanding of the topic.

    \begin{eg}[label={myautocounter}]{}
      In chronic hemodialysis, physical functioning (PF) \uline{is known} to be poor.
    \end{eg}

    While ``previous studies'' or other phrases with similar meaning are used as the subject of the sentence, present perfect tense is preferred. However, simple present tense is still used to refer to the current knowledge or relevant information established by previous studies.

    \begin{eg}[label={myautocounter}]{}
      Previous studies have shown that more active older adults have better cognition and brain health based on a variety of structural neuroimaging measures.
    \end{eg}

    \begin{task}[label={myautocounter}]{\heiti Corpus-based task}
      What kind of verb is often used in present perfect tense in M1?
    \end{task}

    Both active voice and passive voice could be used at this step.

    \begin{eg}[label={myautocounter}]{}
      Physicians \uline{are} often \uline{asked} to prognosticate soon after a patient presents with stroke.
    \end{eg}

    \begin{eg}[label={myautocounter}]{}
      Youth baseball frequently \uline{results} in repetitive strain injuries.
    \end{eg}

    Modal auxiliary verbs, such as ``may'', ``can'', as well as other words indicating likelihood, such as ``possible'', are commonly used in this step.

    \begin{eg}[label={myautocounter}]{}
      Urban design \uline{may} affect children's habitual physical activity by influencing active commuting and neighborhood play.
    \end{eg}

    \begin{eg}[label={myautocounter}]{}
      Hirsutism in females \uline{can} be a source of considerable psychological distress and a threat to female identity.
    \end{eg}

    \begin{eg}[label={myautocounter}]{}
      It is \uline{possible} that psychological counseling could enhance stuttering treatment outcomes.
    \end{eg}

    \subsubsection{Lexical Chunks}

    \begin{enumerate}
      \item has been associated with
      \begin{eg}[label={myautocounter}]{}
        Ethnicity \uline{has been associated with} clinical and experimental pain responses.
      \end{eg}

      \item has been shown to
      \begin{eg}[label={myautocounter}]{}
        Estrogen administration following menopause \uline{has been shown to} support hippocampally mediated cognitive processes.
      \end{eg}
    \end{enumerate}
  
  \subsection{Step 2(1S2) Establishing a niche/problem}

    \subsubsection{Step Analysis}

    In this step, negative evaluations are often given to the information provided at Step 1, including the limitations or the defects of the previous studies, which leaves a gap to be filled, a problem to be addressed or an idea to be tested. In other words, this step implies the value and significance of the authors' research.

    \begin{eg}{}
      However, to date, underlying neuronal mechanisms of these WM load-dependent activation changes in aging remain \uline{poorly understood}.
    \end{eg}

    \subsubsection{Language Realizations}

    Simple present tense is usually seen in this step. But when the word ``study'' serves as the subject of an active sentence, or in a passive sentence as the unexpressed agent, present perfect tense is more often employed.

    \begin{eg}{}
      Despite the availability of objective tests, gastroesophageal reflux disease (GERD) diagnosis and management in infants \uline{remains} controversial and highly variable.
    \end{eg}

    \begin{eg}{}
      However, no study \uline{has} directly \uline{compared} these outcomes between sports.
    \end{eg}

    Words such as ``however'' and ``but'' could be used to lead the readers from Step 1 to Step2, placing an emphasis on the problem or gap to be pointed out.

    \begin{eg}{}
      Delirium is associated with adverse postoperative outcomes, \uline{but} controversy exists regarding whether delirium is an independent predictor of mortality.
    \end{eg}

    \begin{eg}{}
      Excellent outcomes have been reported for anterior cruciate ligament (ACL) reconstruction (ACLR) in professional athletes in a number of different sports. \uline{However}, no study has directly compared these outcomes between sports.
    \end{eg}

    Besides the two words mentioned above, other words indicating a contrast or a negative meaning, such as ``not'' or ``unknown'', can also be seen as a signal for 1S2.

    \begin{eg}{}
      Validated models to predict risk for complications are \uline{not} available, and the effect of treatment on risk is \uline{unknown}.
    \end{eg}

    \subsubsection{Lexical Chunks}

    \begin{enumerate}
      \item little is known about the
      \begin{eg}{}
        However, \uline{little is known about the} effect of pharmacological PHD inhibition on tumor expansion, and on liver regeneration after surgical resection.
      \end{eg}
    \end{enumerate}

  \subsection{Step 3(1S3a) Indicating main purposes}

    \subsubsection{Step Analysis}

    Step 1 and Step 2 show what is already known about the study and what remains to be done. Step 3a then states the purpose or expectations of the research being presented. It outlines the reasons why the research is conducted.

    \subsubsection{Language Realizations}

    The to-infinitive is found in most cases to express the aims in this step. It can be placed either at the beginning of a sentence or in the middle, as can be seen in the following examples.

    \begin{eg}{}
      \uline{To achieve} the efficient usage of contrast material (CM) in high-pitch CT aortography, an appropriate duration of the CM injection is crucial.
    \end{eg}

    \begin{eg}{}
      The purpose of this study was \uline{to evaluate} how travel burden and hospital volume influence treatment and outcomes of patients with locally advanced esophageal cancer.
    \end{eg}

    ``The aim/purpose/objective of this study'' or ``we'' are frequently used as the subject of the sentence. Simple past tense is often seen in this step especially when ``was to'', ``sought to'', ``tested'', etc. serve as the predicates.

    \begin{eg}{}
      \uuline{The objective of this study} \uline{was to} measure the success of intubation of a simulated patient in an upright versus supine position by novice intubators after brief training.
    \end{eg}
    \begin{eg}{}
      \uuline{We} \uline{sought to} determine if time from emergency department (ED) physician evaluation until operative intervention is independently associated with appendiceal perforation (AP) in children.
    \end{eg}
    \begin{eg}{}
      \uuline{We} \uline{tested} whether any association with APOE e4 status on cognitive ability was larger in older ages or in those with cardiometabolic diseases.
    \end{eg}
    
    
    It needs to be mentioned that when ``report'' serves as the predicate, simple present tense would be employed.

    \begin{eg}{}
      We \uline{report} the effect of intravenous alteplase on long-term survival after ischaemic stroke of participants in the Third International Stroke Trial (IST-3).
    \end{eg}

    \subsubsection{Lexical Chunks}

    \begin{enumerate}
      \item the aim of this study was to evaluate the
      \begin{eg}{}
        \uline{The aim of this study was to evaluate the} application of Mindray BC-6800 body fluid (BF) mode in cytometric analysis of CSF compared to light microscopy (LM).
      \end{eg}

      \item the aim of this study was to investigate
      \begin{eg}{}
        \uline{The aim of this study was to investigate} the S100B utility for the determination of concussion in a professional 15-players-a-side rugby team.
      \end{eg}

      \item the aim of this study was to assess
      \begin{eg}{}
        \uline{The aim of this study was to assess} the effect of HDF on serum sST2 and NT-proBNP concentrations in End-stage Renal Disease (ESRD) patients.
      \end{eg}

      \item the purpose/objective of this study was to
      \begin{eg}{}
        \uline{The purpose of this study was to} present our investigation of the utility of a novel flexible robotic system for transoral supraglottic laryngectomy and total laryngectomy.
      \end{eg}

      \item of this study was to determine the
      \begin{eg}{}
        The overall aim \uline{of this study was to determine the} effect of introducing a smartphone pain application (app), for both Android and iPhone devices that enables chronic pain patients to assess, monitor, and communicate their status to their providers.
      \end{eg}

      \item of this study was to investigate/evaluate/compare/examine
      \begin{eg}{}
        The aim \uline{of this study was to evaluate} the diagnostic performance of susceptibility-weighted magnetic resonance imaging (SW-MRI) for the differentiation of osteophytes and disc herniations of the spine compared with that of conventional spine MR sequences and radiography.
      \end{eg}

      \item we aimed to assess the
      \begin{eg}{}
        \uline{We aimed to assess the} efficacy and safety of the MEK inhibitor binimetinib versus that of dacarbazine in patients with advanced NRAS-mutant melanoma.
      \end{eg}

    \end{enumerate}

  \subsection{Step 3(1S3b) Raising hypotheses}
    
    \subsubsection{Step Analysis}

    While most authors set out goals or objectives in M1 to introduce the problems to be solved, a few may choose to put forward a hypothesis to be tested or proved by the research.

    \begin{eg}{}
      We hypothesized that the use of 2-chloroprocaine would be associated with a faster recovery from sensorimotor block.
    \end{eg}

    \subsubsection{Language Realizations}

    In this step, with the verb ``hypothesize'' used as the predicate, the subject could be ``we'' or ``the authors'', the former much more commonly seen than the latter. Simple past tense is more often used than simple present. And the modal verb ``would'' is very often found in the sentence to indicate the probability.

    \begin{eg}{}
      In the current study, \uline{we hypothesized} that this method \uline{would} be equal to or better than the transversus abdominis plane block with regard to pain relief and its duration of action after cesarean delivery.
    \end{eg}

    \begin{eg}{}
      \uline{The authors hypothesized} that the outcomes of meniscal repair associated with concomitant multiligament reconstruction \uline{would} significantly improve from preoperatively to postoperatively at a minimum of 2 years after the index surgery.
    \end{eg}


    \subsubsection{Lexical Chunks}

    \begin{enumerate}
      \item tested the hypothesis that
      \begin{eg}
        We \uline{tested the hypothesis that} extrafascial placement of the catheter tip reduces the rate of hemidiaphragmatic paresis compared with intrafascial tip placement for CISB while providing effective analgesia.
      \end{eg}
    \end{enumerate}

\section{Sample Reading}

Three samples of M1 are presented here, followed by a detailed analysis on the language features of the steps included.

\begin{sample}[label={myautocounter}]{\heiti}
  
  \textbf{BACKGROUND} 
  
  Head impacts and resulting head accelerations cause concussive injuries. There is no standard for reporting head impact data in sports to enable comparison between studies.

  \textbf{OBJECTIVE }
  
  The aim was to outline methods for reporting head impact acceleration data in sport and the effect of the acceleration thresholds on the number of impacts reported.


  \begin{flushright}
    ---The Influence of Head Impact Threshold for Reporting Data in Contact and Collision Sports: Systematic Review and Original Data Analysis. \emph{Sports Medicine (2015)}
  \end{flushright}

  \tcblower

  \noindent \textbf{STEP IDENTIFICATION}

  \vspace*{10pt}
  {\small\noindent
  \begin{tblr}{colspec={X[1,c]X[5,l]},cell{even}{1,2} = {azure9}}
    \toprule
    \textbf{Step} & \textbf{Sample} \\ 
    \midrule
    
    1S1 & \textbf{BACKGROUND}: Head impacts and resulting head accelerations cause concussive injuries. \\
    1S2 & There is no standard for reporting head impact data in sports to enable comparison between studies. \\
    1S3a & \textbf{OBJECTIVE }: The aim was to outline methods for reporting head impact acceleration data in sport and the effect of the acceleration thresholds on the number of impacts reported. \\
      
    \bottomrule
  \end{tblr}
  }

  \noindent \textbf{ANALYSIS}

  This is a typical example of M1 involving three steps. The first sentence informs the readers of the background information related to the research. The second sentence, where the word ``no'' can be seen as a signal for 1S2, identifies the problem---the lack of a standard. The last sentence, with the subject ``aim'' and the to-infinitive, shows the research objective.

\end{sample}

\begin{sample}[label={myautocounter}]{\heiti}
  \textbf{BACKGROUND} 
  
  Ergometrine is a uterotonic agent that is recommended in the prevention and management of postpartum hemorrhage. Despite its long-standing use, the mechanism by which it acts in humans has never been elucidated fully. The objective of this study was to investigate the role of adrenoreceptors in ergometrine's mechanism of action in human myometrium. The study examined the hypothesis that a-adrenoreceptor antagonism would result in the reversal of the uterotonic effects of ergometrine.

  \begin{flushright}
    ---A Role for Adrenergic Receptors in the Uterotonic Effects of Ergometrine in Isolated Human Term Nonlaboring Myometrium. \emph{Anesthesia and-Analgesia (2017)}
  \end{flushright}

  \tcblower

  \noindent \textbf{STEP IDENTIFICATION}

  \vspace*{10pt}
  {\small\noindent
  \begin{tblr}{colspec={X[1,c]X[5,l]},cell{even}{1,2} = {azure9}}
    \toprule
    \textbf{Step} & \textbf{Sample} \\ 
    \midrule
    
    1S1  & BACKGROUND: Ergometrine is a uterotonic agent that is recommended in the prevention and management of postpartum hemorrhage.  \\
    1S2  & Despite its long-standing use, the mechanism by which it acts in humans has never been elucidated fully. \\
    1S3a  & The objective of this study was to investigate the role of adrenoreceptors in ergometrine's mechanism of action in human myometrium. \\
    1S3b  & The study examined the hypothesis that a-adrenoreceptor antagonism would result in the reversal of the uterotonic effects of ergometrine. \\
      
    \bottomrule
  \end{tblr}
  }

  \noindent \textbf{ANALYSIS} 

  This paragraph involves four sentences. The first one provides the readers with the necessary knowledge of Ergometrine, a drug used to promote the contractions of the muscle ofthe womb (uterus). The second sentence, with the word "never"serving as the signal for 1S2, points out the gap that needs to be filled -the mechanism has not "been elucidated fully". Here, "elucidate"means to make clear. Present perfect tense is used, because the omitted agent in this passive sentence is "the previous study". The third sentence introduces the objective and the last one raises a hypothesis. 
  
  What needs to be noticed is that 1S3a and 1S3b are not very often seen with each other. This sample is one of the few exceptions.

\end{sample}

\begin{sample}[label={myautocounter}]{\heiti}
  
  \textbf{OBJECTIVE} 
  
  To investigate the prognostic significance of p16 in patients with hypopharyngeal squamous cell carcinoma (HPSCC) and to evaluate the relationship between p16 and human papillomavirus (HPV). Unlike in oropharyngeal SCC (OPSCC), the prognostic significance of p16 in HPSCC and its association with HPV is unclear.


  \begin{flushright}
    ---p16 not a prognostic marker for hypopharyngeal squamous cell carcinoma.
    
    \emph{Archives of Otorhinolaryngology-Head \& Neck Surgery (2012)}
  \end{flushright}

  \tcblower

  \noindent \textbf{STEP IDENTIFICATION}

  \vspace*{10pt}
  {\small\noindent
  \begin{tblr}{colspec={X[1,c]X[5,l]},cell{even}{1,2} = {azure9}}
    \toprule
    \textbf{Step} & \textbf{Sample} \\ 
    \midrule
    
    1S3a & \textbf{OBJECTIVE}: To investigate the prognostic significance of p16 in patients with hypopharyngeal squamous cell carcinoma (HPSCC) and to evaluate the relationship between p16 and human papillomavirus (HPV). \\
    1S2 & Unlike in oropharyngeal SCC (OPSCC), the prognostic significance of p16 in HPSCC and its association with HPV is unclear. \\
      
    \bottomrule
  \end{tblr}
  }

  \noindent \textbf{ANALYSIS} 
  
  This sample presents the steps with orders different from Sample 1 and 2. It starts with to-infinitives showing the objectives, and proceeds with the research gap, "unclear"being a signal.

\end{sample}

\section{Glossary}

{\small
\begin{longtblr}[
    caption = {Glossary of Chapter 3},
    label = {tab:Glossary of Chapter 3},
    % note{a} = {英文论文中指代当前文献中的差距、问题或缺陷。即现有研究尚未解决的部分。},
]{
    width = \textwidth,
    colspec = {X[1,l,h]  X[1,l,h]  X[3,l,h]},
    rowhead = 1, rowfoot = 0, % 每个分页里表头表尾的数量
    % row{odd} = {blue8}, 
    row{even} = {azure9},
}
    
\toprule
\textbf{WORDS} & \textbf{MEANING} & \textbf{MEANING or EXAMPLE}\\
\midrule

\textbf{chunk}/\textipa{tS2Nk}/ & \emph{n.} 组块;区块;数据块 & a considerable amount \\
{\textbf{mechanism} \\ /\textipa{"mek@""nIz@m}/} & \emph{n.} 机制;机械装置;方法 & a special way of getting something done within a particular system \\
\textbf{modal auxiliary verb} & \emph{n.} 情态动词 & A modal auxiliary verb, often simply called a modal verb or even just a modal, is used to change the meaning of other verbs (commonly known as main verbs) by expressing modality-that is, asserting (or denying) possibility, likelihood, ability, permission, obligation, or future intention. \\
{\textbf{pertinent} \\ /\textipa{"p3:rtIn@nt}/} & \emph{adj.} 有关的;恰当的;相宜的 & relevant to a particular subject \\
{\textbf{predicate} \\ /\textipa{"predIk@t}/} & \emph{n.} 谓语 & a part of a sentence containing a verb that makes a statement about the subject of the verb, such as ``went home''in ``John went home. d'' \\
{\textbf{territory} \\ /\textipa{"ter@t0:ri}/} & \emph{n.} 领域;地域; & refer to an area of knowledge or experience \\

\bottomrule

\end{longtblr}
}

\begin{problemset}
  \item \textbf{Find out how many steps are involved in the following examples.}
  1S1 Presenting current knowledge or relevant information established by previous studies

  1S2 Establishing a niche/problem

  1S3a Indicating main purposes

  1S3b Raising hypotheses

  \begin{enumerate}
    \item At a population level, dietary consumption of fish rich in docosahexaenoic acid (DHA) is associated with prevention of cognitive decline but this association is not clear in carriers of the apolipoprotein E epsilon 4 allele (E4). Plasma and liver DHA concentrations show significant alterations in E4 carriers, in part corrected by DHA supplementation. However, whether DHA sufficiency in E4 carriers has consequences on cognition is unknown.
    \item In chronic hemodialysis, physical functioning (PF) is known to be poor. We set out to assess to what extent chronic dialysis patients are able to maintain a good physical condition over time and what the influence of age is on the trajectory of PF.
    \item Both patient characteristics and intraoperative factors have been associated with a higher risk of stroke after cardiac surgery. We hypothesized that poor systemic oxygenation in the perioperative period is associated with increased risk of stroke following cardiopulmonary bypass.
    \item Therapies to extend healthspan are poised to move from laboratory animal models to human clinical trials. Translation from mouse to human will entail challenges, among them the multifactorial heterogeneity of human aging. To inform clinical trials about this heterogeneity, we report how humans' pace of biological aging relates to personal-history characteristics.
  \end{enumerate}

  \item \textbf{Rearrange the sentences in correct order.}

  \begin{enumerate}
    \item \uline{\hspace*{3em}}
    \begin{itemize}
      \item[A] Regular use of sunbed exposure has been reported to increase 25-hydroxyvitamin-D3 [\ce{25(OH)D}] serum levels.
      \item[B] We investigated the impact of standard sunbed use compliant with the European Union standard on \ce{25(OH)D} serum modulation and well-being.
      \item[C] However, the influence of sunbeds compliant with the recent European Union standard EN-60335-2-27 on \ce{25(OH)D} serum levels is unknown.
    \end{itemize}

    \item \uline{\hspace*{3em}}
    \begin{itemize}
      \item[A] The Indian Health Service provides health care to eligible American Indians and Alaskan Natives.
      \item[B] We seek to determine the characteristics and capabilities of Indian Health Service emergency departments (EDs).
      \item[C] No published data exist on emergency services offered by this unique health care system.
    \end{itemize}

    \item \uline{\hspace*{3em}}
    \begin{itemize}
      \item[A] This trial compared immediate posttreatment effects of family-focused treatment for childhood depression (FFT-CD) with those of individual supportive psychotherapy (IP) for children 7 to 14 years old with depressive disorders 
      \item[B] Integrating family in treatment could have particularly salutary effects during this developmental period.
      \item[C] Despite the morbidity and negative outcomes associated with early-onset depression, few studies have examined the efficacy of psychosocial treatment for depressive disorders during childhood.
    \end{itemize}

    \item \uline{\hspace*{3em}}
    \begin{itemize}
      \item[A] We tested the hypothesis that aquatic treadmill exercise would augment CBF and lower HR compared with land-based treadmill exercise.
      \item[B] However, their effect on cerebral blood flow (CBF) responses has not been examined.
      \item[C] Aquatic treadmills are used as a rehabilitation method for conditions such as spinal cord injury, osteoarthritis, and stroke, and can facilitate an earlier return to exercise training for athletes.
    \end{itemize}
  \end{enumerate}

  \item \textbf{Fill in the blanks with appropriate words or the correct form of the verbs given}

  \begin{enumerate}
    \item This single-centre, randomized trial \uline{\hspace*{3em}} (test) the hypothesis that the administration of FFP after CPB (late FFP group) is superior to FFP priming (early FFP group)
    in terms of postoperative bleeding and overall red blood cell (RBC) transfusion.
    \item We assessed whether addition of the antiepileptic drug levetiracetam to the benzodiazepine clonazepam \uline{\hspace*{3em}}(improve) prehospital treatment of GCSE.
    \item \uline{\hspace*{3em}} (investigate) the prevalence, injury rate, severity, nature, and economic burden of RRIs in Dutch trailrunners.
    \item We \uline{\hspace*{3em}} (aim) to assess the safety and activity of an anti-PD-1 antibody pembrolizumab in patients with locally advanced or metastatic urothelial cancer.
    \item We \uline{\hspace*{3em}} (report) 2-year overall survival data from a randomised controlled trial assessing this treatment in previously untreated advanced melanoma.
    \item Previous studies \uline{\hspace*{3em}} (propose) that faster speed improves movement quality.
    \item We \uline{\hspace*{3em}} (seek) to determine the compliance rate within the pigmented lesions clinic at our academic institution and identify demographic variables that may influence adherence.
    \item Anxiety disorders constitute a major disease and social burden worldwide; \uline{\hspace*{3em}}, many questions concerning the underlying molecular mechanisms still remain open.
    \item Oncologic outcomes for induction chemotherapy and its role in patients with advanced olfactory neuroblastoma (ONB) \uline{\hspace*{3em}} unclear.
    \item Histaminergic neurons are crucial to maintain wakefulness, but their role in cataplexy is \uline{\hspace*{3em}}.
    \item Background and Objectives: Although many studies \uline{\hspace*{3em}} (find) no difference between thoracic epidural block and unilateral thoracic paravertebral block after thoracotomy, no previous studies \uline{\hspace*{3em}} (compare) epidural block with bilateral thoracic paravertebral block (bTPVB) in patients undergoing open liver resection. We \uline{\hspace*{3em}} (aim) investigate whether there was a significant analgesic advantage of thoracic epidural over bTPVB after liver resection.
    \item Background: Chronic tendinopathy \uline{\hspace*{3em}} (be) a commonly occurring clinical problem that affects both athletes and inactive middle-aged patients. Although some studies \uline{\hspace*{3em}} (show) that different platelet-rich plasma (PRP) preparations could exert various therapeutic effects in vitro, the role of leukocytes in PRP \uline{\hspace*{3em}} (be) defined under tendinopathy conditions in vivo.
    
    \hspace*{2em}Purpose: This study \uline{\hspace*{3em}} (compare) the effects of the intratendon delivery of leukocyte-poor PRP (Lp-PRP) versus leukocyte-rich PRP (Lr-PRP) in a rabbit chronic tendinopathy model in vivo.
    \item Background: Eye trackers \uline{\hspace*{3em}} (be) widely used among people with amyotrophic lateral sclerosis, and their benefits to quality of life \uline{\hspace*{3em}} (show). On the contrary, Brain-computer interfaces (BCIs) \uline{\hspace*{3em}} (be) still quite a novel technology, which also serves as an access technology for people with severe motor impairment.
    
    \hspace*{2em}Objective: \uline{\hspace*{3em}} (compare) a visual P300-based BCI and an eye tracker in terms of information transfer rate (ITR), usability, and cognitive workload in users with motor impairments. 
  \end{enumerate}

  \item \textbf{Discussion: Read the following examples and discuss with your partners about the reasons why modal auxiliary verbs and words indicating likelihood are used?}

  \textbf{Example 1}

  \hspace*{2em}In the treatment of anxiety disorders, attention bias modification therapy (ABMT) and cognitive-behavioral therapy (CBT) may have complementary effects by targeting different aspects of perturbed threat responses and behaviors. ABMT may target rapid, implicit threat reactions, whereas CBT may target slowly deployed threat responses. The authors used amygdala-based connectivity during a threat-attention task and a randomized controlled trialdesign to evaluate potential complementary features of these treatments in pediatric anxiety disorders.

  \begin{flushright}
    ---Complementary Features of Attention Bias Modification Therapy and Cognitive-Behavioral Therapy in Pediatric Anxiety Disorders. \emph{The American Journal of Psychiatry (2017)}
  \end{flushright}

  \textbf{Example 2}

  \hspace*{2em}Ticagrelor is an effective antiplatelet therapy for patients with coronary atherosclerotic disease and might be more effective than aspirin in preventing recurrent stroke and cardiovascular events in patients with acute cerebral ischaemia of atherosclerotic origin. Our aim was to test for a treatment-by-ipsilateral atherosclerotic stenosis interaction in a subgroup analysis of patients in the Acute Stroke or Transient Ischaemic Attack Treated with Aspirin or Ticagrelor and Patient Outcomes (SOCRATES) trial.

  \begin{flushright}
    ---Efficacy and safety of ticagrelor versus aspirin in acute stroke or transient ischaemic attack of atherosclerotic origin: a subgroup analysis of SOCRATES, a randomised, double-blind, controlled trial. \emph{The Lancet. Neurology (2017)}
  \end{flushright}

  \textbf{Example 3}

  \hspace*{2em}Psychosocial disorders have been reported in adults who stutter, especially social anxiety disorder. Social anxiety has been linked to childhood victimization. It is possible that recalled childhood victimization could be linked to psychosocial problems reported in some adults who stutter.

  \begin{flushright}
    ---Long-term Consequences of Childhood Bullying in Adults Who Stutter: Social Anxiety, Fear of Negative Evaluation, Self-esteem, and Satisfaction with Life. \emph{Journal of Fluency Disorders (2016)}
  \end{flushright}


\end{problemset}


\chapter{Move Two}\label{chapter4}

Move Two (M2) describes the research process. In this move, the author states the way the problem has been studied. This might include the subjects and the methodology followed, among other things. In a 4-or 5-paragraphed abstract, M2 is usually a single paragraph; in abstracts with more than 5 paragraphs, M2 is divided into several paragraphs with a heading in front of each, which is clearer and more specific.

\section{Steps in M2}

Whichever form is M2 in, it can be further divided into several steps. Depending on different research types, the steps in M2 may include:

2S1 Reporting on medical ethics review2S2 Explaining briefly research design

2S3 Describing subjects or data and their selection criteria

2S4 Describing experimental procedure, such as interventions, examinations, etc.

2S5 Describing main outcomes and their measures

2S6 Describing data analysis methods

2S7 Reporting on registration information 

In most cases these steps follow the order listed above to fulfill the authors' communicative purposes (Sample \ref{sam:M2_1} \& \ref{sam:M2_2}). Sometimes some steps may be embedded with others (Sample \ref{sam:M2_1} \& \ref{sam:M2_3}). For instance, 2S2 could be embedded with 2S3, 2S4, and 2S1; 2S3 could be embedded in 2S2, 2S4, and 2S5, and so on. The order of the steps may differ from the above-mentioned mode. In most abstracts, M2 includes 2S2, 2S3, 2S4 and 2S5. Some of the steps are optional in an abstract, especially 2S1, 2S6 and 2S7 (Sample \ref{sam:M2_1} \& \ref{sam:M2_2}).

\begin{sample}[label={sam:M2_1}]{\heiti}

  \textbf{METHOD} 
  
  Randomized controlled trial for adolescents (12--18 years of age) with recent (past 3 months) suicide attempts or other self-harm. Youth were randomized either to SAFETY or to treatment as usual enhanced by parent education and support accessing community treatment (E-TAU). Outcomes were evaluated at baseline, 3 months, or end of treatment period, and were followed up through 6 to 12 months. The primary outcome was youth-reported incident suicide attempts through the 3-month follow-up.
  
  \begin{flushright}
    ---Cognitive-Behavioral Family Treatment for Suicide Attempt Prevention: A Randomized Controlled Trial. \emph{Journal of the American Academy of Child and Adolescent Psychiatry (2017)}
  \end{flushright}

  \tcblower

  \noindent \textbf{STEP IDENTIFICATION}

  \vspace*{10pt}
  {\small\noindent
  \begin{tblr}{colspec={X[1,c]X[5,l]},cell{even}{1,2} = {azure9}}
    \toprule
    \textbf{Step} & \textbf{Sample} \\ 
    \midrule
    
    2S3 with 2S2 embedded & Randomized controlled trial for adolescents (12-18 years of age) with recent (past 3 months) suicideattempts or other self-harm. \\
    2S4 & Youth were randomized either to SAFETY or to treatment as usual enhanced by parent education and support accessing community treatment (E-TAU). \\
    2S5 & Outcomes were evaluated at baseline, 3 months, or end of treatment period, and were followed up through 6 to 12 months. The primary outcome was youth-reported incident suicide attempts through the 3-month follow-up. \\

    \bottomrule
  \end{tblr}
  }

\end{sample}

\begin{sample}[label={sam:M2_2}]{\heiti}
  
  \textbf{DESIGN} 
  
  Cross-sectional observational study.

  \textbf{PARTICIPANTS} 
  
  One hundred nineteen children 2 to 16 years of age (mean age, 9.4 years; standard deviation [SD], 4.56 years) with glaucoma and their parents.

  \textbf{METHODS} 
  
  Completion of 3 validated instruments for children to assess (1) functional visual ability (FVA) with the Cardiff Visual Ability Questionnaire for Children (CVAQC), (2) VR QoL with the Impact of Vision Impairment for Children (IVI-C), and (3) HR QoL with the Pediatric Quality of Life Inventory (PedsQL) version 4.0.

  \textbf{MAIN OUTCOME MEASURES} 
  
  Cardiff Visual Ability Questionnaire for Children, IVI-C, and PedsQL scores.

  \begin{flushright}
    ---Quality of Life and Functional Vision in Children with Glaucoma. \emph{Ophthalmology (2017)}
  \end{flushright}

  \tcblower

  \noindent \textbf{STEP IDENTIFICATION}

  \vspace*{10pt}
  {\small\noindent
  \begin{tblr}{colspec={X[1,c]X[5,l]},cell{even}{1,2} = {azure9}}
    \toprule
    \textbf{Step} & \textbf{Sample} \\ 
    \midrule
    
    2S2 & DESIGN: Cross-sectional observational study. \\
     2S3 & PARTICIPANTS: One hundred nineteen children 2 to 16 years of age (mean age, 9.4 years; standard deviation [SD], 4.56 years) with glaucoma and their parents. \\
     2S4 & METHODS: Completion of 3 validated instruments for children to assess (1) functional visual ability (FVA) with the Cardiff Visual Ability Questionnaire for Children (CVAQC), (2) VR QoL with the Impact of Vision Impairment for Children (IVI-C), and (3) HR QoL with the Pediatric Quality of Life Inventory (PedsQL) version 4.0. \\
     2S5 & MAIN OUTCOME MEASURES: Cardiff Visual Ability Questionnaire for Children, IVI-C, and PedsQL scores. \\
     
    \bottomrule
  \end{tblr}
  }

\end{sample}

\begin{sample}[label={sam:M2_3}]{\heiti}
  
  \textbf{MATERIALS AND METHODS} 
  
  The institutional review board approved this retrospective study and waived the informed consent requirement. Seventy-four patients with surgically confirmed PNETs and 82 patients with PDACs who underwent gadobutrol-enhanced MR imaging were included. Two radiologists independently evaluated the morphologic characteristics and temporal enhancement patterns of each tumor. Quantitative analysis, including measurement of tumor size, maximal upstream parenchymal thickness (MUPT), contrast-to-noise ratio, and apparent diffusion coefficient values, was performed. Uni-and multivariate logistic regression analyses were performed to identify relevant features to differentiate between PNETs and PDACs.

  \begin{flushright}
    ---Nonhypervascular Pancreatic Neuroendocrine Tumors: Differential Diagnosis from Pancreatic Ductal Adenocarcinomas at MR Imaging-Retrospective Cross-sectional Study. \emph{Radiology (2017)}
  \end{flushright}

  \tcblower

  \noindent \textbf{STEP IDENTIFICATION}

  \vspace*{10pt}
  {\small\noindent
  \begin{tblr}{colspec={X[1,c]X[5,l]},cell{even}{1,2} = {azure9}}
    \toprule
    \textbf{Step} & \textbf{Sample} \\ 
    \midrule
    
     2S1 with 2S2 embedded & The institutional review board approved this retrospective study and waived the informed consent requirement. \\
     2S3 & Seventy-four patients with surgically confirmed PNETs and 82 patients with PDACs who underwent gadobutrol-enhanced MR imaging were included. \\
     2S4 & Two radiologists independently evaluated the morphologic characteristics and temporal enhancement patterns of each tumor. \\
     2S5 & Quantitative analysis, including measurement of tumor size, maximal upstream parenchymal thickness (MUPT), contrast-to-noise ratio, and apparent diffusion coefficient values, was performed. \\
     2S6 & Uni-and multivariate logistic regression analyses were performed to identify relevant features to differentiate between PNETs and PDACs. \\

    \bottomrule
  \end{tblr}
  }

\end{sample}

\section{Language Features in Each Step}

Most steps in M2 are in the past tense, which suggests that the study or research has been conducted. Both active voice and passive voice can be employed in this move, with "we"as the subject in the active voice (E.g.~\ref{eg:M2-1}) and other nouns indicating the research process as the subject in the passive voice (E.g.~\ref{eg:M2-2})

\begin{eg}[label={eg:M2-1}]{}
  \uuline{We} \uline{conducted} a multicentre, double-blind, randomised, placebo-controlled trial at three hospitals in Australia.
\end{eg}

\begin{eg}[label={eg:M2-2}]{}
  A correlation \uuline{analysis} \uline{was used} to examine the relationship between age and weight gain.
\end{eg}

  \subsection{Step 1(2S1) Reporting on medical ethics review}

    \subsubsection{Step Analysis}

    Medical ethics review has become a necessity in most medical research with human beings or animals as the tested. Despite its growing importance it is seldom reported in medical abstracts.

    \subsubsection{Language Realizations}

    Past tense and passive voice are usually employed in this step, with ``study'' or ``approval'' etc. as the subject. The following is one of the very few 2S1 examples found. Notice the past tense.
    
    \begin{eg}
      This \uuline{study} \uline{was} approved by our institutional review board, and informed consent \uline{was} waived due to its retrospective design.
    \end{eg}

    \subsubsection{Lexical Chunks}

    \begin{enumerate}
      \item study was approved by the
      \begin{eg}
        This prospective \uline{study was approved by the} institutional ethics committee, and written informed consent was obtained from all participants.
      \end{eg}
    \end{enumerate}

  \subsection{Step 2(2S2) Explaining briefly research design}

    \subsubsection{Step Analysis}

    A research design here narrowly refers to the study type. In most abstracts, this optional step usually appears at the beginning of M2.

    \subsubsection{Language Realizations}

    2S2 is usually in simple sentences or noun phrases as a single paragraph (E.g.~\ref{eg:2s2-1}) or embedded with other steps, usually with 2S3 (E.g.~\ref{eg:2s2-2}), occasionally with 2S4 or 2S1. Descriptive words are often employed to modify ``study'', ``trial'', etc. The words ``cohort'', ``retrospective'', ``prospective'', ``observational'', ``cross-sectional'' tend to collocate with ``study''. The words ``randomised'', ``controlled'', ``double-blind'' tend to collocate with ``trial''(E.g.~\ref{eg:2s2-1} \& \ref{eg:2s2-2}).

    \begin{eg}[label={eg:2s2-1}]{}
      DESIGN: \uline{Retrospective} \uuline{study}.
    \end{eg}

    \begin{eg}[label={eg:2s2-2}]{}
      In this \uline{randomised, regimen-controlled, double-blind, phase 2} trial \textbf{(2S2)}, we enrolled adult patients with multiple basal-cell carcinomas, including those with basal-cell nevus syndrome, who had one or more histopathologically confirmed and at least six clinically evident basal-cell carcinomas \textbf{(2S3)}.
    \end{eg}

    \subsubsection{Lexical Chunks}

    \begin{enumerate}
      \item a retrospective cohort study (of)
      \begin{eg}{}
        This was \uline{a retrospective cohort study of} singleton pregnancies delivered between 24 0/7 and 39 6/7 weeks, using 2005 through 2006 US national linked birth and death certificate data
      \end{eg}
      
      \begin{eg}{}
        \uline{A retrospective cohort study} was conducted in 2 distinct cohorts of female members of Kaiser Permanente Southern California, which is a large integrated healthcare delivery system.
      \end{eg}

      \item we conducted a retrospective
      \begin{eg}{}
        We conducted a retrospective collaborative study involving centers from 11 countries and 11 US institutions analyzing 102 ASNs by IMS.
      \end{eg}

      \item a cross-sectional study
      \begin{eg}{}
        This was \uline{a cross-sectional study of} 83 patients enrolled in the Morphea in Adults and Children cohort.
      \end{eg}

      \item a secondary analysis of
      \begin{eg}{}
        This was \uline{a secondary analysis of} a prospective cohort study conducted at eight Canadian hospitals.
      \end{eg}
    \end{enumerate}

    \begin{task}{\heiti Corpus-based task}
      What verb tense could be used when a single sentence is used to describe the studyd design?
    \end{task}

  \subsection{Step 3(2S3) Describing subjects or data and their selection criteria}

    \subsubsection{Step Analysis}

    2S3 is a step presenting the subjects or participants, including the location of the study, the subjects' characteristics, number, and selection criteria, etc. For most abstracts, 2S3 is conventional.

    \subsubsection{Language Realizations}

    This step can be an individual paragraph, in noun phrases (E.g. \ref{eg:2s3-1}) or sentences. When it is in sentences, it could be embedded with the other steps, usually with 2S2 (E.g. \ref{eg:2s3-2}) or 2S4 (E.g. \ref{eg:2s3-3}) in sentences. If the participants or selection criteria are used as the sentence subject, passive voice is preferred over active voice (E.g. \ref{eg:2s3-4}). Simple past tense is always used.

    \begin{eg}[label={eg:2s3-1}]{}
      PARTICIPANTS: A total of 152 patients (152 eyes) with DME.
    \end{eg}

    \begin{eg}[label={eg:2s3-2}]{}
      This was a retrospective, multicenter cross-sectional analysis \textbf{(2S2)} of children (<19 years old) presenting to 16 pediatric EDs (2004--2008) \textbf{(2S3)}.
    \end{eg}

    \begin{eg}[label={eg:2s3-3}]{}
      From November 1994 through January 2002, we randomly assigned 731 men with localized prostate cancer \textbf{(2S3)} to radical prostatectomy or observation \textbf{(2S4)}.
    \end{eg}

    \begin{eg}[label={eg:2s3-4}]{}
      All Danish \uuline{patients} $\geqslant 18$ years on January 1,2012 with AD diagnosed by a hospital dermatologist \uline{were included}. \uuline{Patients} \uline{were age-and sex-matched} in a l : 4 ratio with general population controls.
    \end{eg}

    \subsubsection{Lexical Chunks}

    \begin{enumerate}
      \item were included in the/this
      \begin{eg}{}
        All patients who received at least one dose of nivolumab \uline{were included in the} primary and safety analyses.
      \end{eg}
        
      \begin{eg}{}
        Patients diagnosed from 1995 to 2014 \uline{were included in this} study.
      \end{eg}

      \item years or older with
      \begin{eg}{}
        Eligible patients were aged 18 \uline{years or older with} histologically or cytologically confirmed recurrent stage Illb or stage IV, chemotherapy-naive NSCLC.
      \end{eg}

      \item data were collected from
      \begin{eg}{}
        \uline{Data were collected from} the China Health and Nutrition Survey, a prospective open cohort and an ongoing nationwide health and nutrition survey, consisting of 3 199 apparently healthy Chinese girls aged 6 to 18 years at entry from 1991 to 2011.
      \end{eg}
    \end{enumerate}

  \subsection{Step 4(2S4) Describing experimental procedure, such as interventions, examinations, etc.}

    \subsubsection{Step Analysis}

    Experimental procedure here may include interventions, examinations, and so on. 2S4 is usually included in an abstract when the research is experiment-based or when there is an intervention in the research. It is a conventional step in M2.

    \subsubsection{Language Realizations}

    Passive voice is preferred over active voice in this step, as the research subjects or methods are often used as the subject of the sentence. Simple past tense is always used. Time indicators are characteristic of this step.

    \begin{eg}{}
      The asthma \uuline{intervention} \uline{was tailored} to the participant's allergen sensitivity and exposure, and it comprised 4 visits \uline{over the course of 1 year}.
    \end{eg}

    \subsubsection{Lexical Chunks}

    \begin{enumerate}
      \item (patients/participants) were randomly assigned (to) (receive)
      \begin{eg}{}
        From April 3,2014, to Dec 4,2015,667 \uline{patients were randomly assigned to receive} placebo (n=286), erenumab 70 mg (n=191), or erenumab 140 mg (n=190).
      \end{eg}

      \begin{eg}{}
        Dogs \uline{were randomly assigned to receive} intravenous OA to induce ALI (n=7 for each OA group) or saline as an OA control (n=6 for each control).
      \end{eg}

      \begin{eg}{}
        \uline{Patients were randomly assigned to} a choice of 100 or 200 ug ITM or no choice.
      \end{eg}

      \begin{eg}{}
        To induce PAH, Sprague-Dawley rats \uline{were randomly assigned to} treatment with monocrotaline or normal saline.
      \end{eg}

      \begin{eg}{}
        \uline{Participants were randomly assigned} (1: 1) centrally by an interactive voice response system, to receive either ipilimumab 10 mg/kg or placebo every 3 weeks for four doses, then every 3 months for up to 3 years.
      \end{eg}
      
      \item (patients) were randomized to (receive)
      \begin{eg}{}
        Eyes \uline{were randomized to receive} intravitreal injection of bevacizumab (1. 25 mg; n= 182) or aflibercept (2.0 mg; n=180) every 4 weeks through month 6.
      \end{eg}
      \begin{eg}{}
        \uline{Patients were randomized to} nasally inhaled isopropyl alcohol versus nasallyd inhaled normal saline solution.
      \end{eg}

      \item (were) masked to treatment (allocation/assignment)
      \begin{eg}{}
        Patients, study investigators, and study sponsor personnel \uline{were masked to treatment assignment}.
      \end{eg}
      \begin{eg}{}
        Patients were not \uline{masked to treatment allocation}.
      \end{eg}
      \begin{eg}{}
        The study was open label and no-one was \uline{masked to treatment assignment}.
      \end{eg}

      \item we randomly assigned patients
      \begin{eg}{}
        \uline{We randomly assigned patients} with advanced heart failure to receive either the new centrifugal continuous-flow pump or a commercially available axial continuous-flow pump.
      \end{eg}
    \end{enumerate}

  \subsection{Step 5(2S5) Describing main outcomes and their measures}

    \subsubsection{Step Analysis}

    The World Health Organization defines an outcome measure as a change in the health of an individual, group of people, or population that is attributable to an intervention or series of interventions. Outcome measures are chosen to assess the impact of the interventions. In a clinical trial, outcome measures may include mortality, cure, clinical worsening, readmission, etc. Sometimes their measurements might also be included in this step.

    \subsubsection{Language Realizations}

    2S5 can be presented in noun phrases (E.g.~\ref{eg:2s5-1}) or in sentences usually of active voice with ``outcomes'', ``outcome measure'', etc. as the sentence subject (E.g.~\ref{eg:2s5-2}). Simple past tense is always used.

    \begin{eg}[label={eg:2s5-1}]{}
      MAIN OUTCOME MEASURES: Cardiff Visual Ability Questionnaire for Children, IVI-C, and PedsQL scores.
    \end{eg}

    \begin{eg}[label={eg:2s5-2}]{}
      \uuline{The primary outcome} \uline{was} change from baseline to week 6 in the amount of urine leakage, measured by the 1-hour pad test. \uuline{Secondary outcomes} \uline{included} mean 72-hour urinary incontinence episodes measured by a 72-hour bladder diary (72-hour incontinence episodes).
    \end{eg}

    \subsubsection{Lexical Chunks}

    \begin{enumerate}
      \item the primary end point was (the)
      \begin{eg}{}
        \uline{The primary end point was the} duration of treatment for symptoms of neonatal opioid withdrawal.        
      \end{eg}

      \begin{eg}{}
        \uline{The primary end point was} a composite score of anaesthetists' non-technical skills (ANTS) assessed by two blinded evaluators.   
      \end{eg}
      
      \item the primary endpoint was (the)
      \begin{eg}{}
        \uline{The primary endpoint was the} rate of uncomplicated perineal wound healing defined as a Southampton wound score of less than 2 at 30 days postoperatively.   
      \end{eg}

      \begin{eg}{}
        \uline{The primary endpoint was} percentage reduction from baseline in the number of clinically evident basal-cell carcinomas at week 73.   
      \end{eg}

      \item (the primary) outcome (measure) was (the)
      \begin{eg}{}
        \uline{The primary outcome was} the mean difference in the angle of horizontal and vertical deviations after dilation in prism diopters.   
      \end{eg}
      \begin{eg}{}
        \uline{The primary outcome was} expired tidal volume.   
      \end{eg}
      \begin{eg}{}
        \uline{The primary outcome measure was} a three-level outcome-survival without neurodevelopmental impairment, survival with neurodevelopmental impairment, or death. 
      \end{eg}
      \begin{eg}{}
        The secondary \uline{outcome measure was the} impact of postoperative prophylaxis on donor tissue-associated infections.   
      \end{eg}

      \item was the proportion of patients
      \begin{eg}{}
        The primary endpoint, which has been reported previously, \uline{was the proportion of patients} with BRAFV600 wild-type melanoma achieving an investigator-assessed objective response.
      \end{eg}

      \item in all patients who
      \begin{eg}{}
        We assessed safety \uline{in all patients who} received at least one dose of study drug. 
      \end{eg}

      \item secondary end points included
      \begin{eg}{}
        \uline{Secondary end points included} overall survival, objective response rate, duration of response, effects on disease-related symptoms, safety, and tolerability.
      \end{eg}
    \end{enumerate}

  \subsection{Step 6(2S6) Describing data analysis methods}

    \subsubsection{Step Analysis}

    2S6 is included when some specific data extraction and analysis are needed in the abstract. It is usually optional.

    \subsubsection{Language Realizations}

    Sentences in this step are usually presented in passive voice with the data extraction and analysis methods as the sentence subject. Simple past tense is always used.

    \begin{eg}{}
      \uuline{Multivariable logistic regression analysis} \uline{was used} to estimate the risks of these complications among obese pregnancies compared with normal-weight pregnancies.    
    \end{eg}

    \subsubsection{Lexical Chunks}

    \begin{enumerate}
      \item logistic regression was used to
      \begin{eg}{}
        Multivariable logistic regression was used to investigate the association of the MDS score and AMD, taking account of potential confounders and the multicenter study design.   
      \end{eg}
      \item models were used to
      \begin{eg}{}
        Multivariable Cox and logistic regression models were used to examine associations between vaccination history and screening initiation and interval adherence. 
      \end{eg}
      \item regression analysis was used
      \begin{eg}{}
        Logistic regression analysis was used to estimate factors predicting IUBT failure.   
      \end{eg}
    \end{enumerate}

  \subsection{Step 7(2S7) Reporting on registration information}

    \subsubsection{Step Analysis}

    According to the WHO rules, the clinical trials, especially those prospective ones, should be registered, so some clinical trials will include this step at the end of M2. In spite of its importance, it is seldom seen in abstract writing.

    \subsubsection{Language Realizations}

    The study or trial is usually taken as the subject in this step, so passive voice is used more often. Generally speaking, simple present tense is used much more frequently than simple past tense. Following are two of the few 2S7 examples found.

    \begin{eg}{}
      This trial is registered with ClinicalTrials. gov, number NCTO1815840, and the study is ongoing.
    \end{eg}

    \begin{eg}{}
      The KEYNOTE-001 trial was registered with ClinicalTrials. gov, number NCTOI295827.  
    \end{eg}

    \subsubsection{Lexical Chunks}

    \begin{enumerate}
      \item This trial/study is registered with
      \begin{eg}{}
        This trial is registered with ClinicalTrials. gov as NCTO0569127.
      \end{eg}

      \begin{eg}{}
        This study is registered with Current Controlled Trials, number ISRCTN82857232.
      \end{eg}
    \end{enumerate}

    \begin{task}{Corpus-based task}
      Which word can collocate with ``clinical'', ``experiment'', ``research'', ``study'' or ``trial'' in your corpus? List them in frequency order.
    \end{task}

\section{Sample Reading}

Three samples of M2 are presented here, followed by a detailed analysis on the language features of the steps included.

\begin{sample}[label={myautocounter}]{\heiti}
  
  \textbf{METHOD} 
  
  Randomized controlled trial for adolescents (12--18 years of age) with recent (past 3 months) suicide attempts or other self-harm. Youth were randomized either to SAFETY or to treatment as usual enhanced by parent education and support accessing community treatment (E-TAU). Outcomes were evaluated at baseline, 3 months, or end of treatment period, and were followed up through 6 to 12 months. The primary outcome was youth-reported incident suicide attempts through the 3-month follow-up.

  \begin{flushright}
    ---Cognitive-Behavioral Family Treatment for Suicide Attempt Prevention: A Randomized Controlled Trial. \emph{Journal of the American Academy of Child and Adolescent Psychiatry (2017)}
  \end{flushright}

  \tcblower

  \noindent \textbf{STEP IDENTIFICATION}

  \vspace*{10pt}
  {\small\noindent
  \begin{tblr}{colspec={X[1,c]X[5,l]},cell{even}{1,2} = {azure9}}
    \toprule
    \textbf{Step} & \textbf{Sample} \\ 
    \midrule
    
    2S3 with 2S2 embedded & Randomized controlled trial for adolescents (12--18 years of age) with recent (past 3 months) suicide attempts or other self-harm. \\
    2S4 & Youth were randomized either to SAFETY or to treatment as usual enhanced by parent education and support accessing community treatment (E-TAU). \\
    2S5 & Outcomes were evaluated at baseline, 3 months, or end of treatment period, and were followed up through 6 to 12 months. The primary outcome was youth-reported incident suicide attempts through the 3-month follow-up. \\
      
    \bottomrule
  \end{tblr}
  }

  \noindent \textbf{ANALYSIS}

  This is a typical one-paragraphed M2. The first language structure is a noun phrase with the research design (2S2) ``randomized controlled trial'' embedded with the description of the participants and selection criteria (2S3) ``adolescents$\dots$''. The process description (2S4) and the outcome measurement (2S5) are in passive voice with ``youth'', ``outcomes'' and ``the primary outcome'' as the sentence subjects. 2S1, 2S6 and 2S7 are missing, as they are not conventional in M2. Common lexical chunks of M2 are also seen in this paragraph, such as ``randomized controlled trial'', ``$\dots$ were randomized to $\dots$'', ``Outcomes were evaluatedd $\dots$'', ``The primary outcome was $\dots$'', etc.

\end{sample}

\begin{sample}[label={myautocounter}]{\heiti}
  
  \textbf{MATERIALS AND METHODS} 
  
  The institutional review board approved this retrospective study and waived the informed consent requirement. Seventy-four patients with surgically confirmed PNETs and 82 patients with PDACs who underwent gadobutrol-enhanced MR imaging were included. Two radiologists independently evaluated the morphologic characteristics and temporal enhancement patterns of each tumor. Quantitative analysis, including measurement of tumor size, maximal upstream parenchymal thickness (MUPT), contrast-to-noise ratio, and apparent diffusion coefficient values, was performed. Uni-and multivariate logistic regression analyses were performed to identify relevant features to differentiate between PNETs and PDACs.

  \begin{flushright}
    ---Nonhypervascular Pancreatic Neuroendocrine Tumors: Differential Diagnosis from Pancreatic Ductal Adenocarcinomas at MR Imaging-Retrospective Cross-sectional Study. \emph{Radiology (2017)}
  \end{flushright}

  \tcblower

  \noindent \textbf{STEP IDENTIFICATION}

  \vspace*{10pt}
  {\small\noindent
  \begin{tblr}{colspec={X[1,c]X[5,l]},cell{even}{1,2} = {azure9}}
    \toprule
    \textbf{Step} & \textbf{Sample} \\ 
    \midrule
    
    2S1 with 2S2 embedded  & The institutional review board approved this retrospective study and waived the informed consent requirement.\\
    2S3  & Seventy-four patients with surgically confirmed PNETs and 82 patients with PDACs who underwent gadobutrol-enhanced MR imaging were included.\\
    2S4  & Two radiologists independently evaluated the morphologic characteristics and temporal enhancement patterns of each tumor.\\
    2S5  & Quantitative analysis, including measurement of tumor size, maximal upstream parenchymal thickness (MUPT), contrast-to-noise ratio, and apparent diffusion coefficient values, was performed. \\
    2S6  & Uni-and multivariate logistic regression analyses were performed to identify relevant features to differentiate between PNETs and PDACs. \\
      
    \bottomrule
  \end{tblr}
  }

  \noindent \textbf{ANALYSIS}

  Although it is in a single one paragraph, this M2 includes almost every step listed above (except for 2S7). The first sentence offers the ethical investigation information(2S1) embeddedwith the research design ``retrospective study''(2S2), followed by the participants and selection criteria in passive voice ``seventy-four patients $\dots$ were included''(2S3) and the research process in active voice ``Two radiologists $\dots$ evaluated $\dots$''(2S4). Several specific measurements ``$\dots$ of tumor size, maximal upstream parenchymal thickness,$\dots$''(2S5) and data analysis ``uni-and multivariate logistic regression analysis $\dots$''(2S6) are mentioned next. The passive voice is quite prominent in this sample, indicating the objectivity of the research process. Signals for M2 covered in this example include ``retrospective study'', ``informed consent'', ``$\dots$ patients $\dots$ were included'', ``analysis, $\dots$ was performed'', ``logistic regression analyses were performed'', etc.

\end{sample}


\begin{sample}[label={myautocounter}]{\heiti}
  
  \textbf{DESIGN} 
  
  Cross-sectional observational study.

  \textbf{PARTICIPANTS }
  
  One hundred nineteen children 2 to 16 years of age (mean age, 9.4 years; standard deviation [SD], 4.56 years) with glaucoma and their parents.

  \textbf{METHODS }
  
  Completion of 3 validated instruments for children to assess (1) functional visual ability (FVA) with the Cardiff Visual Ability Questionnaire for Children (CVAQC), (2) VR QoL with the Impact of Vision Impairment for Children (IVI-C), and (3) HR QoL with the Pediatric Quality of Life Inventory (PedsQL) version 4.0.

  \textbf{MAIN OUTCOME MEASURES} 
  
  Cardiff Visual Ability Questionnaire for Children, IVI-C, and PedsQL scores.

  \begin{flushright}
    ---Quality of Life and Functional Vision in Children with Glaucoma.
\emph{Ophthalmology (2017)}
  \end{flushright}

  \tcblower

  \noindent \textbf{STEP IDENTIFICATION}

  \vspace*{10pt}
  {\small\noindent
  \begin{tblr}{colspec={X[1,c]X[5,l]},cell{even}{1,2} = {azure9}}
    \toprule
    \textbf{Step} & \textbf{Sample} \\ 
    \midrule
    
    2S2 & {DESIGN: \\Cross-sectional observational study.} \\
    2S3 & {PARTICIPANTS: \\One hundred nineteen children 2 to 16 years of age (mean age, 9.4 years; standard deviation [ SD], 4.56 years) with glaucoma and their parents.} \\
    2S4 & {METHODS: \\Completion of 3 validated instruments for children to assess (1) functional visual ability (FVA) with the Cardiff Visual Ability Questionnaire for Children (CVAQC), (2) VR QoL with the Impact of Vision Impairment for Children (IVI-C), and (3) HR QoL with the Pediatric Quality of Life Inventory (PedsQL) version 4.0.} \\
    2S5 & {MAIN OUTCOME MEASURES: \\Cardiff Visual Ability Questionnaire for Children, IVI-C, and PedsQL scores.} \\
      
    \bottomrule
  \end{tblr}
  }


  \noindent \textbf{ANALYSIS}

  This is a typical multi-paragraphed M2. With subtitles, the steps are quite clear. In such a structured M2, noun phrases are very common. In the "DESIGN"step (2S2), some descriptive medical terms ``cross-sectional'' and ``observational'' are used to modify ``study''. In the ``PARTICIPANTS'' step (2S3), number and selection criteria of the participants are presented in noun or prepositional phrases. In the ``METHODS'' step (2S4), the examinations or assessments are described. In the ``MAIN OUTCOME MEASURES'' step (2S5), the tool used to observe is mentioned.

\end{sample}

\section{Glossary}

{\small
\begin{longtblr}[
    caption = {Glossary of Chapter 4},
    label = {tab:Glossary of Chapter 4},
    % note{a} = {英文论文中指代当前文献中的差距、问题或缺陷。即现有研究尚未解决的部分。},
]{
    width = \textwidth,
    colspec = {X[1,l,h]  X[1,l,h]  X[3,l,h]},
    rowhead = 1, rowfoot = 0, % 每个分页里表头表尾的数量
    % row{odd} = {blue8}, 
    row{even} = {azure9},
}
    
\toprule
\textbf{WORDS} & \textbf{MEANING} & \textbf{MEANING or EXAMPLE}\\
\midrule

{\textbf{collocate}\\/\textipa{"k6l@""keIt}/} & \emph{v.} 【语】词语的组合;排列 & to group or place together in some system or order \\
\textbf{embed}/\textipa{Im"beEd}/ & \emph{v.} 嵌入 & to set or fix firmly in a surrounding mass\\
{\textbf{extraction}\\/\textipa{Ik"str\ae kS@n}/} & \emph{n.} 提取;开采;提炼;拔出 & the act or process of removing or obtaining sth from sth else\\
{\textbf{intervention}\\/\textipa{Int@"benS@n}/} & \emph{n.} 干预;介入;调解 & the act of intervening\\
{\textbf{registration}\\/\textipa{""rE\textdyoghlig I"streIS@n}/} & \emph{n.} 登记 & the act of making an official record of sth/sb \\
{\textbf{regression}\\/\textipa{rI"grES@n}/} & \emph{n.} 回归;退化;倒退 & the process of going back to an earlier or less advanced form or state \\
{\textbf{tentative}\\/\textipa{"tEnt@tIv}/} & \emph{n.} 试探;尝试;实验 & not definite or certain because you may want to change it later \\

\bottomrule

\end{longtblr}
}

\begin{problemset}
  \item \textbf{With the "Steps in M2"offered, try to identify the steps in the following 2 examples of M2 and write down the corresponding numbers in the blanks in front of each part.}
  
  \hspace*{2em}Steps in M2:

  \hspace*{2em}2S1 Reporting on medical ethics review
  
  \hspace*{2em}2S2 Explaining briefly research design

  \hspace*{2em}2S3 Describing subjects or data and their selection criteria

  \hspace*{2em}2S4 Describing experimental procedure, such as interventions, examinations, etc.

  \hspace*{2em}2S5 Describing main outcomes and their measures

  \hspace*{2em}2S6 Describing data analysis methods

  \hspace*{2em}2S7 Reporting on registration information

  \textbf{Example 1}

  \hspace*{2em}\uline{\hspace*{3em}}Design: Retrospective case-control study.

  \hspace*{2em}\uline{\hspace*{3em}}Setting: Nijmegen, the Netherlands.

  \hspace*{2em}\uline{\hspace*{3em}}Patients: Thirty consecutive patients with SSD.
  
  \hspace*{2em}\uline{\hspace*{3em}}Interventions: Patients received a trial with a BCD headband as part of the regular workup for SSD. The patients were divided into 2 groups according to their decision to opt for a BCD (BCD+) or not (BCD-).
  
  \hspace*{2em}\uline{\hspace*{3em}}Main outcome measures: Patients completed a questionnaire on satisfaction with the BCD headband, patient-and BCD-related factors, and benefit in listening situations.
  
  \textbf{Example 2}
  
  \hspace*{2em}Methods: 
  
  \hspace*{2em}\uline{\hspace*{3em}} Eighteen adults with first-ever chronic monohemispheric subcortical stroke participated in \uline{\hspace*{3em}} this randomized, controlled, triple-blinded trial. \uline{\hspace*{3em}} Intervention consisted of priming with real or sham iTBS to the ipsilesional primary motor cortex immediately before 45 minutes of upper limb physical therapy, daily for 10 days. \uline{\hspace*{3em}} Changes in upper limb function (Action Research Arm Test [ARAT]), upper limb impairment (Fugl-Meyer Scale), and corticomotor excitability, were assessed before, during, and immediately, 1 month and 3 months after the intervention. Functional magnetic resonance images were acquired before and at one month after the intervention.

  \item \textbf{Fill in the blanks with the correct form of the verbs given. Notice the tense and voice.}
  
  \begin{enumerate}
    \item Methods: A population-based prospective cohort study\uline{~~~1)~~~}(include) 3,504 male and female Koreans aged 40 to 69 years from the Korean Genome Epidemiology Study. At the beginning of follow-up, all individuals \uline{~~~2)~~~}(be) free of metabolic syndrome and known cardiovascular disease. Each participant\uline{~~~3)~~~}(complete) a food frequency questionnaire. Incident cases of metabolic syndrome \uline{~~~4)~~~}(identify) by biennial health examinations during a follow-up period between April 17, 2003, and November 17, 2006.
    Pooled logistic regression analysis \uline{~~~5)~~~} (apply) to obtain an odds ratio (OR) of metabolic syndrome with its 95\% confidence interval for fish or n-3 fatty acid intake.
    
    \item Methods: This\uline{~~~1)~~~}(be) a retrospective matched cohort study of women with liver cirrhosis between January 2005 and January 2016 in a university hospital. Women in a case group\uline{~~~2)~~~}(match) to women in a control group according to year of delivery, age, body mass index, and parity in a 1:4 ratio. Bivariable and multivariable analyses \uline{~~~3)~~~}(perform) \uline{~~~4)~~~}(compare) the prevalence of the primary composite outcome, which \uline{~~~5)~~~}(include) any one of the following: fetal or neonatal demise, placental abruption, preeclampsia, preterm delivery at less than 37 weeks of gestation, and small-for-gestational age neonate between women in the case group and those in the control group.
    
  \end{enumerate}

  \item The following steps in M2 need improvement. You may have to change a word, add a word or delete a word. Mark out the mistakes and put the corrections in the blank provided. Ifyou change a word, underline it and write the correct word in the corresponding blank. If youd add a word, put an insertion mark ($\wedge$) in the right place and write the added word in the blank. If you delete a word, cross it out and put a slash (/) in the blank. If you consider the sentence as correct, put a tick ($\surd$).
  
  \hspace*{2em}Design, setting and participants: A randomize clinical trial of patients in\sidenote{(1)\uline{\hspace*{3em}}} persistent hypercapnia (Pa\ce{CO2}$>53$ mmHg) 2 weeks to 4 weeks after resolution\sidenote{(2)\uline{\hspace*{3em}}} of respiratory acidemia, who recruited from 13 UK centers between 2010 and\sidenote{(3)\uline{\hspace*{3em}}} 2015. Exclusion criterium included obesity (body mass index $[BMI] >35$),\sidenote{(4)\uline{\hspace*{3em}}} obstructive sleep apnea syndrome, or other causes of respiratory failure. Of 2021 patients screening, 124 were eligible.\sidenote{(5)\uline{\hspace*{3em}}}

  \hspace*{2em}Interventions: There were 59 patients randomized as to home oxygen alone\sidenote{(6)\uline{\hspace*{3em}}} (medial oxygen flow rate, 1.0 L/min [ interquartile range (IQR), 0.5--2.0\sidenote{(7)\uline{\hspace*{3em}}} L/min]) and 57 patients to home oxygen plus home NIV (median oxygen flow rate, 1. 0 L/min [IQR, 0.5--1.5 L/min]). The median home\sidenote{(8)\uline{\hspace*{3em}}} ventilator settings are an inspiratory positive airway pressure of 24 (IQR, 22--\sidenote{(9)\uline{\hspace*{3em}}}26) cm H2O, expiratory positive airway pressure of 4 (IQR, 4--5) cm H2O, and a backup rate of 14 (IQR, 14--16) breaths/minute.

  \hspace*{2em}Main outcomes and measures: Time to readmission or death within 12 months adjusted for the number of previous COPD admissions, previous use of longterm\sidenote{(10)\uline{\hspace*{3em}}} oxygen, age, and BMI.
  

\end{problemset}



\chapter{Move Three}\label{chapter5}

Move Three (M3) is the third part of an abstract. The communicative purpose of this part is to summarize the findings or results of the study. It is the longest part in an abstract. However, M3 is not structurally hierarchical, which is different from the other three moves in an abstract.

\section{Steps in M3}

The steps discovered in M3 include:

3S1 Providing information on valid samples

3S2 Illustrating overall observation or main results 

These two steps follow the order listed above, with rare exceptions, to fulfill the authors' communicative purpose (Sample~\ref{sam:M3-1}). Of these two steps, 3S1 is optional, and 3S2 is conventional (Sample~\ref{sam:M3-2}). It is difficult to further stratify 3S2 in spite of its length.

\begin{sample}[label={sam:M3-1}]{\heiti}

  \textbf{RESULTS}

  60 individuals were recruited over a 9-month period. Three withdrew, leaving 29 intervention and 28 controls participants in the final analysis. 32\% of patients with FMS met the inclusion criteria, of which 90\% enrolled. Acceptability of the intervention was high and there were no adverse events. At 6 months, 72\% of the intervention group rated their symptoms as improved, compared to 18\% in the control group. There was a moderate to large treatment effect across a range of outcomes, including three of eight Short Form 36 (SF36) domains $(d=0.46-0.79$). The SF36 Physical function was found to be a suitable primary outcome measure for a future trial; adjusted mean difference 19.8(95\% CI 10.2 to 29.5). The additional quality adjusted life vears (OALY) with intervention was 0.08(95\% CI 0. 03 to 0. 13), the mean incremental cost per QALY gained was \pounds{}12 087.

  
  \begin{flushright}
    ---Randomised feasibility study of physiotherapy for patients with functional motor symptoms. \emph{Journal of Neurology, Neurosurgery, and Psychiatry (2016)}
  \end{flushright}

  \tcblower

  \noindent \textbf{STEP IDENTIFICATION}

  \vspace*{10pt}
  {\small\noindent
  \begin{tblr}{colspec={X[1,c]X[5,l]},cell{even}{1,2} = {azure9}}
    \toprule
    \textbf{Step} & \textbf{Sample} \\ 
    \midrule
    
    3S1 & 60 individuals were recruited over a 9-month period. Three withdrew, leaving 29 intervention and 28 controls participants in the final analysis. 32\% of patients with FMS met the inclusion criteria, of which 90\% enrolled.\\
    3S2 & Acceptability of the intervention was high and there were no adverse events. At 6 months, 72\% of the intervention group rated their symptoms as improved, compared to 18\% in the control group. There was a moderate to large treatment effect across a range of outcomes, including three of eight Short Form 36 (SF36) domains $(d=0.46-0.79$). The SF36 Physical function was found to be a suitable primary outcome measure for a future trial; adjusted mean difference 19.8(95\% CI 10.2 to 29.5). The additional quality adjusted life vears (OALY) with intervention was 0.08(95\% CI 0. 03 to 0. 13), the mean incremental cost per QALY gained was \pounds{}12 087.\\
      
    \bottomrule
  \end{tblr}
  }


\end{sample}

\begin{sample}[label={sam:M3-2}]{\heiti}
  
  \textbf{RESULTS }
  
  Scores for FVA, VR QoL, and HR QoL were reduced in children with glaucoma: median CVAQC score, -1. 24 (interquartile range [ IQR],-2.2 to -0.11; range, -3. 00 higher visual ability to +2. 80 lower visual ability); mean IVI-C score, 67.3 (SD, 14.4; normal VR QoL, 96); median PedsQL self-report, 78. 8 (IQR, 67.4--90. 2); parent report, 71.2 (IQR, 55.7--85.8); and family impact score, 74.3 (IQR, 56.9-88.5; normal HR QoL, 100). Psychosocial subscores were lower than physical subscores on the PedsQL. Older children reported less impairment on CVAQC, IVI-C, and PedsQL than younger children. Parents reported greater impact on their child's HR QoL than children reported themselves.


  \begin{flushright}
    ---Quality of Life and Functional Vision in Children with Glaucoma. \emph{Ophthalmology (2017)}
  \end{flushright}

  \tcblower

  \noindent \textbf{STEP IDENTIFICATION}

  \vspace*{10pt}
  {\small\noindent
  \begin{tblr}{colspec={X[1,c]X[5,l]},cell{even}{1,2} = {azure9}}
    \toprule
    \textbf{Step} & \textbf{Sample} \\
    \midrule
    
    3S2 & Scores for FVA, VR QoL, and HR QoL were reduced in children with glaucoma: median CVAQC score, -1. 24 (interquartile range [ IQR],-2.2 to -0.11; range, -3. 00 higher visual ability to +2. 80 lower visual ability); mean IVI-C score, 67.3 (SD, 14.4; normal VR QoL, 96); median PedsQL self-report, 78. 8 (IQR, 67.4--90. 2); parent report, 71.2 (IQR, 55.7--85.8); and family impact score, 74.3 (IQR, 56.9-88.5; normal HR QoL, 100). Psychosocial subscores were lower than physical subscores on the PedsQL. Older children reported less impairment on CVAQC, IVI-C, and PedsQL than younger children. Parents reported greater impact on their child's HR QoL than children reported themselves.\\
      
    \bottomrule
  \end{tblr}
  }

\end{sample}

\section{Language Features in Each Step}

Similar to most steps in M2, M3 uses past tense, and both active voice and passive voice.

  \subsection{Step 1(3S1) Providing information on valid samples}
    \subsubsection{Step Analysis}

    In an experiment-based study, extra information about the valid sample(s) may be supplemented, such as the number, age and grouping, etc.

    \subsubsection{Language Realizations}

    Sometimes, 3S1 is in past tense and passive voice with the sample(s) as the sentence subject. Sometimes it is in active voice.

    \begin{eg}{}
      Of 1072 neonates screened, 523 \uline{were assigned} to hydrocortisone (n= 256) or placebo (n=267) and 406 \uline{survived} to 2 years of age.  
    \end{eg}

    \subsubsection{Lexical Chunks}

    \begin{enumerate}
      \item (patients) were randomly assigned to (receive)
      \begin{eg}{}
        Of 381 patients screened, 142 eligible \uline{patients were randomly assigned to} treatment (72 to the vandetanib group and 70 to the placebo group).
      \end{eg}

      \begin{eg}{}
        During this time, 239 infants were assessed and 181 eligible infants \uline{were randomly assigned to receive} an LMA (n=85) or an endotracheal tube (n=95).    
      \end{eg}

      \item (patients) were included in the (study)
      \begin{eg}{}
        In all, 198 \uline{patients were included in the} final model.
      \end{eg}

      \begin{eg}{}
        A total of 702 women \uline{were included in the study}, with 29 (4.1\%) maternal deaths, and a mortality ratio of 56. 98 deaths per 100,000 live births.  
      \end{eg}

      \begin{eg}{}
        These variables \uline{were included in the} RSS and assigned scores ranging from 0 to 6. 
      \end{eg}

      \item were included in the analysis
      \begin{eg}{}
        A total of 4011 eyes of 2057 subjects with T2DM \uline{were included in the analysis}.
      \end{eg}

      \item patients were enrolled and
      \begin{eg}{}
        Between Jan 8,2013, and Jan 31,2014,495 eligible adult \uline{patients were enrolled and} randomly assigned to the cobimetinib plus vemurafenib group (n=247) or placebo plus vemurafenib group (n=248).   
      \end{eg}
    \end{enumerate}

  \subsection{Step 2(3S2) Illustrating overall observation or main results}
    \subsubsection{Step Analysis}

    As an important step in M3, this step states the main findings or results of the study. Generally speaking, the overview of the findings comes before the details, as we can see from the example below.

    \begin{eg}{}
      \par 

      \vspace*{10pt}
      {\small\noindent
      \begin{tblr}{colspec={X[1,c]X[5,l]},cell{even}{1,2} = {azure9},cell{odd}{1,2} = {white}}
        \toprule
        \textbf{Step} & \textbf{Sample} \\ 
        \midrule
        
        1S3a & The authors investigated whether parecoxib-supplemented IV morphine analgesia could decrease the incidence of delirium in elderly patients after total hip or knee replacement surgery.\\
        & $\dots$\\
        3S2 & The incidence of delirium was significantly reduced from 11.0\% (34/310) with placebo to 6. 2\% (19/310) with parecoxib (relative risk 0.56, 95\% confidence interval $0.33-0.96$, $P=.031$). The severity of pain and the cumulative consumptions of morphine at 24, 48, and 72 hours after surgery were significantly lower with parecoxib than with placebo (all $P<. 001$), although the differences were small. There was no difference in the incidence of postoperative complications between the 2 groups (12.3\%[38/310] with placebo versus
        11.6\% [36/310] with parecoxib; P=.80).\\
          
        \bottomrule
      \end{tblr}
      }
    \end{eg}

    This example first answers the research aim and research question ``whether parecoxib-supplemented IV morphine analgesia could decrease the incidence of delirium in elderly patients after total hip or knee replacement surgery'' in 1S3a, and then states the significance of the findings from ``the severity of pain'', ``cumulative consumptions'' and ``postoperative complications''.

    \subsubsection{Language Realizations}

    Past tense is the most obvious language feature of 3S2. Besides, passive voice is preferred over active voice, so as to indicate the objectivity of the research findings. It is found that the verbs used to show the results are also signals of this step, such as ``found'', ``observed'' and so on. Still, there exist cases in active voice. For example, the structure ``analysis showed'' may be adopted to create the leadership of the research analysis with a neutral and scientific-sounding tone.

    \begin{eg}{}
      No change in measures of daily PA was observed at 4-wk compared with baseline($P>0.05$)   
    \end{eg}

    \begin{eg}{}
      Multivariated \uuline{analysis} \uline{showed} that nodules less than 1. 5 cm, ultrasonographic findings suggestive of malignancy and more than 2 results of atypia from repeated FNAs were significant risk factors for malignancy ($P<0.001$).  
    \end{eg}

    Correlation and inter-group differences etc. are often involved in medical studies, so relational signals establish the relations between elements, which include resultative signals (e.g. \emph{these findings suggest that}), contrastive signals (e.g. \emph{no significant difference in}) and inferential signals (e.g. \emph{was found to be}).

    Besides, adjectives indicating epistemic modality, such as ``likely'', which conveys a degree of likelihood, are also characteristic of this step. They show impersonal epistemic stance to express degrees of possibilities in the research results.

    \begin{eg}{}
      Recidivists were more \uline{likely} to be male (P<0.0001), Black (P<0.0001), have a blood alcohol content above 80 mg/dL (P<0.0001) compared with nonrecidivists.  
    \end{eg}

    \subsubsection{Lexical Chunks}

    \begin{enumerate}
      \item (there was) no (significant) difference (in/between)
      \begin{eg}{}
        However, \uline{there was no significant difference in} the per cent of change in those parameters. 
      \end{eg}

      \begin{eg}{}
        After matching, \uline{there was no difference in} the 30-day rate of AMI between testing modalities.  
      \end{eg}

      \begin{eg}{}
        Model 1 showed \uline{no significant difference between} predicted and observed events (risk ratio (RR)=0. 87,95\% Cl0. 16--4.62).
      \end{eg}

      \item (there were) no (significant) differences (in/between)
      \begin{eg}{}
        \uline{There were no significant differences in} the duration of the surgical procedure ($P=0.12$), weight of the surgical specimen ($P=0.54$) or the patients' pain perception ($P=0.28$).
      \end{eg}

      \begin{eg}{}
        \uline{There were no significant differences between} men and women. 
      \end{eg}

      \begin{eg}{}
        The interventions did not raise any safety concerns and \uline{there were no differences between} groups in serious or other adverse events.  
      \end{eg}

      \begin{eg}{}
        \uline{There were no differences in} secondary outcomes except for clinician satisfaction with ease of administration.
      \end{eg}

      \item there was/were no significant
      \begin{eg}{}
        \uline{There was no significant} interaction between BNP and the 2 independent variables (P=.60, and P=.90), respectively.        
      \end{eg}

      \begin{eg}{}
        \uline{There were no significant} changes in number of total and regional nevi count and in the dermoscopic features of nevi between biological and conventional treatment groups.
      \end{eg}

      \item (was/were) not significantly different (between)
      \begin{eg}{}
        Results showed that compared to PWNS, PWS \uline{were not significantly different} in matching either the phase (timing) or the amplitude of the target in both jaw and hand tracking of predictable and unpredictable targets.
      \end{eg}

      \begin{eg}{}
        The allele frequency of this polymorphism in individuals originating from two locations with different malaria endemicity in the past \uline{was not significantly different}.  
      \end{eg}

      \begin{eg}{}
        ... and these rates were \uline{not significantly different between} groups A and B ($P=0. 346$ and $P=0. 370$, respectively).          
      \end{eg}

      \item did not differ (significantly) between (the)
      \begin{eg}{}
        Intraclass correlation coefficients (ICCs) for interreader and intrareader agreement \uline{did not differ significantly between} measurements for FA and MD. 
      \end{eg}

      \begin{eg}{}
        No serious adverse reactions were recorded and other safety measures \uline{did not differ between the} groups, after allowing for missing data. 
      \end{eg}

      \begin{eg}{}
        Complications \uline{did not differ between} devices.
      \end{eg}

      \item was/were (not) (significantly/independently) associated with (a/an)
      \begin{eg}{}
        Additionally, in subgroup analyses, rescue breathing \uline{was not associated with} neurological outcome regardless of the type of rescuer [family member: adjusted OR $0. 83$ (95\% CI 0.39--1.70); or non-family member; adjusted OR 1.91 (95\% CI 0.79--5.35)].  
      \end{eg}

      \begin{eg}{}
        AD \uline{was significantly associated with} 11 of 22 examined autoimmune diseases.   
      \end{eg}

      \begin{eg}{}
        Hypotension \uline{was independently associated with} mortality, acute kidney injury and hospital admission.    
      \end{eg}

      \item (serious) adverse events were reported (in)
      \begin{eg}{}
        Treatment-emergent \uline{adverse events were reported in} 103 patients (97\%), a majority of which were grade 1 to 2 in severity.
      \end{eg}

      \begin{eg}{}
        No \uline{serious adverse events were reported}.    
      \end{eg}

      \item were more/less likely to (have)
      \begin{eg}{}
        Survivors of cancer \uline{were more likely to have} chronic conditions and MCCs compared with adults without a history of cancer. 
      \end{eg}

      \begin{eg}{}
        Men and higher-educated people \uline{were more likely to} be highly sedentary, while women and lower-educated people \uline{were more likely to} be inactive.
      \end{eg}

      \item (was) (significantly) higher/lower in (the)$\dots$
      \begin{eg}{}
        Percent MVIC retained \uline{was significantly higher in} ACB patients at 6 ($P<0. 000 1$) and 24 hours ($P<0. 000 1$).         
      \end{eg}

      \begin{eg}{}
        PCT levels were \uline{significantly higher in the} population with Gram-negative rod (GNR) infections than in the population with Gram-positive coccal (GPC) infections.         
      \end{eg}

      \item was found to be
      \begin{eg}{}
        The SF36 Physical function \uline{was found to be} a suitable primary outcome measure for a future trial; adjusted mean difference 19. 8(95\% Cl 10.2 to 29.5).   
      \end{eg}

      \begin{eg}{}
        No statistically significant difference in overall survival was observed \uline{between the two groups} (102.7 vs 115.7 months, respectively). 
      \end{eg}
    \end{enumerate}

\section{Sample Reading}

Two samples of M3 are presented here, followed by a detailed analysis on the language features of the steps included.

\begin{sample}[label={myautocounter}]{\heiti}
  
  RESULTS

  60 individuals were recruited over a 9-month period. Three withdrew, leaving 29 intervention and 28 controls participants in the final analysis. 32\% of patients with FMS met the inclusion criteria, of which 90\% enrolled. Acceptability of the intervention was high and there were no adverse events. At 6 months, 72\% of the intervention group rated their symptoms as improved, compared to 18\% in the control group. There was a moderate to large treatment effect across a range of outcomes, including three of eight Short Form 36 (SF36) domains ($d=0.46-0.79$). The SF36 Physical function was found to be a suitable primary outcome measure for a future trial; adjusted mean difference 19. 8 (95\% CI 10.2 to 29.5). The additional quality adjusted life years (QALY) with intervention was 0.08 (95\% CI 0.03 to 0.13), the mean incremental cost per QALY gained was \pounds{}12 087.


  \begin{flushright}
    ---Randomised feasibility study of physiotherapy for patients with functional motor symptoms. \emph{Journal of Neurology, Neurosurgery, and Psychiatry (2016)}
  \end{flushright}

  \tcblower

  \noindent \textbf{STEP IDENTIFICATION}

  \vspace*{10pt}
  {\small\noindent
  \begin{tblr}{colspec={X[1,c]X[5,l]},cell{even}{1,2} = {azure9}}
    \toprule
    \textbf{Step} & \textbf{Sample} \\ 
    \midrule
  
    3S1 & 60 individuals were recruited over a 9-month period. Three withdrew, leaving 29 intervention and 28 controls participants in the final analysis. 32\% of patients with FMS met the inclusion criteria, of which 90\% enrolled.\\
    3S2 & Acceptability of the intervention was high and there were no adverse events. At 6 months, 72\% of the intervention group rated their symptoms as improved, compared to 18\% in the control group. There was a moderate to large treatment effect across a range of outcomes, including three of eight Short Form 36 (SF36) domains ($d=0.46-0.79$). The SF36 Physical function was found to be a suitable primary outcome measure for a future trial; adjusted mean difference 19. 8 (95\% CI 10.2 to 29.5). The additional quality adjusted life years (QALY) with intervention was 0.08 (95\% CI 0.03 to 0.13), the mean incremental cost per QALY gained was \pounds{}12 087.\\
    
    \bottomrule
  \end{tblr}
  }

  \noindent \textbf{ANALYSIS}

  This is an M3 sample with two steps in past tense. The first step 3S1 offers extra information about the valid samples, that is, 90\% of those with FMS who met the inclusion criteria. Passive voice is employed in this step with ``60 individuals'' as the subject. The second step 3S2 contains the major findings beginning with an overview, that is, ``acceptability of the intervention was high and there were no adverse events.'' More data of the outcome are offered in the following part, with some signals and lexical chunks commonly seen in 3S2, such as ``compared to'', ``in the control group'', ``was found to be'', ``a suitable primary outcome measure'', and ``mean difference'', etc. Both active voice (e.g. ``72\% of the intervention group rated their symptoms as$\dots$'') and passive voice (e.g. ``the SF36 Physical function was found to be$\dots$'') are used.

\end{sample}

\begin{sample}[label={myautocounter}]{\heiti}

  \textbf{RESULTS}
  
  Scores for FVA, VR QoL, and HR QoL were reduced in children with glaucoma: median CVAQC score,-1.24 (interquartile range [IQR], -2.2 to -0.11; range, -3.00 higher visual ability to +2.80 lower visual ability); mean IVI-C score, 67.3 (SD, 14.4; normal VR QoL, 96); median PedsQL self-report, 78.8 (IQR, 67.4-90.2); parent report, 71.2 (IQR, 55.7--85.8); and family impact score, 74.3 (IQR, 56.9--88.5; normal HR QoL, 100). Psychosocial subscores were lower than physical subscores on the PedsQL. Older children reported less impairment on CVAQC, IVI-C, and PedsQL than younger children. Parents reported greater impact on their child's HR QoL than children reported themselves.

  
  \begin{flushright}
    ---Quality of Life and Functional Vision in Children with Glaucoma. \emph{Ophthalmology (2017)}
  \end{flushright}

  \tcblower

  \noindent \textbf{STEP IDENTIFICATION}

  \vspace*{10pt}
  {\small\noindent
  \begin{tblr}{colspec={X[1,c]X[5,l]},cell{even}{1,2} = {azure9}}
    \toprule
    \textbf{Step} & \textbf{Sample} \\ 
    \midrule
  
    3S2 & Scores for FVA, VR QoL, and HR QoL were reduced in children with glaucoma: median CVAQC score,-1.24 (interquartile range [IQR], -2.2 to -0.11; range, -3.00 higher visual ability to +2.80 lower visual ability); mean IVI-C score, 67.3 (SD, 14.4; normal VR QoL, 96); median PedsQL self-report, 78.8 (IQR, 67.4-90.2); parent report, 71.2 (IQR, 55.7--85.8); and family impact score, 74.3 (IQR, 56.9--88.5; normal HR QoL, 100). Psychosocial subscores were lower than physical subscores on the PedsQL. Older children reported less impairment on CVAQC, IVI-C, and PedsQL than younger children. Parents reported greater impact on their child's HR QoL than children reported themselves.\\
    
    \bottomrule
  \end{tblr}
  }

  \noindent \textbf{ANALYSIS}

  This M3 is made up of only one step (3S2). An overview comes first, that is, ``Scores for FVA, VR QoL, and HR QoL were reduced in children with glaucoma'', followed by a presentation of a lot of data and their significance. Comparisons of subscores and of different groups (older children vs. younger children, parents vs. children) are made in the latter half of this step. As for the language realization, it is obvious that this step is in the past tense, and when the subject of the research, that is, ``scores'' is taken as the sentence subject, the passive voice is preferred to indicate the objectivity of the findings, but the active voice is also used, as we can see from ``older children reported$\dots$'' and ``parents reported$\dots$''.
  
\end{sample}

\section{Glossary}

{\small
\begin{longtblr}[
    caption = {Glossary of Chapter 5},
    label = {tab:Glossary of Chapter 5},
    % note{a} = {英文论文中指代当前文献中的差距、问题或缺陷。即现有研究尚未解决的部分。},
]{
    width = \textwidth,
    colspec = {X[1,l,h]  X[1,l,h]  X[3,l,h]},
    rowhead = 1, rowfoot = 0, % 每个分页里表头表尾的数量
    % row{odd} = {blue8}, 
    row{even} = {azure9},
}
    
\toprule
\textbf{WORDS} & \textbf{MEANING} & \textbf{MEANING or EXAMPLE}\\
\midrule

{\textbf{adverse}\\/\textipa{\ae d"v3rs}/} & \emph{adj.} 不利的;有害的;反面的 & negative and unpleasant; not likely to produce a good result \\
{\textbf{correlation}\\/\textipa{""k0r@"leIS@n}/} & \emph{n.} 相关;关联;相互关系 & a connection between two things in which one thing changes as the other does \\
{\textbf{epistemic}\\/\textipa{""EpI"stimIk}/} & \emph{adj.} 认识的;知识的 & of or pertaining to knowledge or the conditions for acquiring it \\
{\textbf{hierarchical}\\/\textipa{""haI@r"arkIk@l}/} & \emph{adj.} 阶层式的;分层的;分等级的;层级式的 & classified according to various criteria into successive levels or layers \\
{\textbf{stratify}\\/\textipa{"str\ae t@""faI}/} & \emph{v.} (使)分层 & to arrange sth in layers \\

\bottomrule

\end{longtblr}
}

\begin{problemset}
  \item \textbf{Fill the blanks with the correct forms of the given words. Notice the tense and voice.}
  \begin{enumerate}
    \item Results: The utilization of regional anesthetic techniques \uline{~~~1)~~~}(not differ) by OSA status and overall $<25\%$ and $15\%$ \uline{~~~2)~~~}(receive) neuraxial anesthesia and peripheral nerve blocks, respectively. Trend analysis \uline{~~~3)~~~}(show) a significant increase in peripheral nerve block use by $>50\%$ and a concurrent decrease in opioid prescription. Interestingly, while the absolute number of patients with OSA receiving perioperative oximetry, supplemental oxygen, and positive airway pressure therapy significantly \uline{~~~4)~~~}(increase) over time, the proportional use significantly \uline{~~~5)~~~}(decrease) by approximately 28\%, 36\%, and 14\%, respectively. Ashift from utilization of intensive care to telemetry and stepdown units \uline{~~~6)~~~}(see).
    
    \item Results: After controlling for potential cardiovascular risk factors, multivariate OR for metabolic syndrome \uline{~~~1)~~~}(be) 0.43(95\% CI 0.23 to 0.83) for men who \uline{~~~2)~~~}(eat) fish daily when compared with those eating fish less than once a week. Similarly, metabolic syndrome risk \uline{~~~3)~~~}(halve) for men in the top decile of n-3 fatty acid intake when compared with those in the bottom decile (OR 0.53,95\% CI 0.28 to 0.99). In particular, fish intake \uline{~~~4)~~~}(associate) significantly with triglyceride level and high-density lipoprotein cholesterol level among the metabolic syndrome components. For women, apparent associations \uline{~~~5)~~~}(not observe) between fish intake or n-3 fatty acid intake and metabolic syndrome risk.
  \end{enumerate}

  \item \textbf{``Significance'' and ``difference'' are two of the key words in M3, so we should be familiar with their different forms. Please fill in the blanks with the proper forms of the words given.}
  \begin{enumerate}
    \item \fbox{significance, significant, significantly}
    \begin{enumerate}
      \item Overall survival with pembrolizumab was \uline{\hspace*{3em}} longer in patients who previously received any radiotherapy than in patients without previous radiotherapy.
      \item This benefit was consistent across all subgroups examined (all $P<0.05$), and no \uline{\hspace*{3em}} heterogeneity of treatment effect was observed (all $P>0.05$).
      \item Different trajectories and \uline{\hspace*{3em}} of B-type natriuretic peptide, congestion and acute kidney injury in patients with heart failure.
      \item Nonetheless, \uline{\hspace*{3em}} deficits in knowledge, particularly for the diagnostic criteria for delirium, remained.
      \item Both the amount of ICG injected ($P<0.001$) and the experimental temperature ($P<0.001$) \uline{\hspace*{3em}} affected the measurements.
      \item Further, there were no \uline{\hspace*{3em}} between-group differences in motor practice effects for either jaw or hand tracking.
    \end{enumerate}

    \item \fbox{differ, difference(s), different, differently}
    \begin{enumerate}
      \item These rates were not significantly \uline{\hspace*{3em}} between groups A and B (P=0.346 and P=0.370, respectively).
      \item \uline{\hspace*{3em}} in health insurance benefit phase, drug choice, brand name, and coverage type were the greatest determinants of patient cost (P<0.001).
      \item US medical examiners and coroners apply variable practices to classify and investigate SUID, and thus, they certify the same deaths \uline{\hspace*{3em}}.
      \item Total costs of a 30-day supply from a specialist \uline{\hspace*{3em}} from family and internal medicine physicians by \${}7.36--\${}14.57.
      \item Additionally, three mutations in three \uline{\hspace*{3em}} genes were found.
      \item The two groups of patients \uline{\hspace*{3em}} for comorbidities and treatment profile.
      \item We examined whether piRNAs are expressed \uline{\hspace*{3em}} between AD cases and controls and explored the potential regulatory effects of risk SNPs on piRNA expression levels.
      \item No serious adverse reactions were recorded and other safety measures did not \uline{\hspace*{3em}} between the groups, after allowing for missing data.
    \end{enumerate}
  \end{enumerate}
\end{problemset}

\chapter{Move Four}\label{chapter6}

Move Four (M4) is the last move in the abstract for medical academic paper, where the pivotal results may be highlighted and explained, the significance of the study indicated and recommendations given. Suggestions may also be made for future work or follow-up research. In a hypothesis-driven study, it also needs to be stated clearly in this move whether the hypothesis is supported by the results or not.

\section{Steps in M4}

M4 involves four steps with a relatively fixed order.

4S1 Reiterating pivotal results

4S2 Indicating limitations

4S3 Stating the significance of the results

4S4 Predicting future studies

\begin{sample}[label={myautocounter}]{\heiti}
  
  \textbf{INTERPRETATION} 
  
  In patients with chronic migraine, erenumab 70 mg and 140 mg reduced the number of monthly migraine days with a safety profile similar to placebo, providing evidence that erenumab could be a potential therapy for migraine prevention. Further research is needed to understand longterm efficacy and safety of erenumab, and the applicability of this study to real-world settings.


  \begin{flushright}
    ---Safety and efficacy of erenumab for preventive treatment of chronic migraine: a randomised, double-blind, placebo-controlled phase 2 trial. \emph{The Lancet Neurology (2017)}
  \end{flushright}

  \tcblower

  \noindent \textbf{STEP IDENTIFICATION}

  \vspace*{10pt}
  {\small\noindent
  \begin{tblr}{colspec={X[1,c]X[5,l]},cell{even}{1,2} = {azure9}}
    \toprule
    \textbf{Step} & \textbf{Sample} \\ 
    \midrule
  
    4S1 & In patients with chronic migraine, erenumab 70 mg and 140 mg reduced the number of monthly migraine days with a safety profile similar to placebo, \\
    4S3 & providing evidence that erenumab could be a potential therapy for migraine prevention. \\
    4S4 & Further research is needed to understand longterm efficacy and safety of erenumab, and the applicability of this study to real-world settings. \\
    
    \bottomrule
  \end{tblr}
  }

\end{sample}

\begin{sample}[label={myautocounter}]{\heiti}
  
  \textbf{CONCLUSION} 
  
  Including information from \ce{^{18}F}-fluciclovine PET in the treatment-planning process led to significant differences in the defined target volume, with higher doses to the penile bulb but no significant differences in rectal or bladder dose or in acute genitourinary or gastrointestinal toxicity. Longer follow-up is needed to determine the impact of 18F-fluciclovine PET on cancer control and late toxicity endpoints.

  \begin{flushright}
    ---Impact of \ce{^{18}F}-fluciclovine PET on Target Volume Definition for Postprostatectomy Salvage Radiotherapy: Initial Findings from a Randomized Trial. \emph{The Journal of Nuclear Medicine (2017)}
  \end{flushright}

  \tcblower

  \noindent \textbf{STEP IDENTIFICATION}

  \vspace*{10pt}
  {\small\noindent
  \begin{tblr}{colspec={X[1,c]X[5,l]},cell{even}{1,2} = {azure9}}
    \toprule
    \textbf{Step} & \textbf{Sample} \\ 
    \midrule
  
    4S3 & CONCLUSION: Including information from \ce{^{18}F}-fluciclovine PET in the treatment-planning process led to significant differences in the defined target volume, with higher doses to the penile bulb but no significant differences in rectal or bladder dose or in acute genitourinary or gastrointestinal toxicity.\\
    4S4 & Longer follow-up is needed to determine the impact of 18F-fluciclovine PET on cancer control and late toxicity endpoints.\\
    
    \bottomrule
  \end{tblr}
  }
  
\end{sample}

\begin{sample}[label={myautocounter}]{\heiti}
  
  \textbf{CONCLUSION }
  
  Our approach results in a response rate of 40\% or more, with acceptable toxicity. \ce{^{18}F}-FMISO uptake in NSCLC patients is strongly associated with poor prognosis features that could not be reversed by radiotherapy doses up to 86 Gy.


  \begin{flushright}
    ---Phase II study of a radiotherapy total dose increase in hypoxic lesions identified by F-miso PET/CT in patients with non-small cell lung carcinoma. \emph{The Journal of Nuclear Medicine (2017)}
  \end{flushright}

  \tcblower

  \noindent \textbf{STEP IDENTIFICATION}

  \vspace*{10pt}
  {\small\noindent
  \begin{tblr}{colspec={X[1,c]X[5,l]},cell{even}{1,2} = {azure9}}
    \toprule
    \textbf{Step} & \textbf{Sample} \\ 
    \midrule
  
    4S1 & CONCLUSION Our approach results in a response rate of 40\% or more, with acceptable toxicity.\\
    4S3 & \ce{^{18}F}-FMISO uptake in NSCLC patients is strongly associated with poor prognosis features that could not be reversed by radiotherapy doses up to 86 Gy.\\
    
    \bottomrule
  \end{tblr}
  }

\end{sample}

\begin{sample}[label={myautocounter}]{\heiti}
  
  \textbf{CONCLUSION}
  
  These findings support the hypothesis and endorse ARIP as a safer APD for alleviating behavioral disturbances after TBI.


  \begin{flushright}
    ---Relative to typical antipsychotic drugs, aripiprazole is a safer alternative for alleviating behavioral disturbances after experimental brain trauma. \emph{Neurorehabilitation and Neural Repair (2016)}
  \end{flushright}

  \tcblower

  \noindent \textbf{STEP IDENTIFICATION}

  \vspace*{10pt}
  {\small\noindent
  \begin{tblr}{colspec={X[1,c]X[5,l]},cell{even}{1,2} = {azure9}}
    \toprule
    \textbf{Step} & \textbf{Sample} \\ 
    \midrule
  
    4S3 & CONCLUSION These findings support the hypothesis and endorse ARIP as a safer APD for alleviating behavioral disturbances after TBI.\\

    \bottomrule
  \end{tblr}
  }
  
\end{sample}

From the samples listed above it could be observed that 4S3, in many cases, is a conventional step in M4, while other steps are optional. By analyzing the major findings, M4 points out the significance or implications of the research and draws a conclusion to it.

\section{Language Features in Each Step}
  \subsection{Step 1(4S1) Reiterating pivotal results}
    \subsubsection{Step Analysis}

    Usually more than one result is achieved by the research, as presented in M3, but only the most pivotal ones are summarized and restated in this step.

    \begin{eg}{}
      \par
      \vspace*{10pt}
      {\small\noindent
      \begin{tblr}{colspec={X[1,c]X[5,l]},cell{even}{1,2} = {azure9},cell{odd}{1,2} = {white}}
        \toprule
        \textbf{Step} & \textbf{Sample} \\ 
        \midrule
        
        3S2 & {RESULTS:\\ \uline{N-back task accuracy (N2 and N3) improved} after real-rTMS (and not after sham-rTMS) compared with baseline ($p=0.029$ and $p=0.015$, respectively), only in patients. \uline{At baseline, patients with MS, compared with HCs, showed higher task-related frontal activation (left DLPFC, N2>NO), which disappeared after real-rTMS.} Task-related (N1 $>$ NO) functional connectivity between the right DLPFC and the right caudate nucleus and bilateral (para) cingulate gyrus increased in patients after real-rTMS when compared with sham stimulation.}\\
        4S1 & {CONCLUSIONS:\\ In patients with MS, N-back accuracy improved while frontal hyperactivation (seen at baseline relative to HCs) disappeared after real-rTMS.}\\
          
        \bottomrule
      \end{tblr}
      }  
    \end{eg}

    \begin{eg}{}
      \par
       \vspace*{10pt}
       {\small\noindent
       \begin{tblr}{colspec={X[1,c]X[5,l]},cell{even}{1,2} = {azure9},cell{odd}{1,2} = {white}}
         \toprule
         \textbf{Step} & \textbf{Sample} \\ 
         \midrule
         
         3S1 & RESULTS: Seventy-nine patients were pre-included, 54 were included, and 34 were F-miso positive, 24 of whom received escalated doses of up to 86 Gy.\\
         3S2 & \uline{The response rate at 3 months was 31/54} (57\%, 95\% confidence interval [43\%--71\%]). DFS and OS at 1 year were 0.86[0.77--0.96] and 0.63[0.49--0.74], respectively. DFS was longer in the F-miso negative patients ($P=0.004$). The RT dose was not associated with DFS when adjusting for the F-miso status. \uline{One toxic death (66 Gy) and 1 case of grade 4 pneumonitis ($>66$ Gy) were reported.}\\
         4S1 & Our approach results in a response rate $>40\%$ with acceptable toxicity.\\
           
         \bottomrule
       \end{tblr}
       }   
    \end{eg}

    \subsubsection{Language Realizations}

    As in 3S2, simple past tense is usually adopted here to present the findings of the study, which is locally true.

    \begin{eg}{}
      Conclusions and Relevance Individuals with ADHD \uline{had} higher scores on the higher education entrance tests during periods they were taking ADHD medication vs nonmedicated periods.
    \end{eg}

    \begin{eg}{}
      Conclusion combinations of eye diseases \uline{were} frequent at old age. 
    \end{eg}

    \subsubsection{Lexical Chunks}

    \begin{enumerate}
      \item was associated with a
      \begin{eg}{}
        Depression severity \uline{was associated with a} decrease in measures of body composition in older adolescents over a mean of 1. 5 years, whereas SSRI treatment was positively associated with these outcomes, with differential effects across treatment groups.
      \end{eg}
      \item was not associated with
      \begin{eg}{}
        In this large population-based cohort study, HG \uline{was not associated with} an increased risk of long-term all-cause mortality.
      \end{eg}
    \end{enumerate}

  \subsection{Step 2(4S2) Indicating limitations}
    \subsubsection{Step Analysis}

    Though often listed in the research articles to add credibility, this step is seldom seen in abstracts. Major limitations might include the scope of the research; the ages, races or genders of participants; unknown factors, such as existing medical conditions; and researcher bias.

    \begin{eg}{}
      Limitations Observational studies do not establish cause and effect.
    \end{eg}

    \subsubsection{Language Realizations}

    Words indicating a contrast or negative meanings such as ``not'' or ``lack'' are often seen as the signals of this step.

    \begin{eg}{}
      In this trial, we provide results for 3 years of treatment, with the limitation that withdrawn individuals were \uline{not} followed up after discontinuation.   
    \end{eg}

    \begin{eg}{}
      A major limitation of this study is the \uline{lack} of histopathologic proof in most patients.     
    \end{eg}

  \subsection{Step 3(4S3) Stating the significance of the results}
    \subsubsection{Step Analysis}

    This is an essential step in M4, providing the readers with the impact that the study might have on future ones or in the relevant field, giving recommendations about possible applications of the findings, stating whether the driven hypotheses are proved or denied by the results, comparing the results with those of the previous studies, or indicating the breakthroughs made by this study.

    \begin{eg}{}
      These findings suggest that ADHD medications may help ameliorate educationally relevant outcomes in individuals with ADHD.
    \end{eg}

    \subsubsection{Language Realizations}

    Simple past and simple present tense can both be employed in this step, the latter being more frequently used, due to which the conclusion would seem more universally true.
    
    The word ``suggest'' which is commonly seen in this step suggests that the authors are interpreting the results of the research.

    \begin{eg}{}
      CONCLUSIONS: These results supported the hypothesis and \uline{suggest} that DSW is an effective exercise intervention for elderly obese women to improve their health and fitness.
    \end{eg}
    
    The chunk ``this is the first'' is often used in this step to express affirmation of the research.

    \begin{eg}{}
      To our knowledge, \uline{this is the first} observational quality improvement initiative in otolaryngology to study the operative flow of a specific procedure and provide insight into areas of patient risk and opportunities for improvement in efficiency.  
    \end{eg}
    
    In addition, the modal verb ``may'' is found widely used at 4S3, which to some degree might weaken the authority of the author's claims and provide the readers with more room for discussion. This, in turn, would protect the conclusion drawn by the authors from being questioned or negated.

    \begin{eg}{}
      These results \uline{may} inform future childhood cancer treatment protocols and SMN surveillance guidelines for CCSs.
    \end{eg}

    \begin{eg}{}
      CONCLUSIONS: PCATS serves as a useful, and valid, predictor of ASA PS classification. Thus, it \uline{may} also serve as a tool to triage patients to an appropriate venue for preoperative assessment that can be utilized by nonclinical schedulers. Using a simple tool such as PCATS may help streamline the presurgical patient experience and improve clinic staff utilization.
    \end{eg}

    \begin{task}{\heiti Corpus-based task}
      What other modal verbs are used in this step?
    \end{task}

    \subsubsection{Lexical Chunks}

    \begin{enumerate}
      \item the results of this study
      \begin{eg}{}
        \uline{The results of this study} confirm previous research findings that there is gadolinium deposition in wider distribution throughout the brain.
      \end{eg}

      \item this is the first
      \begin{eg}{}
        To the best of our knowledge, \uline{this is the first} study that investigates the effect of laser acupuncture in SAIS.  
      \end{eg}

      \item these findings suggest that
      \begin{eg}{}
        \uline{These findings suggest that} adaptive WMT may be an effective adjunctive therapy for WM deficits in HIV participants.  
      \end{eg}

      \item our/these results suggest that
      \begin{eg}{}
        \uline{Our results suggest that} parental pesticide exposure before or during pregnancy may play a role in the development of childhood retinoblastoma. 
      \end{eg}

      \item can be used to
      \begin{eg}{}
        Exhaled breath \uline{can be used to} detect recent cannabis exposure.
      \end{eg}

      \item for the treatment of
      \begin{eg}{}
        Combined RF/PDL technology is promising \uline{for the treatment of} recalcitrant PWS. 
      \end{eg}

      \item has the potential to
      \begin{eg}{}
        Reducing indoor tanning \uline{has the potential to} reduce melanoma incidence, mortality, and treatment costs. 
      \end{eg}

    \end{enumerate}

  \subsection{Step 4(4S4) Predicting future studies}
    \subsubsection{Step Analysis}

    This step predicts the future research direction and points out the problems remaining to be solved. The authors might try to explain to the readers what more could be done or what are the next actions to take.

    \subsubsection{Language Realizations}

    Words of deontic modality, which convey a degree of obligation and necessity, such as ``should'' or ``need'', are often used in this step to highlight the authors' wishes that further studies be done in the future.

    \begin{eg}{}
      Further research \uline{is needed} to assess the clinical importance of these differences and measure longer-term associations.  
    \end{eg}

    \begin{eg}{}
      Larger trials \uline{should} further elucidate the effect of changes in epidermal AQP-3 expression in development of vitiligo, and that might pave the road for discovering new therapeutic modalities for the disease. 
    \end{eg}

    \subsubsection{Lexical Chunks}

    \begin{enumerate}
      \item further research is needed to
      \begin{eg}{}
        \uline{Further research is needed to} assess generalizability and cost-effectiveness of this intervention and to understand which components may have contributed most to the outcome. 
      \end{eg}

      \item studies are needed to
      \begin{eg}{}
        Further prospective \uline{studies are needed to} establish which patients benefit from imaging.  
      \end{eg}

    \end{enumerate}

\section{Sample Reading}

Two samples of M4 are presented here, followed by a detailed analysis on the language features of the steps included.

\begin{sample}[label={myautocounter}]{\heiti}

  \textbf{CONCLUSIONS} 
  
  In patients with MS, N-back accuracy improved while frontal hyperactivation (seen at baseline relative to HCs) disappeared after real-rTMS. Together with the changes in functional connectivity after real-rTMS in patients, these findings may represent an rTMS-induced change in network efficiency in patients with MS, shifting patients' brain function towards the healthy situation. This implicates a potentially relevant role for rTMS in cognitive rehabilitation in MS.

  
  \begin{flushright}
    ---rTMS affects working memory performance, brain activation and functional connectivity in patients with multiple sclerosis. \emph{Journal of Neurology, Neurosurgery, and Psychiatry (2017)}
  \end{flushright}

  \tcblower

  \noindent \textbf{STEP IDENTIFICATION}

  \vspace*{10pt}
  {\small\noindent
  \begin{tblr}{colspec={X[1,c]X[5,l]},cell{even}{1,2} = {azure9}}
    \toprule
    \textbf{Step} & \textbf{Sample} \\ 
    \midrule
  
    4S1 & CONCLUSIONS: In patients with MS, N-back accuracy improved while frontal hyperactivation (seen at baseline relative to HCs) disappeared after real-rTMS. \\
    4S3 & Together with the changes in functional connectivity after real-rTMS in patients, these findings may represent an rTMS-induced change in network efficiency in patients with MS, shifting patients' brain function towards the healthy situation. This implicates a potentially relevant role for rTMS in cognitive rehabilitation in MS. \\
    
    \bottomrule
  \end{tblr}
  }

  \noindent \textbf{ANALYSIS}

  The sample above first restates the pivotal findings of the research in simple past tense to indicate that the findings are locally true and then points out its significance in simple present tense to indicate that the significance is universally true. In 4S3, the expression ``$\dots$ may represent $\dots$'' reveals the possible effect of real-rTMS which ``implicates a potentially relevant role'' for it. The word ``implicate'' means to show or to claim.
  
\end{sample}

\begin{sample}[label={myautocounter}]{\heiti}
  
  \textbf{CONCLUSIONS} 
  
  At 12 months, both FES and AFOs continue to demonstrate equivalent gains in gait speed. Results suggest that long-term FES use may lead to additional improvements in walking endurance and functional ambulation; further research is needed to confirm these findings.


  \begin{flushright}
    ---Long-Term Follow-up to a Randomized Controlled Trial Comparing Peroneal Nerve Functional Electrical Stimulation to an Ankle Foot Orthosis for Patients With Chronic Stroke. \emph{Neurorehabilitation \& Neural Repair (2015)}
  \end{flushright}

  \tcblower

  \noindent \textbf{STEP IDENTIFICATION}

  \vspace*{10pt}
  {\small\noindent
  \begin{tblr}{colspec={X[1,c]X[5,l]},cell{even}{1,2} = {azure9}}
    \toprule
    \textbf{Step} & \textbf{Sample} \\ 
    \midrule
  
    4S1 & CONCLUSIONS: At 12 months, both FES and AFOs continue to demonstrate equivalent gains in gait speed. \\
    4S3 & Results suggest that long-term FES use may lead to additional improvements in walking endurance and functional ambulation; \\
    4S4 & further research is needed to confirm these findings. \\
    
    \bottomrule
  \end{tblr}
  }

  \noindent \textbf{ANALYSIS}

  The first sentence restates the most important result of the research. Different from other examples, simple present tense is used in this case to indicate the ongoing situation. The first half of the second sentence presents the significance of the results in simple present tense, with the words ``suggest'' and ``may'', which are very often used in M4 to interpret the findings. At the end of the sample, prediction of the future studies is made with the use of the chunk ``further research is needed to $\dots$''.
  
\end{sample}

\section{Glossary}

{\small
\begin{longtblr}[
    caption = {Glossary of Chapter 6},
    label = {tab:Glossary of Chapter 6},
    % note{a} = {英文论文中指代当前文献中的差距、问题或缺陷。即现有研究尚未解决的部分。},
]{
    width = \textwidth,
    colspec = {X[1,l,h]  X[1,l,h]  X[3,l,h]},
    rowhead = 1, rowfoot = 0, % 每个分页里表头表尾的数量
    % row{odd} = {blue8}, 
    row{even} = {azure9},
}
    
\toprule
\textbf{WORDS} & \textbf{MEANING} & \textbf{MEANING or EXAMPLE}\\
\midrule

{\textbf{bias}\\/\textipa{"baI@s}/} & \emph{n.} 偏倚 & mental tendency or inclination, esp an irrational preference or prejudice \\
{\textbf{deontic modality}\\/\textipa{di:"6ntIkm@U"d\ae lItI}/} & \emph{n.} 义务情态;道义情态;性情态 & Deontic modality refers to the instances when the speaker orders, promises or places an obligation to someone. \\

\bottomrule

\end{longtblr}
}

\begin{problemset}
  \item \textbf{Find out how many steps are involved in the following examples.}
  
  \hspace*{2em}4S1 Reiterating pivotal results

  \hspace*{2em}4S2 Indicating limitations 
  
  \hspace*{2em}4S3 Stating the significance of the results

  \hspace*{2em}4S4 Predicting future studies

  \begin{enumerate}
    \item Conclusions: The Short model performed well during continuous infusion up to 545 min. This model might be preferable for target-controlled infusion for long-duration anaesthesia in children.
    \item Conclusions: Using an entrustment scoring system, where supervisors score trainees on the level of supervision required, mini-CEX scores demonstrated moderate reliability within a feasible number of assessments, and evidence of validity. When scores were adjusted against an expected standard, underperforming trainees could be identified, and reliability much improved. Taken together with other evidence on trainee ability, the mini-CEX is of sufficient reliability for inclusion in high stakes decisions on trainee progression towards independent specialist practice.
    \item Conclusions: This is the first study, to our knowledge, that demonstrates that patients with advanced laryngeal cancer with 1 or more comorbidities are more likely to receive surgery than chemoradiation compared with patients without any comorbidity, independent of numerous clinical and nonclinical variables among a large national cohort. A limitation of this study is the use of comorbidity data from the National Cancer Database, which gathers its information from hospital discharge face sheets. We recognize that the National Cancer Database may be an imperfect system for the collection of comorbidity data and encourage discussion on different methods to improve the system, including incorporating comorbidity data from the Surveillance, Epidemiology, and End Results Medicare Database and medical chart-based comorbidity data collection by cancer registrars.
    \item Interpretation: Home-based CIMT can enhance the perceived use of the stroke-affected arm in daily activities more effectively than conventional therapy, but was not superior with respect to motor function. Further research is needed to confirm whether home CIMT leads to clinically significant improvements and if so to identify patients that are most likely to benefit.
  \end{enumerate}

  \item \textbf{Rearrange the sentences in correct order.}
  \begin{enumerate}
    \item \uline{\hspace*{3em}}
    
    Conclusions: 
    \begin{itemize}
      \item[A.] These findings suggest vitamin D supplementation may have a therapeutic role in the treatment of MS.
      \item[B.] However, there is uncertainty with regards to the most appropriate dose, with higher dose potentially being associated with worse outcomes.
      \item[C.] There remains the need for a well performed randomised, dose-comparison, placebo-controlled trial of vitamin D in MS.
    \end{itemize}

    \item \uline{\hspace*{3em}}
    
    Conclusions: 
    \begin{itemize}
      \item[A.] Our study was not focused on outcomes or efficacy therefore further research is needed on whether tele-epilepsy provides equal outcomes in care.
      \item[B.] Despite significantly higher face-to-face patient satisfaction, tele-epilepsy patients reported high levels of satisfaction with the tele-epilepsy model of care.
      \item[C.] As such, tele-epilepsy is a viable model for the delivery of epileptic care within regional settings.
    \end{itemize}

    \item \uline{\hspace*{3em}}
    
    Conclusions: 
    \begin{itemize}
      \item[A.] The low seroconversion rates seen after the first year of exposure suggests that less frequent JCV testing of natalizumab treated patients may be safe.
      \item[B.] In contrast to recently published data from colleagues in France and Germany, in our longitudinal cohort study rates of durable positive JCV seroconversion correlate closely with expected background rates of JCV seroconversion in the general population.
      \item[C.] In addition, JCV seroconversion and positive serostatus is not increased in patients with prior exposure to immunosuppressive agents.
    \end{itemize}

    \item \uline{\hspace*{3em}}
    
    Conclusions: 
    \begin{itemize}
      \item[A.] We suggest that controlling the inflammation, even using higher doses of systemic and topical corticosteroids, is of importance in preventing ocular complications, such as cataract.
      \item[B.] In this study, we found that development of cataract is common among pediatric eyes with uveitis and is most strongly related to the extent of inflammation recurrences and ocular complications.
    \end{itemize}

  \end{enumerate}
  
  \item \textbf{Fill in the blanks with appropriate words or the correct form of the verbs given.}
  \begin{enumerate}
    \item Partial least squares regression showed that alcohol-induced changes in mood/drug effects \uline{\hspace*{3em}}(associate) with changes in thalamic FCD in both groups.
    \item A major limitation of this study is the \uline{\hspace*{3em}} of histopathologic proof in most patients.
    \item Our findings also \uline{\hspace*{3em}}(suggest) that the association between social engagement and mental health varies by type of engagement and initial depression level.
    \item The \uline{\hspace*{3em}} of this study indicate that activation of latent HPV infections may contribute to the increased risk of HPV-related (pre) malignant lesions in female RTRs.
    \item These study findings indicate that masitinib is an effective and well tolerated agent \uline{\hspace*{3em}} the treatment of severely symptomatic indolent or smouldering systemic mastocytosis.
    \item Further research is \uline{\hspace*{3em}} to assess the clinical importance of these differences and measure longer-term associations.
  \end{enumerate}
\end{problemset}

\restoregeometry

\chapter*{Key to Exercises}

\paragraph*{Chapter 1}\par
\begin{enumerate}
  \item \textbf{Identify whether the following abstracts are structured or unstructured and tell the reasons.}
  
  \textbf{Abstract 1}
  
  \hspace*{2em}structured abstract 
  
  \textbf{Abstract 2}

  \hspace*{2em}unstructured abstract 
  
  \textbf{Abstract 3}

  \hspace*{2em}structured abstract
\end{enumerate}

\paragraph*{Chapter 2}\par
\begin{enumerate}
  \item \textbf{Identify the four moves in the abstracts.}
  
  \textbf{Abstract 1}

  \hspace*{2em}M1: Objective 

  \hspace*{2em}M2: Design, Setting, and Methods 

  \hspace*{2em}M3: Results M4: Conclusion 
  
  \textbf{Abstract 2}

  \hspace*{2em}M1: Skeletal muscle... walking speed.

  \hspace*{2em}M2: Muscle mitochondrial.. respectively.

  \hspace*{2em}M3: In multivariate linear regression analyses,... four walking tasks.
  
  \hspace*{2em}M4: This is the first.. through this mechanism.
  
  \textbf{Abstract 3}
  
  \hspace*{2em}M1: Propacetamol... in ED patients.
  
  \hspace*{2em}M2: This was a... vasopressor administration.
  
  \hspace*{2em}M3: Postinfusion SBP and DBP... hemodynamic change.
  
  \hspace*{2em}M4: Administration of propacetamol.. propacetamol.
  
  \item \textbf{Reorder the four moves in the abstracts.}

  \textbf{Abstract 1}: 2-1-4-3
  
  \textbf{Abstract 2}: 3-1-2-4

  \textbf{Abstract 3}: 4-2-1-3

\end{enumerate}

\paragraph*{Chapter 3}\par
\begin{enumerate}
  \item \textbf{Find out how many steps are involved in the following examples.}
  \begin{enumerate}
    \item 1S1+1S2
    \item 1S2+1S3a
    \item 1S1+1S3b
    \item 1S1+1S2+1S3a
  \end{enumerate}

  \item \textbf{Rearrange the sentences in correct order.}
  \begin{enumerate}
    \item ACB
    \item ACB
    \item CBA
    \item CBA
  \end{enumerate}

  \item \textbf{Fill in the blanks with appropriate words or the correct form of the verbs given.}
  \begin{enumerate}
    \item tested
    \item would improve
    \item To investigate
    \item aimed
    \item report
    \item have proposed
    \item sought
    \item however
    \item remain
    \item unknown/unclear/not known
    \item have found; have compared; aimed to
    \item is; have shown; has not (yet) been; compared
    \item are; have been (previously) shown; are;To compare
  \end{enumerate}

  \item \textbf{Discussion:Read and the following examples and discuss with your partners about the reasons why modal auxiliary verbs and words indicating likelihood are used?}
  
  略
\end{enumerate}

\paragraph*{Chapter 4}\par
\begin{enumerate}
  \item \textbf{With the ``Steps in M2'' offered, try to identify the steps in the following 2 examples of M2 and write down the corresponding numbers in the blanks.}
  
  \textbf{Example 1}\par
  \vspace*{10pt}
  {\small\noindent
  \begin{tblr}{colspec={X[1,c]X[5,l]},cell{even}{1,2} = {azure9}}
    \toprule
    \textbf{Step} & \textbf{Sample} \\ 
    \midrule
    
    2S2 & DESIGN: Retrospective case-control study.\\
    2S3 & SETTING: Nijmegen, the Netherlands.\\
    2S3 & PATIENTS: Thirty consecutive patients with SSD.\\
    2S4 & INTERVENTIONS: Patients received a trial with a BCD headband as part of the regular workup for SSD. The patients were divided into 2 groups according to their decision to opt for a BCD (BCD+) or not (BCD-). \\
    2S5 & MAIN OUTCOME MEASURES: Patients completed a questionnaire on satisfaction with the BCD headband, patient-and BCD-related factors, and benefit in listening situations. \\
      
    \bottomrule
  \end{tblr}
  }

  \textbf{Example 2}\par
  \vspace*{10pt}
  {\small\noindent
  \begin{tblr}{colspec={X[1,c]X[5,l]},cell{even}{1,2} = {azure9}}
    \toprule
    \textbf{Step} & \textbf{Sample} \\ 
    \midrule
    
    2S3 with 2S2 embedded & (2S3) Eighteen adults with first-ever chronic monohemispheric subcortical stroke participated in (2S2) this randomized, controlled, triple-blinded trial. \\
    2S4 & Intervention consisted of priming with real or sham iTBS to the ipsilesional primary motor cortex immediately before 45 minutes of upper limb physical therapy, daily for 10 days. \\
    2S5 & Changes in upper limb function (Action Research Arm Test [ARAT]), upper limb impairment (Fugl-Meyer Scale), and corticomotor excitability, were assessed before, during, and immediately, 1 month and 3 months after the intervention. Functional magnetic resonance images were acquired before and at one month after the intervention. \\
      
    \bottomrule
  \end{tblr}
  }

  \item \textbf{Fill in the blanks with the correct form of the verbs given. Notice the tense and voice.}
  \begin{enumerate}
    \item ~
    \begin{enumerate}
      \item included
      \item were
      \item completed
      \item were identified
      \item was applied
    \end{enumerate}
    
    \item ~
    \begin{enumerate}
      \item is/was
      \item were matched
      \item were performed
      \item to compare
      \item included
    \end{enumerate}
  \end{enumerate}

  \item \textbf{Error Correction}
  \begin{enumerate}
    \item randomized
    \item with
    \item were ($\wedge$)
    \item criteria
    \item screened
    \item as (/)
    \item median
    \item were
    \item an
    \item $\surd$
  \end{enumerate}
\end{enumerate}

\paragraph*{Chapter 5}\par
\begin{enumerate}
  \item \textbf{Fill the blanks with the correct forms of the given words. Notice the tense and voice.}
  \begin{enumerate}
    \item ~
    \begin{enumerate}
      \item did not differ
      \item received
      \item showed
      \item increased
      \item decreased
      \item was seen
    \end{enumerate}
    
    \item ~
    \begin{enumerate}
      \item was
      \item ate
      \item was halved
      \item was associated
      \item were not observed
    \end{enumerate}
  \end{enumerate}

  \item \textbf{``Significance'' and ``difference'' are two of the key words in M3, so we should be familiar with their different forms. Please fill in the blanks with the proper forms of the words given.}
  \begin{enumerate}
    \item ~
    \begin{enumerate}
      \item significantly
      \item significant
      \item significance
      \item significant
      \item significantly
      \item significant
    \end{enumerate}

    \item ~
    \begin{enumerate}
      \item different
      \item Differences
      \item differently
      \item differed
      \item different
      \item differed
      \item differently
      \item differ
    \end{enumerate}
  \end{enumerate}
\end{enumerate}

\paragraph*{Chapter 6}\par
\begin{enumerate}
  \item \textbf{Find out how many steps are involved in the following examples.}
  \begin{enumerate}
    \item 4S1+4S3
    \item 4S1+4S3
    \item 4S3+4S2+424
    \item 4S3+4S4
  \end{enumerate}

  \item \textbf{Rearrange the sentences in correct order.}
  \begin{enumerate}
    \item ABC
    \item BCA
    \item BCA
    \item BA
  \end{enumerate}

  \item \textbf{Fill in the blanks with appropriate words or the correct form of the verbs given.}
  \begin{enumerate}
    \item were associated
    \item lack
    \item suggest
    \item results/findings
    \item for
    \item needed
  \end{enumerate}
\end{enumerate}


\appendix

\chapter{Four-word and Longer Lexical Bundles in Moves}

{\small
\begin{longtblr}[
    caption = {Four-word and Longer Lexical Bundles in Moves},
    label = {tab:Four-word and Longer Lexical Bundles in Moves},
    remark{Note 1} = {structures (Str), functions (Fun), noun based bundles (N), preposition based bundles (P), verb based bundles (V), otherbundles (O), research based bundles (R), text based bundles (T), stance based bundles (S)},
    remark{Note 2} = {Lexical bundles that appear more than once are in bold.}
]{
    width = \textwidth,
    colspec = {X[12,l,h]  X[7,c,h]  X[7,c,h] X[1,c,h] X[1,c,h]},
    rowhead = 1, rowfoot = 0, % 每个分页里表头表尾的数量
    % row{odd} = {blue8}, 
    row{even} = {azure9},
}
    
\toprule
\textbf{Bundles} & \textbf{Tokens} & \textbf{Dispersion} & \textbf{Str} & \textbf{Fun} \\ 
\midrule
\SetCell[c=5]{c,white}\textbf{M1 Bundles (60 types)} & & \\ 
\midrule
the aim of this study was to evaluate the & 20 & 20 & V & R \\
the aim of this sudy was to investigate & 31 & 31 & V & R \\
the aim of this study was to assess & 23 & 23 & V & R \\
the aim of this study was to & 160(86   only this bundle) & 160(86   only this bundle) & V & R \\
the purpose of this study was to & 106 & 106 & V & R \\
the objective of this study was to & 46 & 46 & V & R \\
of this study was to detemine the & 26 & 26 & V & R \\
of this study was to investigate & 54(23   only this bundle) & 54(23   only this bundle) & V & R \\
of this study was to determine & 48(22   only this bundle) & 48(22   only this bundle) & V & R \\
of this study was to evaluate & 44(22   only this bundle) & 44(22   only this bundle) & V & R \\
of this study was to compare & 31 & 31 & V & R \\
of this study was to examine & 25 & 25 & V & R \\
aim of the study was to & 20 & 20 & V & R \\
of this study was to & 352(49   only this bundle) & 352(49   only this bundle) & V & R \\
study was to evaluate the & 47 & 47 & V & R \\
study was to invstigate the & 44 & 44 & V & R \\
of the study was to & 41(21   only this bundle) & 41(21   only this bundle) & V & R \\
the efficacy and safety of & 37 & 37 & N & R \\
study was to asses the & 36 & 36 & V & R \\
。 the safety and efficacy of & 32 & 32 & N & R \\
study was to determine the & 29 & 29 & V & R \\
of this study were to & 28 & 28 & V & R \\
we aimed to assess the & 27 & 27 & V & R \\
of this study is to & 26 & 26 & V & R \\
little is known about the & 23 & 23 & V & T \\
study was to compare the & 22 & 22 & V & R \\
to detemine the effect of & 21 & 21 & O & R \\
to examine the association between & 20 & 20 & O & R \\
the aim of this & 205(45   only this bundle) & 205(45   only this bundle) & N & R \\
aim of this study & 189(29   only this bundle) & 189(29   only this bundle) & N & R \\
the purpose of this & 127(21   only this bunde) & 127(21   only this bundle) & N & R \\
study was to invesigate & 71(27   only this bundle) & 71(27   only this bundle) & V & R \\
study was to evaluate & 67(20   only this bundle) & 67(20   only this bundle) & V & R \\
this sudy aimed to & 67 & 67 & V & R \\
study was to assess & 53(30   only this bundle) & 53(30   only this bundle) & V & R \\
we aimed to assess & 53(26   only this bundle) & 53(26   only this bundle) & V & R \\
little is known about & 49(26   only this bundle) & 49(26   only this bundle) & V & T \\
of the study was & 41(20   oly this bundle) & 41(20   only this bundle) & V & R \\
the study was to & 41(20   only this bundle) & 41(20   only this bundle) & V & R \\
study was to compare & 40 & 40 & V & R \\
the aim of the & 40 & 40 & N & R \\
for the treatment of & 34 & 34 & P & R \\
in patients with advanced & 28 & 26 & P & R \\
in the emergency department & 27 & 27 & P & R \\
our objective was to & 27 & 27 & V & R \\
in the treatment of & 26 & 26 & P & R \\
has been associated with & 25 & 25 & V & T \\
has been shown to & 25 & 25 & V & T \\
our aim was to & 25 & 25 & V & R \\
we sought to determine & 24 & 24 & V & R \\
aimed to determine the & 23 & 23 & V & R \\
aimed to investigate the & 23 & 23 & V & R \\
was to determine whether & 23 & 23 & V & R \\
aimed to evaluate the & 21 & 21 & V & R \\
we aimed to evaluate & 21 & 21 & V & R \\
is one of the & 20 & 20 & V & T \\
study aimed to determine & 20 & 20 & V & R \\
tested the hypothesis that & 20 & 20 & V & R \\
the objective of the & 20 & 20 & N & R \\
was to examine the & 20 & 20 & V & R \\
\midrule
\SetCell[c=5]{c,white}\textbf{M1 Bundles (50 types)} & & \\ 
\midrule
all   patiens who recived at least one dose of & 21 & 21 & O & R \\
patients were randomly assigned to receive & 26 & 26 & V & R \\
the primary end point was the & 20 & 20 & V & R \\
this trial is regisered with & 76 & 76 & V & R \\
\textbf{were randomly assigned to receive} & 65(39   only this bundle) & 65(39   only this bundle) & V & R \\
the primary end point was & 59(39   only this bundle) & 58(38   only this bundle) & V & R \\
patients were randomly assigned to & 51(25   only this bundle) & 50(24   only this bundle) & V & R \\
the primary outcome was the & 44 & 44 & V & R \\
this study is regisered with & 44 & 44 & V & R \\
the primary endpoint was the & 33 & 33 & V & R \\
the primary outcome measure was & 31 & 31 & V & R \\
a retrospective cohort study of & 26 & 26 & N & R \\
was the proportion of patients & 24 & 23 & V & R \\
logisic regression was used to & 23 & 23 & V & R \\
study was approved by the & 22 & 22 & V & R \\
the primary oucome was & 187(143   only this bundle) & 186(142   only this bundle) & V & R \\
were randomly assigned to & 145(80   only this bundle) & 141(77   only this bundle) & V & R \\
the primary endpoint was & 117(84   only this bundle) & 115(82   only this bundle) & V & R \\
a retropective cohort sudy & 54(28   only this bundie) & 54(28   only this bundle) & N & R \\
were included in the & 46 & 44 & V & R \\
at the time of & 44 & 43 & P & R \\
on the basis of & 42 & 41 & P & T \\
were randomized to receive & 41 & 37 & V & R \\
were masked to treatment & 39 & 39 & V & R \\
models were used to & 36 & 35 & V & R \\
analysis was used to & 34 & 32 & V & R \\
we conducted a retrospecive & 34 & 34 & V & R \\
was used to identify & 30 & 30 & V & R \\
were included in this & 28 & 28 & V & R \\
at the end of & 27 & 25 & P & R \\
at a dose of & 26 & 22 & P & R \\
was assessed using the & 26 & 25 & V & R \\
years or older with & 26 & 26 & O & R \\
patients were randomized to & 25 & 23 & V & R \\
was used to assess & 25 & 24 & V & R \\
were used to assess & 25 & 25 & V & R \\
we randomly assigned patients & 25 & 25 & V & R \\
masked to treatment allocation & 24 & 24 & V & R \\
participants were randomly assigned & 24 & 24 & V & R \\
analyses were performed to & 23 & 23 & V & R \\
outcome measure was the & 23 & 23 & V & R \\
were used to detemine & 23 & 23 & V & R \\
masked to treatment assignment & 22 & 21 & V & R \\
data were collected from & 21 & 21 & V & R \\
in all patients who & 21 & 20 & P & R \\
a cross sectional study & 20 & 20 & N & R \\
a secondary analysis of & 20 & 20 & N & R \\
regression analysis was used & 20 & 20 & V & R \\
secondary end points included & 20 & 20 & V & R \\
were used to evaluate & 20 & 20 & V & R \\
outcome measure was the & 23 & 23 & V & R \\
were used to detemine & 23 & 23 & V & R \\
masked to treatment asignment & 22 & 21 & V & R \\
daa were cllcted from & 21 & 21 & V & R \\
\midrule
\SetCell[c=5]{c,white}\textbf{M3 Bundles (38 types)} & & \\ 
\midrule
there was no significat dfference in & 25 & 23 & V & T \\
there were no significant differences in & 24 & 24 & V & T \\
there were no significant differences between & 22 & 21 & V & T \\
there was no difference in & 34 & 31 & V & T \\
adverse events were reported in & 33 & 28 & V & R \\
there were no differences in & 30 & 28 & V & T \\
paticnts were randomly asigned to & 30 & 25 & V & R \\
serious adverse events were reported & 29 & 28 & V & R \\
\textbf{were randomly assigned to receive} & 26 & 22 & V & R \\
were more likely to have & 25 & 21 & V & S \\
did not differ significantly between & 24 & 24 & V & T \\
patiets were included in the & 24 & 24 & V & R \\
did not differ between the & 24 & 22 & V & T \\
at a median follow-up of & 21 & 21 & P & R \\
were incuded in the study & 20 & 20 & V & R \\
in the placebo group & 162 & 76 & P & R \\
in the control group & 117 & 72 & P & R \\
\textbf{was associated with a/an} & 131 & 117 & V & T \\
\textbf{were included in the} & 102(78   only this bundle) & 94(70   only this bundle) & V & R \\
were more likely to & 82(57   only this bundle) & 68(47   only this bundle) & V & S \\
there was no significant & 76(51   only this bundle) & 70(47   only this bundle) & V & T \\
there were no significant & 68(44   only this bundle) & 67(43   only this bundle) & V & T \\
were randomly assigned to & 68(38   only this bundle) & 61(33   only this bundle) & V & R \\
a media fllow-up of & 55(34   only this bundle) & 55(34   only this bundle) & N & R \\
did not dfffe betwe & 52(28   only this bundle) & 47(25   only this buodle) & V & T \\
\textbf{was not associated with} & 52 & 47 & V & T \\
at the time of & 51 & 46 & P & R \\
was signifcantly associated with & 44 & 42 & V & T \\
was signifcantly higher in & 42 & 39 & V & T \\
area under the curve & 40 & 37 & N & R \\
during the study period & 40 & 37 & P & R \\
the most common grade & 40 & 35 & N & R \\
were less likely to & 40 & 35 & V & S \\
events were reported in & 37 & 30 & V & T \\
compared with the control & 33 & 26 & V & R \\
were not significantly dfferent & 33 & 32 & V & T \\
between the two gropgs & 31 & 27 & P & T \\
was not significantly dfferent & 31 & 29 & V & T \\
significantly higher in the & 30 & 30 & O & T \\
was found to be & 29 & 28 & V & T \\
was significantly lower in & 29 & 26 & V & T \\
incuded in the analys & 28 & 28 & V & R \\
an increased risk of & 28 & 24 & P & R \\
not significantly dfferent between & 28 & 26 & V & T \\
was the most common & 26 & 25 & V & R \\
were not associated with & 26 & 25 & V & T \\
group compared with the & 26 & 23 & O & R \\
was associated with increased & 25 & 23 & V & T \\
associated with an increased & 25 & 22 & V & T \\
no saisticallly signifcant difference & 24 & 23 & N & T \\
were significantly higbher in & 24 & 24 & V & T \\
sigpificant dffcence in the & 23 & 22 & N & T \\
no significant dffcence between & 23 & 21 & N & T \\
significant dfftences between the & 23 & 21 & N & T \\
was independenly associated with & 23 & 23 & V & T \\
patients were enrolled and & 22 & 22 & V & R \\
were associated with a & 21 & 20 & V & T \\
no sigificat dffrences were & 20 & 20 & V & T \\
\midrule
\SetCell[c=5]{c,white}\textbf{M4 Bundles (18 types)} & & \\ 
\midrule
further research is needed to & 20 & 20 & V & S \\
the results of this study & 20 & 20 & N & T \\
\textbf{was associated with a} & 44 & 42 & V & T \\
this is the first & 39 & 39 & V & S \\
\textbf{was not associated with} & 36 & 36 & V & T \\
these findings suggest that & 32 & 32 & V & T \\
studies are needed to & 28 & 28 & V & S \\
is associated with a & 27 & 26 & V & T \\
our results suggest that & 27 & 27 & V & T \\
these results suggest that & 27 & 27 & V & T \\
can be used to & 26 & 24 & V & S \\
an increased risk of & 22 & 22 & N & R \\
associated with an increased & 22 & 22 & V & T \\
for the treatment of & 20 & 20 & P & R \\
has the potential to & 20 & 20 & V & S \\
our findings suggest that & 20 & 20 & V & T \\ 

\bottomrule

\end{longtblr}
}

\chapter{Four-word and Longer Lexical Bundles in Steps}

\begin{landscape}
  {\tiny
  \begin{longtblr}[
      caption = {Four-word and Longer Lexical Bundles in Steps},
      label = {tab:Four-word and Longer Lexical Bundles in Steps},
  ]{
      width = \linewidth,
      colspec = {X[2,c,h]  X[8,l,h]  X[8,l,h]  X[6,l,h]},
      rowhead = 1, rowfoot = 0, % 每个分页里表头表尾的数量
      % row{odd} = {blue8}, 
      row{even} = {azure9},
  }
      
  \toprule
  \textbf{Move/ Step} & \textbf{Communicative functions} & \textbf{Specific bundles} & \textbf{Common bundles}\\
  \midrule
  M1 & Creating a research territory/ space &  & \\
  1S1 & Presenting current knowledge or relevant information established by previos studies & {has been asociated with \\
      has been shown to \\
      is one of the} & {for the teatment of \\
      in patients with advanccd \\
      in the emergency depatmen \\
      in the treatment of}\\
  1S2 & Establishing a niche/problem & {little is known about the \\
      little is known about} & \\
  1S3a & Indicating main purposs & {the aim of this study was to evaluate the \\
      the aim of this study was to invesigate \\
      the aim of this study was to asscss \\
      the aim of this study was to \\
      the pupose of this study was to \\
      the objecive of this study was to \\
      of this study was to determine the \\
      of this study was to invsigate \\
      of this sudy was to determine \\
      of this study was to evaluate \\
      of this sudy was to compare \\
      of this study was to examine \\
      aim of the study was to \\
      of this study was to \\
      study was to evaluate the \\
      study was to investigate the \\
      of the study was to \\
      the efficacy and safety of \\
      study was to assess the \\
      the safety and efficacy of \\
      study was to determine the \\
      of this study were to \\
      we aimed to asses the\\
      of this study is to\\
      study was to compare the\\
      to determine the effect of\\
      to examine the association between\\
      the aim of this\\
      aim of this study\\
      the purpose of this\\
      study was to investigate\\
      study was to evaluate\\
      this study aimed to\\
      study was to assess\\
      we aimed to assess\\
      of the study was\\
      the study was to\\
      study was to compare\\
      the aim of the\\
      our objective was to\\
      our aim was to\\
      we sought to determine\\
      aimed to determine the\\
      aimed to investigate the\\
      was to determine whether\\
      aimed to evaluate the\\
      we aimed to evaluate\\
      study aimed to determine\\
      the objective of the\\
      was to examine the} & {for the treatment of \\
      in patients with advanced \\
      in the emergency departnent \\
      in the treatment of}\\
  1S3b & Raising hypotheses & tested the hypothesis that & in patients with advanced\\
  M2 & Describing research process &  & \\
  2S1 & Reporting on medical ethics review & study was approved by the & \\
  2S2 & Explaining briefly research design & {a retrospective cohort study of\\
      a retrospective cohort study\\
      we conducted a retrospective\\
      a cross-sectional study\\
      a secondary analysis of} & \\
  2S3 & Describing subjects or data and their selection criteria & {were included in the\\
      were included in this\\
      years or older with\\
      data were collected from} & {at the time of\\
      on the basis of}\\
  2S4 & Describing experimental procedure, such as interventions, examinations,
    etc. & {patients were randomly assigned to receive\\
      were randomly assigned to receive\\
      patients were randomly assigned to\\
      were randomly assigned to\\
      were randomized to receive\\
      were masked to treatment\\
      patients were randomized to\\
      we randomly assigned patients\\
      masked to treatment allocation\\
      participants were randomly assigned\\
      masked to treatment assignment} & {all patients who received at least one dose of \\
      at the time of \\
      on the basis of \\
      at the end of\\
      at a dose of\\
      was assessed using the\\
      was used to assess\\
      were used to assess\\
      were used to evaluate}\\
  2S5 & Describing main outcomes and their measures & {the primary end point was the\\
      the primary end point was\\
      the primary outcome was the\\
      the primary endpoint was the\\
      the primary outcome measure was\\
      was the proportion of patients\\
      the primary outcome was\\
      the primary endpoint was\\
      outcome measure was the\\
      in all patients who\\
      secondary end points included} & {all patients who received at least one dose of \\
      at the time of \\
      on the basis of \\
      was used to identify\\
      at the end of \\
      at a dose of \\
      was assessed using the\\
      was used to assess \\
      were used to assess\\
      analysis was used to \\
      analyses were performed to \\
      were used to determine\\
      were used to evaluate}\\
  2S6 & Describing data analysis methods & {logistic regression was used to\\
      models were used to\\
      regression analysis was used} & {was used to identify\\
      analysis was used to\\
      analyses were performed to}\\
  2S7 & Reporting on registration information & {this trial is registered with\\
      this study is registered with} & were used to determine\\
  M3 & Summarizing results &  & \\
  3S1 & Explaining information on valid samples & {patients were randomly assigned to\\
      were randomly assigned to receive\\
      patients were included in the\\
      were included in the study\\
      were included in the\\
      included in the analysis\\
      patients were enrolled and} & {at a median follow-up of\\
      in the placebo group\\
      in the control group\\
      a median follow-up of during the study period}\\
  3S2 & Illustrating overall observation or main results & {there was no significant difference in\\
      there were no significant differences in\\
      there were no significant differences between\\
      there was no difference in\\
      adverse events were reported in\\
      there were no differences in\\
      serious adverse events were reported\\
      were more likely to have\\
      did not differ significantly between\\
      did not differ between the\\
      was associated with a/an\\
      were more likely to\\
      there was no significant\\
      there were no significant\\
      were randomly assigned to\\
      did not differ between\\
      was not associated with\\
      at the time of\\
      was significantly associated with\\
      was significantly higher in\\
      area under the curve\\
      the most common grade\\
      were less likely to\\
      events were reported in\\
      compared with the control\\
      were not significantly different between the two groups\\
      was not significantly different\\
      significantly higher in the\\
      was found to be\\
      was significantly lower in\\
      an increased risk of\\
      not significantly different between\\
      was the most common\\
      were not associated with\\
      group compared with the\\
      was associated with increased\\
      associated with an increased\\
      no statistically significant difference\\
      were significantly higher in\\
      significant difference in the\\
      no significant difference between\\
      significant differences between the\\
      was independently associated with\\
      were associated with a\\
      no significant differences were} & \\
  M4 & Drawing conclusions &  & \\
  4S1 & Reiterating pivotal results & {was associated with a\\
      was not associated with} & {is associated with a\\
      an increased risk of \\
      associated with an increased}\\
  4S2 & Indicating limitations &  & \\
  4S3 & Stating the significance of the results & {the results of this study\\
      this is the first\\
      these findings suggest that\\
      our results suggest that\\
      these results suggest that\\
      can be used to\\
      for the treatment of\\
      has the potential to\\
      our findings suggest that} & {is associated with a\\
      an increased risk of \\
      associated with an increased}\\
  4S4 & Predicting future studies & {further research is needed to\\
      studies are needed to} & \\
  
  \bottomrule
  
  \end{longtblr}

  % \begin{longtable} to \textwidth {XXXX}

  %   % \toprule
  %   % Move/ Step & Communicative functions & Specific bundles & Common bundles \\
  %   % \midrule 
  %   % \endfirsthead

  %   % Appear the table header at the top of every page
  %   \toprule
  %   Move/ Step & Communicative functions & Specific bundles & Common bundles \\
  %   \midrule 
  %   \endhead 
  
  %   % Appear \hline at the bottom of every page
  %   \bottomrule
  %   \endfoot 


  %   M1 & Creating a research territory/ space &  &  \\
  %   1S1 & Presenting current knowledge or relevant information established by previos studies & \begin{tabular}[c]{@{}l@{}}has been asociated with \\ has been shown to \\ is one of the\end{tabular} & \begin{tabular}[c]{@{}l@{}}for the teatment of \\ in patients with advanccd \\ in the emergency depatmen \\ in the treatment of\end{tabular} \\
  %   1S2 & Establishing a niche/problem & \begin{tabular}[c]{@{}l@{}}little is known about the \\ little is known about\end{tabular} &  \\
  %   1S3a & Indicating main purposs & \begin{tabular}[c]{@{}l@{}}the aim of this study was to evaluate the \\ the aim of this study was to invesigate \\ the aim of this study was to asscss \\ the aim of this study was to \\ the pupose of this study was to \\ the objecive of this study was to \\ of this study was to determine the \\ of this study was to invsigate \\ of this sudy was to determine \\ of this study was to evaluate \\ of this sudy was to compare \\ of this study was to examine \\ aim of the study was to \\ of this study was to \\ study was to evaluate the \\ study was to investigate the \\ of the study was to \\ the efficacy and safety of \\ study was to assess the \\ the safety and efficacy of \\ study was to determine the \\ of this study were to \\ we aimed to asses the\\ of this study is to\\ study was to compare the\\ to determine the effect of\\ to examine the association between\\ the aim of this\\ aim of this study\\ the purpose of this\\ study was to investigate\\ study was to evaluate\\ this study aimed to\\ study was to assess\\ we aimed to assess\\ of the study was\\ the study was to\\ study was to compare\\ the aim of the\\ our objective was to\\ our aim was to\\ we sought to determine\\ aimed to determine the\\ aimed to investigate the\\ was to determine whether\\ aimed to evaluate the\\ we aimed to evaluate\\ study aimed to determine\\ the objective of the\\ was to examine the\end{tabular} & \begin{tabular}[c]{@{}l@{}}for the treatment of \\ in patients with advanced \\ in the emergency departnent \\ in the treatment of\end{tabular} \\
  %   1S3b & Raising hypotheses & tested the hypothesis that & in patients with advanced \\
  %   M2 & Describing research process &  &  \\
  %   2S1 & Reporting on medical ethics review & study was approved by the &  \\
  %   2S2 & Explaining briefly research design & \begin{tabular}[c]{@{}l@{}}a retrospective cohort study of\\ a retrospective cohort study\\ we conducted a retrospective\\ a cross-sectional study\\ a secondary analysis of\end{tabular} &  \\
  %   2S3 & Describing subjects or data and their selection criteria & \begin{tabular}[c]{@{}l@{}}were included in the\\ were included in this\\ years or older with\\ data were collected from\end{tabular} & \begin{tabular}[c]{@{}l@{}}at the time of\\ on the basis of\end{tabular} \\
  %   2S4 & Describing experimental procedure, such as interventions, examinations, etc. & \begin{tabular}[c]{@{}l@{}}patients were randomly assigned to receive\\ were randomly assigned to receive\\ patients were randomly assigned to\\ were randomly assigned to\\ were randomized to receive\\ were masked to treatment\\ patients were randomized to\\ we randomly assigned patients\\ masked to treatment allocation\\ participants were randomly assigned\\ masked to treatment assignment\end{tabular} & \begin{tabular}[c]{@{}l@{}}all patients who received at least one dose of \\ at the time of \\ on the basis of \\ at the end of\\ at a dose of\\ was assessed using the\\ was used to assess\\ were used to assess\\ were used to evaluate\end{tabular} \\
  %   2S5 & Describing main outcomes and their measures & \begin{tabular}[c]{@{}l@{}}the primary end point was the\\ the primary end point was\\ the primary outcome was the\\ the primary endpoint was the\\ the primary outcome measure was\\ was the proportion of patients\\ the primary outcome was\\ the primary endpoint was\\ outcome measure was the\\ in all patients who\\ secondary end points included\end{tabular} & \begin{tabular}[c]{@{}l@{}}all patients who received at least one dose of \\ at the time of \\ on the basis of \\ was used to identify\\ at the end of \\ at a dose of \\ was assessed using the\\ was used to assess \\ were used to assess\\ analysis was used to \\ analyses were performed to \\ were used to determine\\ were used to evaluate\end{tabular} \\
  %   2S6 & Describing data analysis methods & \begin{tabular}[c]{@{}l@{}}logistic regression was used to\\ models were used to\\ regression analysis was used\end{tabular} & \begin{tabular}[c]{@{}l@{}}was used to identify\\ analysis was used to\\ analyses were performed to\end{tabular} \\
  %   2S7 & Reporting on registration information & \begin{tabular}[c]{@{}l@{}}this trial is registered with\\ this study is registered with\end{tabular} & were used to determine \\
  %   M3 & Summarizing results &  &  \\
  %   3S1 & Explaining information on valid samples & \begin{tabular}[c]{@{}l@{}}patients were randomly assigned to\\ were randomly assigned to receive\\ patients were included in the\\ were included in the study\\ were included in the\\ included in the analysis\\ patients were enrolled and\end{tabular} & \begin{tabular}[c]{@{}l@{}}at a median follow-up of\\ in the placebo group\\ in the control group\\ a median follow-up of during the study period\end{tabular} \\
  %   3S2 & Illustrating overall observation or main results & \begin{tabular}[c]{@{}l@{}}there was no significant difference in\\ \\ there were no significant differences in\\ there were no significant differences between\\ there was no difference in\\ adverse events were reported in\\ there were no differences in\\ serious adverse events were reported\\ were more likely to have\\ did not differ significantly between\\ did not differ between the\\ was associated with a/an\\ were more likely to\\ there was no significant\\ there were no significant\\ were randomly assigned to\\ did not differ between\\ was not associated with\\ at the time of\\ was significantly associated with\\ was significantly higher in\\ area under the curve\\ the most common grade\\ were less likely to\\ events were reported in\\ compared with the control\\ were not significantly different between the two groups\\ was not significantly different\\ significantly higher in the\\ was found to be\\ was significantly lower in\\ an increased risk of\\ not significantly different between\\ was the most common\\ were not associated with\\ group compared with the\\ was associated with increased\\ associated with an increased\\ no statistically significant difference\\ were significantly higher in\\ significant difference in the\\ no significant difference between\\ significant differences between the\\ was independently associated with\\ were associated with a\\ no significant differences were\end{tabular} &  \\
  %   M4 & Drawing conclusions &  &  \\
  %   4S1 & Reiterating pivotal results & \begin{tabular}[c]{@{}l@{}}was associated with a\\ was not associated with\end{tabular} & \begin{tabular}[c]{@{}l@{}}is associated with a\\ an increased risk of \\ associated with an increased\end{tabular} \\
  %   4S2 & Indicating limitations &  &  \\
  %   4S3 & Stating the significance of the results & \begin{tabular}[c]{@{}l@{}}the results of this study\\ this is the first\\ these findings suggest that\\ our results suggest that\\ these results suggest that\\ can be used to\\ for the treatment of\\ has the potential to\\ our findings suggest that\end{tabular} & \begin{tabular}[c]{@{}l@{}}is associated with a\\ an increased risk of \\ associated with an increased\end{tabular} \\
  %   4S4 & Predicting future studies & \begin{tabular}[c]{@{}l@{}}further research is needed to\\ studies are needed to\end{tabular} &  \\
  
  % \end{longtable}
  
  }
\end{landscape}



\backmatter



	
\end{document}